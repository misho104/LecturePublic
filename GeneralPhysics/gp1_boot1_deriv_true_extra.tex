%!TEX program=lualatex
%!TEX options=-synctex=1 -interaction=nonstopmode -halt-on-error "%DOC%.tex"
\def\LectureName/{General Physics}
%\def\LectureNumber/{}
%\def\LectureDate/{}
\PassOptionsToPackage{fleqn}{amsmath}
\PassOptionsToPackage{hyperfootnotes=false}{hyperref}
\documentclass[11pt,pdfa,lastpage]{MishoNote}
\title{\LectureName/: Derivative Boot Camp Extra Session}
\author{Sho Iwamoto}
\hypersetup{
  pdflang={en-US},
  pdfauthortitle={Assistant Professor, National Sun Yat-sen University},
  pdfsubject={An extra session of Derivative Boot Camp, which is a preview of \LectureName/ lectures.},
  pdfcontactemail={iwamoto@g-mail.nsysu.edu.tw},
  pdfcontacturl={https://www2.nsysu.edu.tw/iwamoto/},
  pdfcaptionwriter={Sho Iwamoto},
  pdfcopyright={2024–2025 Sho Iwamoto\textLF This document is licensed under the Creative Commons CC BY–NC 4.0 International Public License.},
  pdflicenseurl={https://creativecommons.org/licenses/by-nc/4.0/},
}

\usepackage{GP}
\setlist{itemsep=4pt}
\makeatletter
\def\thesection{A$\mathbf{'}$}
\def\theenumi{\@Alph\c@enumi$\mathbf{'}$}
\makeatother

\newcommand\INT[2][\relax]{\item$\displaystyle\int#2\,\dd x$\ifx\relax#1\relax\relax\relax\else{\quad#1}\fi}
\newcommand\INP[1]{\INT{\left(#1\right)}}
\newcommand\INC[3]{\item$\displaystyle\int^{#2}_{#1}#3\,\dd x$}
\NewDocumentCommand\DIFF{O{x}om}{\item$\displaystyle\gdiff[\dd][#2]{}{#1}#3$}

\let\origfootnote\footnote
\let\origfootnoterule\footnoterule

\begin{document}%
\title{Derivative Boot Camp Extra Session}\begin{maketitle}
\let\footnote\origfootnote
\let\footnoterule\origfootnoterule

\subsection*{Preface}
Welcome to this extra session!
You, who completed the true form of the Derivative Boot Camp, are now ready to take on more challenging tasks for your engineering studies.
For those who still have the energy and motivation to go deeper, Sho has prepared this extra session of the Boot Camp.

This session aims to complete your knowledge and skills of derivatives to the university standard.
In the previous session, \Emph*{you have learned} the derivatives of
\begin{miniitemize}
  \item $x^a$
  \item $\sin x$ and other trigonometric functions,
  \item $f(x)g(x)$ and $f(x)/g(x)$, and
  \item $f(g(x))$
\end{miniitemize}
In this extra session, \Emph*{you will learn}
\begin{miniitemize}
  \item exponential functions ($a^x$) and their derivatives
  \item logarithmic functions ($\log_a x$ and $\ln x$) and their derivatives, and
  \item hyperbolic functions ($\sinh x$ etc.) and their derivatives.
\end{miniitemize}
By the end, you will be able to calculate ``simple'' derivatives such as
\[
\diff{}{x}\frac{x^2+\tan(\ln x)}{\sinh(2x^2+1)}.
\]

Tasks are classified to $\bigstar$ (minimal), $^{***}$ (basic), $^{**}$ (intermediate), and $^{*}$ (for motivated students).
You don't have to solve all of them, but try to solve as many as you can.
Good Luck!

\enlargethispage{-5em}

\makeatletter
\begin{tikzpicture}[remember picture,overlay]
  \begingroup
  \fontsize{9}{13}\selectfont
    \node[xshift=\@total@leftsep,yshift=25.5mm,anchor=south west,align=left,text width=\textwidth] at (current page.south west) {%
      \href{https://creativecommons.org/licenses/by-nc/4.0/}{\includegraphics[width=2.2cm]{../figs/by-nc.pdf}}\\[.4em]
      \noindent\textsf{\color{gray}%
      Visit \url{https://github.com/misho104/LecturePublic} for further information, updates, and to report issues.}\par
    };
    \node[xshift=-\@total@leftsep+26mm,yshift=32mm,align=left,text width=\textwidth,anchor=south east] at (current page.south east) {%
      \noindent\textsf{\color{gray}%
      This document is licensed under
      \href{https://creativecommons.org/licenses/by-nc/4.0/}{the Creative Commons CC--BY--NC 4.0 International Public License.}\\
      You may use this document only if you do in compliance with the license.}\par
    };
  \endgroup
\end{tikzpicture}
\makeatother

\end{maketitle}
\newpage


\subsection{Highschool Review 1: Exponential Functions}

We want to consider the functions
\[f(x)=b^x\qquad\text{with}\quad b>0.\]
They are called exponential functions.
Let's begin with a review of highschool mathematics.


\ornamentskip

\begin{enumerate}
  \itemA Let $f(x)=2^x$ and $g(x)=0.5^x$.  Calculate the following values, but you can use a calculator \Emph{only for} (5) and (6).
  \begin{menumerate}{2}
    \item $f(0)$, $f(1)$, $f(2)$, $f(10)$.
    \item $g(0)$, $g(1)$, $g(2)$.
    \item $f(-1)$, $f(-3)$, $g(-1)$, $g(-2)$.
    \item $f(1/2)$, $f(-1/2)$, $g(1/2)$, $g(-1/2)$.
    \item $f(1.585)$, $g(1.585)$, $f(1.585\times 2)$.
    \item $f(0.01)$, $f(-0.01)$, $g(0.01)$, $g(-0.01)$.
  \end{menumerate}
  \itemA Simplify the following expressions, where $x$ and $y$ are real numbers, $a>0$, and $b>0$.
  \begin{menumerate}{3}
    \item $4^{2.5}\times 4^{3.5}$
    \item $(a^2)^3$
    \item $a^2\times a^3$
    \item $3^{x}\times 3^{y}$
    \item $\dfrac{7^{x+5}}{7^{2}}$
    \item $1/a^x$
    \item $a^x/a^y$
    \item $\dfrac{6^a}{2^a}$
    \item $\dfrac{(ab)^3}{b^2}$
    \item $a^5\times a^5\times a^5$
    \item $(a^x)^2$
    \item $\dfrac{(a^x)^y}{a^y}$
  \end{menumerate}
  \itemC It is worth noticing that the expression $b^x$ is a bit tricky. Check and be convinced/comfortable with the following facts.
  \begin{itemize}
    \item For positive $b$, the exponential $b^x$ is well-defined for any $x\in\RR$, i.e., $x$ can be positive, negative, fractional, or zero.
    \item If $b$ is zero, $b^x=0^x$ is well-defined only for $x>0$. The value is always zero.
    \item If $b$ is negative, $b^x$ is well-defined only for an integer $x\in\ZZ$.
  \end{itemize}
  Now, choose the well-defined values from the following.
  \begin{multicols}{7}\begin{enumerate}[labelsep=0em,labelwidth=0em,itemindent=0em,label={},leftmargin=0.3em]
   \item $3^{-5}$
   \item $3^{0.2}$
   \item $3^{-\sqrt{3}}$\columnbreak
   \item $(-3)^{-2}$
   \item $(-3)^{2}$
   \item $(-3)^{0.5}$
   \item $(-3)^0$
   \item $0.2^0$
   \item $0.2^5$
   \item $0.2^{-0.5}$
   \item $0.2^{\sqrt{0.2}}$
   \item $(-0.4)^{3}$
   \item $(-0.4)^{\pi}$
   \item $\pi^{0.5}$
   \item $\pi^{-\pi}$
   \item $0^{3}$
   \item $0^{\sqrt{2}}$
   \item $0^{-0.5}$
   \item $(-0.1)^{0}$
   \item $0^{-1}$
   \item $0^{0}$
\end{enumerate}\end{multicols}
\end{enumerate}
\Remark{
After you learn complex analysis, you can define $b^x$ for $b<0$ and any real number $x\in\RR$, which is given by
\[
b^x=|b|^x(-1)^x=|b|^x(\ee^{\ii\pi})^x=|b|^x\ee^{\ii\pi x}.
\]
For example, $(-3)^{0.2} \simeq 1.12+0.81\ii$ and $(-\pi)^{-\sqrt{3}}\simeq 0.092+0.10\ii$.
Meanwhile, there is no good way to define $0^x$ for $x\le 0$. For example, the value of $0^0$ is not well-defined. We sometimes use $0^0\stackrel!=1$ but it is not a global consensus.
}
\newpage
\subsection{Highschool Review 2: Logarithmic Functions}
Logarithmic functions are defined as follows. Notice the allowed range of $a$ and $x$.
\begin{definition}{Logarithm and Logarithmic Functions}{log}
  If a number $x$ satisfies $b^x=A$, then $y$ is called the \Emph{logarithm of $A$ to the base $b$}. We write it by
  \begin{align}
    &\notag x=\log_b A\qquad(A>0,\quad b>0,~b\neq 1),\\
    &\qquad \therefore \quad x=\log_b A\quad\iff\quad b^x=A.
  \end{align}
  For $b>0$ and $b\neq 1$, the \Emph{logarithmic function} of base $b$ is defined by
  \[f(x)=\log_b x\]
  with the ``domain of definition'' $x>0$.
\end{definition}
\Remark{If $z=\log_b A$, then $b^z=A$. If we combine them, we get
\[
  z=\log_b b^z\qquad\text{and}\qquad b^{\log_b A}=A.
\]
Students tend to understand this second equation as a complicated theorem, but it isn't. Rather, this is a \emph{definition} of $\log_b A$.
}
Now you need to recall various equations related to $\log_b A$.


\ornamentskip

\begin{enumerate}[resume]
  \itemA Calculate the following values (without using a calculator).
\begin{menumerate}{4}
  \item $\log_{10}100$
  \item $\log_{10}0.01$
  \item $\log_{10}0.001$
  \item $\log_{3}9$
  \item $\log_2 16$
  \item $\log_{3}9^9$
  \item $\log_{5}25^{-25}$
  \item $\log_5\sqrt{5}$
  \item $\log_{9}3$
  \item $\log_{0.5}0.5$
  \item $\log_{0.5}0.25$
  \item $\log_{1/3}3$
\end{menumerate}
\itemA Simplify the following expressions. Here, $a>0$, $a\neq 1$, and $x>0$.
\begin{menumerate}{3}
  \item $\dfrac{\log_{2}8}{\log_{2}4}$
  \item $\log_3{10}+\log_3{5}$
  \item $\log_3{10}-\log_3{5}$
  \item $\log_{10}\sqrt{5}+\log_{10}\sqrt2$
  \item $\log_2 x^{10}$
  \item $\log_2 \dfrac{1}{x^{10}}$
\end{menumerate}
\itemB Simplify the following expressions. Here, $a>0$, $a\neq 1$, and $b>0$, $b\neq 1$.
\begin{menumerate}{3}
  \item $2\log_{10}8 + \log_{10}0.25$
  \item $\dfrac{\log_{a}8}{\log_{a}4}$
  \item $\log_7 2^{20}+\log_7 3^{10}$
  \item $\dfrac{\log_{a}5}{\log_{b}5}$
  \item $(\log_a b)(\log_b a)^2$
  \item $\dfrac{\log_{a}5}{\log_{1/a}5}$
\end{menumerate}
\end{enumerate}

\newpage

\subsection{Toward the Derivative of Exponential Functions}
Now we try to calculate their derivatives, but we don't have do do anything special.
In principle, we can calculate any derivatives from the definition,
\begin{equation}
  \diff{}{x}f(x)=\lim_{h\to 0}\frac{f(x+h)-f(x)}{h}.\label{eq:diff}
\end{equation}
What you need to do is to apply the substitution $f(x)=2^x$.

\ornamentskip

\begin{enumerate}[resume]
 \itemB
 \begin{enumerate}
  \item Derive the following equations from Eq.~\eqref{eq:diff}.
\[
  \diff{}{x}2^x = 2^x\times\lim_{h\to 0}\frac{2^h-1}{h},\qquad
  \diff{}{x}3^x = 3^x\times\lim_{h\to 0}\frac{3^h-1}{h}.
\]
\item Using online resources, evaluate the above limits. The correct answer should be
\[
  \lim_{h\to 0}\frac{2^h-1}{h}\simeq0.7, \qquad \lim_{h\to 0}\frac{3^h-1}{h}\simeq1.1,
\]
which means
\[
  \diff{}{x}2^x \simeq 0.7\times2^x,\qquad
  \diff{}{x}3^x \simeq 1.1\times3^x.
\]
\item Let's define a function $\displaystyle\lambda(x)=\lim_{h\to 0}\frac{x^h-1}{h}$. Then, we can write down
\[
  \diff{}{x}2^x = \lambda(2)\times2^x,\qquad
  \diff{}{x}3^x = \lambda(3)\times3^x.
\]
Express $\displaystyle\diff{}{x}a^x$ for general $a>0$ with using $\lambda$.
\end{enumerate}
\itemC We have known $\lambda(2)\simeq 0.7$ and $\lambda(3)\simeq 1.1$ in the previous problem, so there should be a number $\xi$ that satisfies $\lambda(\xi)=1$.
Using online resources, evaluate $\lambda(2.6)$, $\lambda(2.7)$, $\lambda(2.73)$, etc., and try to find the number $\xi$.
\end{enumerate}

\newpage
\subsection{Napier's Constant and the Derivative of Exponential Functions}
\begin{definition}{Napier's Constant and Natural Logarithm}{e}
The number,
\begin{equation}
  \lim_{n\to\infty}\left(1+\frac{1}n\right)^n=\lim_{n\to\infty}\sum_{k=0}^n\frac{1}{k!}=2.7182818284590452\cdots\label{eq:napier}
\end{equation}
is called Napier's constant, and is denoted by $\ee$.

The logarithmic function with base $\ee$ is called the \Emph{natural logarithm}:
\[
  \ln x \coloneq \log_{\ee} x \quad(=\log_{2.71828\cdots}x).
\]
\end{definition}
This number satisfies $\lambda(\ee)=1$ in the previous page:
\[
  \lim_{h\to 0}\frac{\ee^h-1}{h}= \lim_{h\to 0}\frac{(2.718\cdots)^h-1}{h}=1.
\]
Namely, recalling the previous problems,
$ \diff{}{x}\ee^x=\lambda(\ee)\ee^x=\ee^x$, and now
we can prove this theorem:
\begin{theorem}{}{}
  \begin{equation}
      \diff{}x{\ee^x}=\ee^x,\qquad
      \diff{}x{a^x}=a^x\ln a\qquad(a>0).
  \end{equation}
\end{theorem}
Note that this is valid even for $a=1$. Can you explain why? \Hint{What is $\ln 1$?}

\ornamentskip

\begin{enumerate}[resume]
  \itemB Prove the following equations, where $a>0$.
  \begin{menumerate}{3}
    \item $a=\ee^{\ln a}$
    \item $a^x=\ee^{x\ln a}$
    \item $\displaystyle\diff{}{x}a^x=a^x\ln a$
  \end{menumerate}
  \itemS Calculate the following derivatives.
    \begin{menumerate}{4}
      \DIFF{3\ee^x}
      \DIFF{4^x}
      \DIFF{5\ee^{2x}}
      \DIFF{4^{x+2}}
      \DIFF{x\ee^x}
      \DIFF{x\ee^{-x}}
      \DIFF{x^2\ee^{-x^2}}
      \DIFF{\frac{\ee^{x}}{x^2+1}}
    \end{menumerate}
  \itemB Calculate the following derivatives.
    \begin{menumerate}{3}
      \DIFF{\sin(\ee^x)}
      \DIFF{\sqrt{\ee^x+1}}
      \DIFF{\frac{\ee^x+\ee^{-x}}{2}}
      \DIFF{\frac{\ee^x+\ee^{-x}}{\ee^x-\ee^{-x}}}
      \DIFF{\ee^{x^2-x}}
      \DIFF{2^{x^2-x}}
    \end{menumerate}
\end{enumerate}

\newpage
\subsection{Derivatives of Logarithmic Functions}
This is the last formula for your memory: derivatives of logarithmic functions.
\begin{theorem}{}{}
  \begin{equation}
      \diff{}x{\ln x}=\frac{1}{x},\qquad
      \diff{}x{\log_a x}=\frac{1}{x\ln a}\qquad(a>0,a\neq 1).\label{eq:log-deriv}
  \end{equation}
\end{theorem}
The proof is a bit tricky and postponed to the later pages. You first practice the calculation.

\ornamentskip

\begin{enumerate}[resume]
 \itemA Calculate the following expression. Use a calculator for (2), (3), and (4).
 \begin{menumerate}{2}
\item $\ln \ee$, $\ln \sqrt{\ee}$, $\ln \dfrac{1}{\ee}$.
\item $\ln 2$, $\ln 3$, $\ln 10$.
\item $\ln 1$, $\ln 1.001$, $\ln 1.000001$.
\item $\ln 0.1$, $\ln 0.001$, $\ln 0.000001$.
 \end{menumerate}
\itemB Rewrite the following expressions using natural logarithm ($\ln x$) but not having other logarithms (such as $\log_2 x$). Here, $a>0$ and $b>0$.
\begin{menumerate}{3}
  \item $\log_2 3$
  \item $\log_2a^b$
  \item $\log_7(3\ee^2/49)$
\end{menumerate}
  \itemS Calculate the following derivatives.
  \begin{menumerate}{4}
  \DIFF{2\ln x}
  \DIFF{\ln x^2}
  \DIFF{\ln 2x}
  \DIFF{\log_{10}x}
  \DIFF{\log_{10}99x}
  \DIFF{\ln(x^2 + x)}
  \DIFF{\ln(x^2 + 1)}
  \DIFF{\ln(x^2 - 1)}
  \DIFF{\ln \sin x}
  \DIFF{\ln \tan x}
  \DIFF{\ln \sqrt{x}}
  \DIFF{\ln(\ee^x + 1)}
  \DIFF{\log_{10}\ee^x}
  \DIFF{\ln10^x}
  \DIFF{\log_2\frac{x+1}{x-1}}
  \DIFF{\log_3\frac{x}{333}}
  \end{menumerate}
  \itemB Calculate the following derivatives, where $n$ is a positive integer.
  \begin{menumerate}{3}
  \DIFF{\ln(x^2 + \sin x)}
  \DIFF{\ln\sqrt{x^n + 1}}
  \DIFF{\ln x^{-n}\ee^x}
  \DIFF{\ln\frac{\tan x}{x}}
  \DIFF{\ln{{(x+1)}^n(x-1)}}
  \DIFF{\ln\sqrt{{(x+1)}^n(x-1)}}
\end{menumerate}
\end{enumerate}

\newpage

\subsection{Hyperbolic Functions}
You've learned the derivatives of exponential and logarithmic functions, which is almost everything that you need.
Here we introduce hyperbolic functions, but this is just for completeness and you only to know the names and basic properties.
\begin{definition}{Hyperbolic Functions}{}
  The hyperbolic functions are defined by
  \begin{align}
    \sinh(x) &\coloneq \frac{\ee^x-\ee^{-x}}{2},&
    \cosh(x) &\coloneq \frac{\ee^x+\ee^{-x}}{2},&
    \tanh(x) &\coloneq \frac{\sinh(x)}{\cosh(x)}.
  \end{align}
\end{definition}
These are similar to the trigonometric functions. For example,
\[ \sinh^2x-\cosh^2x=1 \qquad\text{(note the sign!)} \]

\ornamentskip

\begin{enumerate}[resume]
  \itemA Prove the following equations. Then, based on the equations, draw graphs of $\sinh x$, $\cosh x$, and $\tanh x$.
  \begin{menumerate}{2}
    \item $\sinh(-x)=-\sinh x$
    \item $\cosh(-x)=\cosh x$
    \item $\sinh^2x-\cosh^2x=1$
    \item $\cosh 0=1$ and $\sinh 0=0$
    \item $\displaystyle\lim_{x\to\infty}\cosh x=\lim_{x\to\infty}\sinh=\infty$
    \item $\displaystyle\lim_{x\to\infty}\tanh x=1$
    \DIFF{{\sinh x}=\diff{}{x}\frac{\ee^x-\ee^{-x}}{2}=\cosh x}
    \DIFF{{\cosh x}=\sinh x}
    \DIFF{{\tanh x}=\frac{1}{\cosh^2 x}}
  \end{menumerate}
  \itemC Calculate the following derivatives, where $n$ is a positive integer.
  \begin{menumerate}{3}
  \DIFF{\frac{1}{\sinh x}}
  \DIFF{\frac{\ee^x}{\sinh x}}
  \DIFF{\sinh^n x}
  \DIFF{\ln(\tanh x)}
  \DIFF{\sinh x^2}
  \DIFF{\sinh(\ln x)}
\end{menumerate}
\end{enumerate}

\newpage

\subsection{Basics of Differential Equations (1)}
The last part of this session, before the exercises, is a brief introduction to differential equations.
A simple example is
\[ f'(x)=4x. \]
Now you need to find $f(x)$ that satisfies this equation.
\Emph{One} solution is $f(x)=2x^2$. However, you will see that there are many more solutions in the following problems.

Another simple differential equation is
\[ f'(x) = f(x). \]
It is not easy to find the solution at first glance, but if you have done this Extra Session carefully, you will notice that $f(x)=\ee^x$ satisfies this equation. Indeed, $f(x)=\ee^x$ is \Emph{a} solution.

\ornamentskip

\begin{enumerate}[resume]
  \itemA Verify each of the following statement.
  \begin{enumerate}
    \item $f(x)=2x^2$ satisfies the differential equation $f'(x)=4x$.
    \item $f(x)=2x^2+5$ satisfies $f'(x)=4x$.
    \item $f(x)=2(x^2-1)$ satisfies $f'(x)=4x$.
    \item $f(x)=2x^2+x$ does not satisfy $f'(x)=4x$.
    \item For any number $C$, $f(x)=2x^2 + C$ satisfies $f'(x)=4x$.
    \item If $f(x)=ax^3+bx^2+cx+d$ satisfies $f'(x)=4x$, then $a=c=0$ and $b=2$, but $d$ can be any number.
  \medskip
    \item $f(x)=\ee^x$ satisfies the differential equation $f'(x)=f(x)$.
    \item $f(x)=-\ee^x$ and $f(x)=2\ee^x$ both satisfy $f'(x)=f(x)$.
    \item $f(x)=0$ satisfies $f'(x)=f(x)$.
    \item For any number $C$, $f(x)=C\ee^{x}$ satisfies $f'(x)=f(x)$.
    \item If $f(x)=ax^3+bx^2+cx+d$ satisfies $f'(x)=f(x)$, then $a=b=c=d=0$.
  \end{enumerate}
  \itemB Verify each of the following statement.
  \begin{enumerate}
    \item $f(x)=\sin x$ and $f(x)=-\sin x$ both satisfy $f''(x)=-f(x)$.
    \item $f(x)=\cos x$ and $f(x)=-\cos x$ also satisfy $f''(x)=-f(x)$.
    \item For any number $A$ and $B$, $f(x)=A\cos x+B\sin x$ satisfies $f''(x)=f(x)$.
  \end{enumerate}
  \itemC Verify each of the following statement.
  \begin{enumerate}
    \item $f(x)=\ee^x$, $f(x)=-\ee^x$, and $f(x)=\ee^{-x}$ all satisfy $f''(x)=f(x)$.
    \item For any number $A$ and $B$, $f(x)=A\ee^x+B\ee^{-x}$ satisfies $f''(x)=f(x)$.
    \item $f(x)=\sinh x$ and $f(x)=\cosh x$ also satisfy $f''(x)=f(x)$.
    \item We can express $\sinh x$ in the form of $A\ee^{x}+B\ee^{-x}$ if we choose $A$ and $B$. We can do the same for $\cosh x$.
  \end{enumerate}
\end{enumerate}

\newpage

\subsection{Basics of Differential Equations (2)}
If a differential equation only contains $f'(x)$ [in addition to $f(x)$ and $x$], it is called \Emph{first-order}.
Similarly, if it contains $f''(x)$ [in addition to $f'(x)$, $f(x)$, and $x$], it is called \Emph{second-order}.

At this level, it is sufficient for you to know the following facts.
\begin{theorem}{}{}
  Let $k$ be a real constant. Then,
\begin{itemize}
  \item the \Emph{general solution} of $\displaystyle f'(x) = kx$ is given by $f(x)=\dfrac{k}{2}x^2+C$,
  \item the {general solution} of $\displaystyle f'(x) = kf(x)$ is given by $f(x)=C\ee^{kx}$,
\end{itemize}
where $C$ is any number. (Notice that they are first-order differential equations.)
\end{theorem}
\begin{theorem}{}{}
  Let $k$ be a \Emph{positive} constant. Then,
\begin{itemize}
  \item the general solution of $\displaystyle f''(x) = +kf(x)$ is given by $f(x)=C_1\ee^{+\sqrt{k}x}+C_2\ee^{-\sqrt kx}$.
  \item the general solution of $\displaystyle f''(x) = -kf(x)$ is given by $f(x)=C_1\cos(\sqrt k x)+C_2\sin(\sqrt kx)$,
  \item the general solution of $\displaystyle f''(x) = k$ is given by $f(x)=\dfrac{k}{2}x^2+C_1x+C_2$,
\end{itemize}
where $C_1$ and $C_2$ are any numbers. (Notice that they are second-order differential equations.)
\end{theorem}
\emph{In most of simple cases}, the \Emph{general solution} of a first-order differential equation has one unknown constant $C$, and the general solution of second-order differential equation has two unknown constant $C_1$ and $C_2$.
(As you see in the following problems, they may be determined by other conditions.)


\ornamentskip

\begin{enumerate}[resume]
  \itemB Write down the general solution for the following differential equations.
  \begin{menumerate}{3}
    \item $f'(x)=x$
    \item $f''(x)=2$
    \item $f'(x)=-3x$
    \item $f''(x)=-1$
    \item $f'(x)=4f(x)$
    \item $f'(x)=-4f(x)$
    \item $f''(x)=4f(x)$
    \item $f''(x)=-4f(x)$
    \item $f''(x)=0$
  \end{menumerate}
  \itemB For each problem, find the function that satisfies all the given equations.
  \begin{menumerate}{2}
    \item $f'(x)=x$, $f(0)=0$.
    \item $f'(x)=-3x$, $f(0)=10$.
    \item $f'(x)=-4f(x)$, $f(0)=4$.
    \item $f'(x)=-4f(x)$, $f(0)=0$.
    \item $f''(x)=4$, $f(1)=2$, $f'(1)=3$.
    \item $f''(x)=4f(x)$, $f(0)=2$, $f'(0)=4$.
    \item $f''(x)=-4f(x)$, $f(0)=2$, $f'(0)=4$.
    \item $f''(x)=-9f(x)$, $f(0)=0$, $f'(0)=0$.
  \end{menumerate}
\end{enumerate}

\subsection{Workout 3: More Practice}
Now, again, it's time to do exercise! In fact, for physics, the above discussions are not so important. It's OK if you can calculate the following, and not-OK if you can't.

\begin{enumerate}[resume]
  \itemA \textsf{(basic to intermediate-level problems)}
\begin{menumerate}{4}
\DIFF{\sin 2x}
\DIFF{\ee^x}
\DIFF{\ln x}
\DIFF{\frac{1}{x}}
\DIFF{\frac{1}{\tan x}}
\DIFF{\cos^2 x}
\DIFF{\frac{1}{\sin^4 x}}
\DIFF{\sqrt{x^2+2}}
\DIFF{\sin(x + 1)}
\DIFF{\ln(x - 1)}
\DIFF{\ee^x \sin x}
\DIFF{\ee^x \ln x}
\DIFF{\frac{\sin x}{x}}
\DIFF{\frac{\sin(x^2)}{x}}
\DIFF{\frac{\ee^x}{x}}
\DIFF{\frac{\ln x}{x}}
\DIFF{\sin x \cos x}
\DIFF{\ee^{\sin x}}
\DIFF{\ee^{\ln x}}
\DIFF{3^{x^2+x}}
\end{menumerate}
\itemB \textsf{(intermediate to advanced-level problems)}
\begin{menumerate}{3}
  \DIFF{\frac{x^3 + 1}{\sqrt{x - 1}}}
  \DIFF{\frac{1}{\sqrt{\ln x}}}
\DIFF{\sin x^3\ln(x^3)}
\DIFF{(x^4+1)^{-3/5}}
\DIFF{x(x - 1)^{-3/4}}
\DIFF{x^{-2}e^{x^2} \sin x^2}
\DIFF{\ee^{\sqrt{x+1}}}
\DIFF{\ee^{x^2 \sinh x}}
\DIFF{\cosh(x^2 \ln x)}
\DIFF{\sin(x \sinh x)}
\DIFF{e^{x^3} \sinh x^3}
\DIFF{(5x^2 - 2x + 1)^{-3}}
\DIFF{\sin^2[(x^2+2x)^2]}
\DIFF{\tan \left(x+\sqrt{2x}\right)}
\DIFF{\frac{\sin(x^2 + 1)}{\sin(x - 1)}}

\DIFF{\frac{1}{\sqrt{x^2 + 4}}}
\DIFF{x^{-1/2}\cos x^2}
\DIFF{\frac{2\sin x}{(x - 1)^{3/4}}}

\DIFF{\frac {x^3\tan x}{\cos x}}
\DIFF{\frac {x\sin x}{\cos2x}}
\DIFF{\tan^2\left(x \sqrt{x}\right)}
\end{menumerate}
\end{enumerate}

\newpage

\subsection{Advanced Practice}

If you have not satisfied with the above problems, you can try the following ones.
First, you need to understand the following procedure.
\begin{itemize}
  \item Consider $\sin\sqrt{x^2+1}$ and we let $u=x^2+1$.
  \[
    \diff{}{x}\sqrt{x^2+1}=\diff{}{x}\sqrt u=\diff{\sqrt u}{u}\diff ux=\frac{1}{2\sqrt u}\cdot2x=\frac{x}{\sqrt{x^2+1}}.
  \]
  Therefore, with letting $v=\sqrt{x^2+1}$,
  \[
  \diff{}{x}\sin\sqrt{x^2+1}=\diff{\sin v}{x}=\diff{\sin v}{v}\diff vx=\cos v\cdot\frac{x}{\sqrt{x^2+1}}=\frac{x\cos\sqrt{x^2+1}}{\sqrt{x^2+1}}.
  \]
  \item You can calculate it at once, where we let $u(x)=x^2+1$ and $v=v(u)=\sqrt{u}$:
  \[
   \diff{}{x}\sin\sqrt{x^2+1}=\diff{\sin v}{v}\diff{v}{u}\diff{u}{x}=(\cos v)\frac{1}{2\sqrt{u}}(2x)=\frac{x\cos\sqrt{x^2+1}}{\sqrt{x^2+1}}.
   \]
\end{itemize}
Now you are ready to calculate any simple functions!

\ornamentskip

\begin{enumerate}[resume]
 \itemC \textsf{(The final problems: Vale Tudo)}
    \begin{menumerate}{3}
    \DIFF{\frac{\sinh x}{x^3 \ln x}}
    \DIFF{\left[\sin\left(x+\sqrt{x}\right)\right]^2}
    \DIFF{\frac{1}{\sqrt{\cos(2x^2 + 1)}}}
    \DIFF{\frac{3x + 1}{\sin(3x+1)}}
    \DIFF{\cos^2(x^2 - 1)}
    \DIFF{\left(x+ \tan x\right)^{4/3}}
    \DIFF{\left(\frac{\sin x}{3x^2 + 2x} \right)^4}
    \DIFF{\left(x - \sqrt{x^2 + 1} \right)^2}
    \DIFF{\tan\sqrt{x+x^{-1}}}
    \DIFF{\cos^5(x^2 + x)}
    \DIFF{x^2\sin x\tanh2x}
    \DIFF{\cos(\sin(\cos x))}
    \DIFF{\cos(\sinh(\cos x))}
    \DIFF{\frac{x^2+\tan x}{\sin(2x+1)}}
    \DIFF{\frac{x^2+\tan(\ln x)}{\sinh(2x^2+1)}}
   \end{menumerate}
 \end{enumerate}

\newpage

\subsection{Appendix: Derivative of Logarithmic Functions}
As an appendix for math-interested students, a path to the proof of the derivative of logarithmic functions is provided.

\Remark{In this Boot Camp, we simply accept the fact
\[ \ee\coloneq \lim_{n\to\infty}\left(1+\frac{1}n\right)^n
\qquad\text{satisfies}\qquad
\lim_{h\to 0}\frac{\ee^h-1}{h}=1
\]
without proof.
The proof is a bit involved and will be introduced in calculus lectures.
}

\ornamentskip
\begin{enumerate}[resume]
  \itemC Let's derive the derivative of logarithms,  Eq.~\eqref{eq:log-deriv}. Notice that $x>0$ throughout this problem.
  \begin{enumerate}
    \item Write down the definition of the derivative of $\ln x$.
    \item Replace $h$ in the equation by $h=x(\ee^z-1)$ and show
    $\displaystyle\diff{}{x}\ln x = \frac{1}{x}\lim_{z\to 0} \frac{z}{\ee^z-1}$.
    \item Show the two equations in Eq.~\eqref{eq:log-deriv}. In the derivation of the latter, you may need to use the formula $\log_a b = ({\log_c b})/({\log_c a})$, which is valid for $a>0$, $b>0$, $c>0$, $a\neq 1$, $c\neq 1$.
  \end{enumerate}
\end{enumerate}

\end{document}

