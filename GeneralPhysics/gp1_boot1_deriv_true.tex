%!TEX program=lualatex

\def\LectureName/{General Physics I}
%\def\LectureNumber/{}
%\def\LectureDate/{}
\PassOptionsToPackage{fleqn}{amsmath}
\PassOptionsToPackage{hyperfootnotes=false}{hyperref}
\documentclass[11pt,pdfa,lastpage]{MishoNote}
\title{General Physics I: Derivative Boot Camp (true form)}
\author{Sho Iwamoto}
\hypersetup{
  pdflang={en-US},
  pdfauthortitle={Assistant Professor, National Sun Yat-sen University},
  pdfsubject={Derivative Boot Camp as an introduction (and an initiation) to the university. A part of General Physics 1 lecture.},
  pdfcontactemail={iwamoto@g-mail.nsysu.edu.tw},
  pdfcontacturl={https://www2.nsysu.edu.tw/iwamoto/},
  pdfcaptionwriter={Sho Iwamoto},
  pdfcopyright={2024 Sho Iwamoto\textLF This document is licensed under the Creative Commons CC BY–NC 4.0 International Public License.},
  pdflicenseurl={https://creativecommons.org/licenses/by-nc/4.0/},
}

\usepackage{GP}
\usepackage{pgfornament}
\setlist{itemsep=4pt}
\def\thesection{A}

\newcommand\INT[2][\relax]{\item$\displaystyle\int#2\,\dd x$\ifx\relax#1\relax\relax\relax\else{\quad#1}\fi}
\newcommand\INP[1]{\INT{\left(#1\right)}}
\newcommand\INC[3]{\item$\displaystyle\int^{#2}_{#1}#3\,\dd x$}
\NewDocumentCommand\DIFF{O{x}om}{\item$\displaystyle\gdiff[\dd][#2]{}{#1}#3$}

\newcommand\hrefFN[2]{\href{#1}{#2}\footnote{\url{#1}}}
\newcommand\starskip{\bigskip\begin{center}\pgfornament[width=7cm]{88}\end{center}\medskip}

\let\origfootnote\footnote
\let\origfootnoterule\footnoterule

\begin{document}%
\title{Derivative Boot Camp (true form)}\begin{maketitle}
\let\footnote\origfootnote
\let\footnoterule\origfootnoterule

\subsection*{Preface}
Welcome to your first year of university!
As a university student in engineering, \Emph{you must be able to calculate derivatives of ``simple'' functions} such as
\[
\diff{}{x}\frac{x^2+\tan(\ln x)}{\sinh(2x^2+1)}.
\]
This Boot Camp is designed to help you prepare for your first year, which is unexpectedly tough for most of you!
Take your time, go through each problem carefully, and don't hesitate to ask for help!

\starskip

\noindent
Have you have completed the ``Derivative Boot Camp (Basic)''?
Congratulations! Yes, unfortunately, that was just the beginning of the Boot Camp; the problems you solved were \emph{the minimal problems}.

This \emph{true form} of the Boot Camp contains more problems for you. They
are classified into three categories:
$^{***}$ (basic), $^{**}$ (intermediate), and $^{*}$ (for motivated students).
You don't have to solve all of them, but try to solve as many as you can.
Good Luck!

\enlargethispage{-5em}

\makeatletter
\begin{tikzpicture}[remember picture,overlay]
  \begingroup
  \fontsize{9}{13}\selectfont
    \node[xshift=\@total@leftsep,yshift=25.5mm,anchor=south west,align=left,text width=\textwidth] at (current page.south west) {%
      \href{https://creativecommons.org/licenses/by-nc/4.0/}{\includegraphics[width=2.2cm]{../figs/by-nc.pdf}}\\[.4em]
      \noindent\textsf{\color{gray}%
      Visit \url{https://github.com/misho104/LecturePublic} for further information, updates, and to report issues.}\par
    };
    \node[xshift=-\@total@leftsep+26mm,yshift=32mm,align=left,text width=\textwidth,anchor=south east] at (current page.south east) {%
      \noindent\textsf{\color{gray}%
      This document is licensed under
      \href{https://creativecommons.org/licenses/by-nc/4.0/}{the Creative Commons CC--BY--NC 4.0 International Public License.}\\
      You may use this document only if you do in compliance with the license.}\par
    };
  \endgroup
\end{tikzpicture}
\makeatother

\end{maketitle}
\newpage
\subsection{The first step: High-school review}
\begin{enumerate}[start=14]
  \itemB Let $f(x)=(x+1)^4$. Calculate $f'(x)$, $f'(c)$, $f'(0)$ and $f'(1)$, but only with using high-school level formulae.
\end{enumerate}

\subsection{Trigonometric functions}

\begin{enumerate}[resume]
  \itemA Calculate the following values.
  \begin{menumerate}{4}
    \item $\displaystyle\sin\frac{7\pi}6$
    \item $\displaystyle\tan\frac{8\pi}3$
    \item $\displaystyle\cos\frac{-3\pi}2$
    \item $\displaystyle\sin\frac{-5\pi}6$
    \item $\displaystyle\cos\frac{-8\pi}3$
    \item $\cos(-4\pi)$
    \item $\displaystyle\cos^2\frac\pi6$
    \item $\displaystyle\sin^2\frac\pi4$
  \end{menumerate}
  \itemA Calculate the following values using calculators.
  \begin{menumerate}{4}
    \item $\sin1$
    \item $\cos\pi^2$
    \item $\sin0.0000123$
    \item $\tan0.0000777$
   \end{menumerate}
\end{enumerate}

\subsection{Get into the University}
Recall that derivatives are \Emph{defined by}
$\displaystyle\diff{}{x}f(x)=\lim_{h\to 0}\frac{f(x+h)-f(x)}{h}$. For example,
\begin{equation}
  \diff{}{x}x^2=\lim_{h\to 0}\frac{(x+h)^2-x^2}{h}=\lim_{h\to 0}\frac{2xh+h^2}{h}=\lim_{h\to 0}(2x+h)=2x.
  \label{eq:diffdefex}\tag{A.2}
\end{equation}
\medskip
\begin{enumerate}[resume]
  \itemC Calculate the following derivatives, not using formulae but \Emph{starting from the definition}. Namely, do the same thing as \eqref{eq:diffdefex} for each function.
  \medskip
  \begin{menumerate}{3}
    \DIFF{\frac1x}
    \DIFF{\frac1{x^2}}
    \DIFF{\sqrt{x}}
  \end{menumerate}
  For the last problem, the equation $\displaystyle\sqrt{a}-\sqrt{b}=\dfrac{a-b}{\sqrt{a}+\sqrt{b}}$ will be useful.

  \itemA We know that $f(x)$ satisfies $f''(x)=5$. Find $f(x)$. If you can, find more than one.
  \itemB We know that $f(x)$ satisfies $f'(x)=x$ and $f(1)=3$. Find $f(x)$. (Are there more than one?)
  \itemC We know that $f(x)$ satisfies $f''(x)=a$, $f'(0)=b$, and $f(x)=c$, where $a$, $b$, and $c$ are real constants. Find $f(x)$.
\end{enumerate}

\subsection{The formulae you need to memorize}

\begin{enumerate}[resume]
 \itemC Prove the following equations.
\begin{enumerate}
  \item $\displaystyle\diff{}{x}\sin x=\cos x$. Namely, do the same thing as \eqref{eq:diffdefex} for $f(x)=\sin x$. You may use
\begin{equation*}
  \lim_{x\to 0}\frac{\sin x}{x}=1\qquad\text{and}\qquad1+\cos\theta = \frac{1-\cos^2\theta}{1-\cos\theta} = \frac{\sin^2\theta}{1-\cos\theta}.
\end{equation*}
  \item Equation (A.5) from from Eqs.~(A.3) and (A.4).
  \item Equations (A.3) and (A.4). \Hint{This is not easy.}
\end{enumerate} 
\end{enumerate}


\subsection{Workout 1: Practice!}
\begin{enumerate}[resume]
\itemA Practice more. Here, $n$ is a positive integer.
\begin{menumerate}{3}
  \DIFF{\tan^2x}
  \DIFF{(x^2 + 3x)(4x^3 - 2x)}
  \DIFF{\frac{2x + 1}{x^4}}
  \DIFF{\frac{x+\sin x}{x+\cos x}}
  \DIFF{(\sin x + \cos x)^2}
  \DIFF{\frac{x+1}{x \cos x}}
  \DIFF{\frac{x^4 + x^2 + 1}{x^3 + x}}
  \DIFF{\frac{\cos x}{x^3}}
  \DIFF{x^{-n}\sin x}
\end{menumerate}
\itemB Use the formula (A.4) repeatedly to calculate the following derivatives.
\begin{menumerate}{2}
  \DIFF{\left(x\cdot\tan x \cdot\sin x\right)}
  \DIFF{(x^2 + 1)(x+1)(x+2)}
  \DIFF{(x^2+1)^3}
  \DIFF{x(\sin x + \cos x)^2}
  \DIFF{\frac{\sin^2 x}{\cos x}}
  \DIFF{(x^2+1)\sin^2x\tan x}
\end{menumerate}
\end{enumerate}
\subsection{One more formula}
(No extra problem for this part.)

\subsection{The last step: Composite functions}

\begin{enumerate}[resume]
  \itemB Practice with the following problems, which are a bit tough.
  \begin{menumerate}{3}
    \DIFF{\cos(\sin x)}
    \DIFF{\sin(3x^2 + 2x)}
    \DIFF{\sqrt{5x^2 + 7x}}
    \DIFF{\frac1{\sqrt{5x^2 + 7x}}}
    \DIFF{\sin\sqrt{x}}
    \DIFF{(3x^2+ 1)^{5/2}}
    \DIFF{\tan(6x^2 - 5x)}
    \DIFF{\tan x^3}
    \DIFF{\tan(\tan x)}
  \end{menumerate}

\end{enumerate}

\subsection{Workout 2: Practice, Practice, Practice!}
\begin{enumerate}[resume]
\itemA \textsf{(intermediate-level problems)}
\begin{menumerate}{3}
  \DIFF{(5x^2 - 2x + 1)^3}
  \DIFF{\sqrt{x^4+1}}
  \DIFF{\sin(x^2+2x+2)}
  \DIFF{\tan\sqrt{2x}}
  \DIFF{\frac{(x + 1)^2}{(x - 1)^2}}

  \DIFF{\left(\frac{1}{x} + x\right)^2}
  \DIFF{x^{-1/2}\cos x}
  \DIFF{x^2\tan2x}
  \DIFF{(x - 1)^{-1/4}}
  \DIFF{\frac{2}{(x - 1)^{1/4}}}

  \DIFF{\frac {x^3}{\cos x}}
  \DIFF{\frac {x}{\cos2x}}
  \DIFF{\frac {x^2}{\cos x^2}}
  \DIFF{\cos(x^2+1)^2}
  \DIFF{\tan^2(x^2+1)}
\end{menumerate}
\newpage
\itemB \textsf{(a bit tough problems)}
\begin{menumerate}{3}
  \DIFF{(5x^2 - 2x + 1)^{-3}}
  \DIFF{(x^4+1)^{-3/5}}
  \DIFF{\sin^2[(x^2+2x)^2]}
  \DIFF{\tan \left(x+\sqrt{2x}\right)}
  \DIFF{\frac{\sin(x^2 + 1)}{\sin(x - 1)}}

  \DIFF{\frac{1}{\sqrt{x^2 + 4}}}
  \DIFF{x^{-1/2}\cos x^2}
  \DIFF{x^2\sin x\tan2x}
  \DIFF{x(x - 1)^{-3/4}}
  \DIFF{\frac{2\sin x}{(x - 1)^{3/4}}}

  \DIFF{\frac {x^3\tan x}{\cos x}}
  \DIFF{\frac {x\sin x}{\cos2x}}
  \DIFF{\frac {x^2\sin^2x}{\cos x^2}}
  \DIFF{\frac{x^3 + 1}{\sqrt{x - 1}}}
  \DIFF{\tan^2\left(x \sqrt{x}\right)}
\end{menumerate}
\end{enumerate}

\vfill
\subsection*{Afterwords}

Have you finished all the problems? Great job! You're now 100\% ready for your university learning!

If you are going to take Sho's lecture, \Emph{you can email your answers to Sho}.
Sho will look at it and give you feedback. This will \emph{not} be included in the grade evaluation, but Sho will acknowledge see your hard work and you might get some recognition for your effort.


\vfill

\noindent
\emph{P.S.~This is the end of this Boot Camp, but just the beginning of your learning. I mean, \textbf{I am ready to provide motivated students with more advanced tasks.} Send me an email if you are interested.}

\end{document}
