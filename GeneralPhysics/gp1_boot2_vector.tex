%!TEX program=lualatex
%!TEX options=-synctex=1 -interaction=nonstopmode -halt-on-error "%DOC%.tex"
\def\LectureName/{General Physics}
%\def\LectureNumber/{}
%\def\LectureDate/{}
\PassOptionsToPackage{fleqn}{amsmath}
\PassOptionsToPackage{hyperfootnotes=false}{hyperref}
\documentclass[11pt,pdfa,lastpage]{MishoNote}
\title{\LectureName/: Vector Boot Camp}
\author{Sho Iwamoto}
\hypersetup{
  pdflang={en-US},
  pdfauthortitle={Assistant Professor, National Sun Yat-sen University},
  pdfsubject={Vector Boot Camp as a preparation to \LectureName/ lecture.},
  pdfcontactemail={iwamoto@g-mail.nsysu.edu.tw},
  pdfcontacturl={https://www2.nsysu.edu.tw/iwamoto/},
  pdfcaptionwriter={Sho Iwamoto},
  pdfcopyright={2024–2025 Sho Iwamoto\textLF This document is licensed under the Creative Commons CC BY–NC 4.0 International Public License.},
  pdflicenseurl={https://creativecommons.org/licenses/by-nc/4.0/},
}

\usepackage{GP}
\setlist{itemsep=4pt}
\tikzset{>={Stealth[scale=1]}}
\usetikzlibrary{fadings}

\def\thesection{B}

\let\origfootnote\footnote
\let\origfootnoterule\footnoterule

\begin{document}%
\title{Vector Boot Camp}
\begin{maketitle}
\let\footnote\origfootnote
\let\footnoterule\origfootnoterule

\subsection*{Preface}
Vectors are likely a familiar concept to you from high school mathematics.
However, vectors are far more important than you might realize.
Vectors are among the most fundamental mathematical tools, essential not only in science but also in modern technology and society.

A vector can take on many forms: in high school math, it's often seen as two or three numbers; in university physics, it's represented as an arrow; and in university mathematics, it's an element of a vector space.
What you have learned is just the beginning, and you will need to develop a deeper and more versatile understanding of vectors.

This Boot Camp is designed to bridge the gap between high school math and university physics.
When you pass this program, you will be well-prepared to tackle challenges and opportunities of advanced physics studies.
Good luck!

\subsection*{Remarks}
This Boot Camp contains all the topics that you will need in General Physics 1 and 2 lectures.
Most of the topics will be discussed during Sho's lecture, so you do not have to do all by yourself. Still, it is beneficial if you can learn these topics by yourselves, so that you can focus more on physics during the lectures.

\medskip

Sho never provides you with solutions\addnote{Think why. Sho thinks that giving you solutions would violate Sho's integrity as a scientist.}. \Emph{You students} need to make the solution. To this end,
\begin{miniitemize}
  \item Share your answers to other colleagues, using LINE or \hrefFN{https://docs.google.com/}{Google Docs}. Compare your answers with theirs.
  \item Ask questions to colleagues, to the TA, or to Sho. You can utilize Sho's \hrefFN{https://www2.nsysu.edu.tw/iwamoto/}{office hours}.
\end{miniitemize}

\OutputNote

\enlargethispage{-5em}

\makeatletter
\begin{tikzpicture}[remember picture,overlay]
  \begingroup
  \fontsize{9}{13}\selectfont
    \node[xshift=\@total@leftsep,yshift=25.5mm,anchor=south west,align=left,text width=\textwidth] at (current page.south west) {%
      \href{https://creativecommons.org/licenses/by-nc/4.0/}{\includegraphics[width=2.2cm]{../figs/by-nc.pdf}}\\[.4em]
      \noindent\textsf{\color{gray}%
      Visit \url{https://github.com/misho104/LecturePublic} for further information, updates, and to report issues.}\par
    };
    \node[xshift=-\@total@leftsep+26mm,yshift=32mm,align=left,text width=\textwidth,anchor=south east] at (current page.south east) {%
      \noindent\textsf{\color{gray}%
      This document is licensed under
      \href{https://creativecommons.org/licenses/by-nc/4.0/}{the Creative Commons CC--BY--NC 4.0 International Public License.}\\
      You may use this document only if you do in compliance with the license.}\par
    };
  \endgroup
\end{tikzpicture}
\makeatother

\end{maketitle}
\newpage


\newsavebox{\VectorSetA}
 \begin{lrbox}{\VectorSetA}
 \begin{tikzpicture}[scale=0.74]
  \draw[help lines,thin,dashed] (-3.2,-1.2) grid ++(18.6,6.6);
  %\draw[->]                     (-3.2,  0 ) --   ++(18.6, 0 ) node[below left,xshift=1.3mm]{$x$};
  %\draw[->]                     (  0 ,-1.2) --   ++( 0  ,6.6) node[below left,yshift=1.3mm]{$y$};
  %\coordinate (O) at (0,0) node [below left,xshift=.5mm,yshift=.5mm] at (O) {O}; %\fill (O) circle (4pt);
  \draw[->, very thick] (1,4)--node [above]{$\vc a$} ++(5,0);
  \draw[->, very thick] (-2,4)--node [left]{$\vc b$} ++(0,-2);
  \draw[->, very thick] (1, 0)--node [left]{$\vc c$} ++(0,3);
  \draw[->, very thick] (2,-1)--node [above left,yshift=-3mm]{$\vc p$} ++(3,3);
  \draw[->, very thick] (6,2)--node [above right]{$\vc q$} ++(2,-2);
  \draw[->, very thick] (9,5)--node [above left]{$\vc r$} ++(-2,-2);
  \draw[->, very thick] (8,-1)--node [above left,xshift=2mm]{$\vc A$} ++(3,4);
  \draw[->, very thick] (13,5)--node [above]{$\vc B$} ++(-1,-1);
  \draw[->,very thick] (13,1)--node [below]{$\vvex$} ++(1,0);
  \draw[->,very thick] (13,1)--node [left] {$\vvey$} ++(0,1);
  %\fill (O) circle (4pt) (A) circle (4pt) (B) circle (4pt) (C) circle (4pt) (D) circle (4pt);
  \end{tikzpicture}
\end{lrbox}
\newsavebox{\VectorSetB}
 \begin{lrbox}{\VectorSetB}
 \begin{tikzpicture}[scale=0.74]
  %\draw (0,0) rectangle ++ (7,4);
  \coordinate (A) at (1,3) node [above,xshift=-2mm] at (A) {A};
  \coordinate (B) at (5,5) node [right,yshift=2mm] at (B) {B};
  \coordinate (C) at (6,0) node [above,xshift=2mm] at (C) {C};
  \fill (A) circle (4pt) (B) circle (4pt) (C) circle (4pt);
  \end{tikzpicture}
\end{lrbox}
\newsavebox{\VectorSetC}
\begin{lrbox}{\VectorSetC}
  \begin{tikzpicture}[scale=0.74]
   \draw[help lines,thin,dashed] (-0.2,-0.2) grid ++(7.6,6.6);
   \draw[->]                     (-0.2,  0 ) --   ++(7.6, 0 ) node[below left,xshift=1.3mm]{$x$};
   \draw[->]                     (  0 ,-0.2) --   ++( 0  ,6.6) node[below left,yshift=1.3mm]{$y$};
   \coordinate (O) at (0,0) node [below left,xshift=.5mm,yshift=.5mm] at (O) {O}; \fill (O) circle (4pt);
  \coordinate (A) at (1,3) node [above,xshift=-2mm] at (A) {A};
  \coordinate (B) at (5,5) node [right,yshift=2mm] at (B) {B};
  \coordinate (C) at (6,0) node [above,xshift=2mm] at (C) {C};
  \fill (A) circle (4pt) (B) circle (4pt) (C) circle (4pt);
  \end{tikzpicture}
\end{lrbox}
\subsection{Definition of Vectors}
You might already know vectors are $\vc a=(1,0,0)$ or $\vc A=(4,4)$, but let's forget everything for a moment (until Section~\ref{sec:component}).
We \Emph{define} vectors in an easier way.

\begin{definition}{Vectors (for physics)}{vec}
  A vector is a physical quantity that has both \Emph*{magnitude} and \Emph*{direction}.
\end{definition}
\begin{definition}{How to write vectors}{notation}
\begin{miniitemize}
\item We describe vectors by a symbol with arrows. For example, $\vc v$, $\vc V$, $\vc x$, and $\vc p\w i$ are vectors.%
\addnote{%
    Professionals use boldface, e.g., $\mathbf{v}$, $\mathbf{V}$, $\mathbf x$, $\mathbf{p}\w i$ but most textbooks for beginners use the arrow style.}
\item We describe their magnitude by, e.g., $|\vc v|$, $|\vc V|$, $|\vc x|$, and $|\vc p\w i|$.
  \item In figures, vectors can be drawn by arrows. The arrow's direction should match the vector's direction. The arrow's length should be proportional to the vector's magnitude.
\end{miniitemize}
\end{definition}
\OutputNote

Two symbols $a$ and $b$ usually describe different numbers. Similarly, $v$ and $V$ are different symbols, even though look similar. So, $\vc v$ and $\vc V$ are different vectors; they are totally unrelated. Since $\vc a$ and $\vc A$, or $\vc x$ and $\vc X$, are totally unrelated, different objects, we need to write them clearly so the difference is obvious and we can distinguish them easily.




Furthermore, \Emph{$v$ and $\vc v$ are totally unrelated, different objects}.
While $\vc v$ is a vector and has a direction, $v$ is a number without direction.
Just as $a$ and $B$ are completely different symbols, $\vc v$ and $v$ are also treated as entirely different symbols, in principle.

However, most physicists are lazy. Consider a vector $\vc v$. To describe its magnitude, we should write $|\vc v|$ (and mathematicians do so) but Sho and other lazy physicists tend to write $v$. In other words, in physics lectures, \emph{sometimes}, $E$ means $|\vc E|$ (the magnitude of a vector $\vc E$).

Further bad news is that we sometimes write an equation like
\[\vc x=\pmat{x\\y\\z}.\]
Here, $\vc x$ is a vector and $x$, $y$, and $z$ are its components.
It means that $\vc x$ and $x$ are totally unrelated objects and, of course, $|\vc x|$ is not equal to $x$. Rather, you might have already learned $|\vc x|=\sqrt{x^2+y^2+z^2}\neq x$.
Sho tries to avoid this kind of bad notations, but you will see them occasionally and you need to be prepared.

\Remark{The following symbols are similar and confusing but we often see all of them:
  \begin{multicols}{4}\begin{itemize}
   \item $a$ / $\alpha$
   \item $b$ / $\beta$
   \item $c$ / $C$
   \item $e$ / $E$ / $\epsilon$
   \item $g$ / $q$ / $9$
   \item $i$ / $l$ / $1$ / $I$
   \item $k$ / $K$ / $\kappa$
   \item $m$ / $\mu$
   \item $p$ / $\rho$
   \item $s$ / $S$
   \item $t$ / $\tau$ / $T$
   \item $u$ / $v$ / $\nu$
   \item $w$ / $\omega$ / $W$
   \item $x$ / $X$ / $\chi$
  \end{itemize}\end{multicols}
}

\begin{definition}{Vector quantity and Scalar quantity}{vecsca}
  \begin{miniitemize}
    \item Physical quantities with direction are called \Emph*{vector quantity}.\addnote{\Hint[]{a very technical note} If you notice a circular argument in this and the previous definitions and become unhappy, you may define a vector as ``a mathematical object that is used to describe a physical quantity with magnitude and direction,'' but this statement will raise a question ``what are mathematical objects?'', which is beyond this course.}
    \begin{miniitemize} \item They can be described by vectors.\end{miniitemize}
    \item Physical quantities without direction are called \Emph*{scalar quantity}.
    \begin{miniitemize} \item They can be described by numbers.\end{miniitemize}
    \item Consider a vector quantity $\vc A$. Its magnitude $|\vc A|$ is a scalar quantity.
  \end{miniitemize}
  \end{definition}
Let us discuss a few examples.
\Emph*{Mass} $m$ is obviously a scalar quantity. \Emph*{Temperature} $T$ can be positive or negative but is a scalar quantity, because it has no direction. \Emph*{Velocity} $\vc v$ is a vector quantity, and its magnitude $|\vc v|$ is a scalar quantity. This quantity has a special name ``\Emph*{speed}''.
Furthermore, \Emph*{area} and \Emph*{volume} are scalar quantities, while \Emph*{acceleration} $\vc a$ is a vector quantity.

\begin{quizzes}
  \Quiz[S] Choose vector quantities. Choose scalar quantities.
  \begin{menumerate}[labelsep=0em,labelwidth=0em,itemindent=1em,label={},leftmargin=0.3em]{4}
    \item air pressure
    \item wind speed
    \item wind velocity
    \item electric charge
    \item position
    \item distance
    \item time difference
    \item time
    \item duration
    \item resistance
    \item conductor
    \item conductance
    \item conductivity
    \item $\vc x$
    \item $x$
    \item $|\vc x|$
    \item $v+|\vc x|$
    \item magnitude of $\vc v$
    \item direction of $\vc v$
    \item Sho
    \item Sho's height
    \item Sho's weight
    \item Sho's mass
  \end{menumerate}
  \Quiz[S]
   \QSub{1} Assume $\vc a$ is a vector quantity. What does $|\vc a|$ mean?
   \\[.5em]\TAB
   \QSub{2} Assume $v$ is a scalar quantity. What does $|v|$ mean?\\[.5em]
    \TAB\TAB\Hint{They are different! The answer can be found in these two pages.}
\end{quizzes}


\OutputNote

\subsection{How to describe directions}
When you describe a vector, you need to specify its magnitude and direction. The magnitude is easy (we can use a number), but the direction is not, because now you need to describe it \Emph{in English}!

\def\UDL#1{$\langle$\textsf{#1}$\rangle$}

If the space is one-dimensional, we can describe the direction by $\pm$. For example, if we only consider vertical direction (up-down), we may define \UDL{up} as $+$ and \UDL{down} as $-$, and then the number $+3$ means \UDL{up, 3} and $-5$ means \UDL{down, 5}.
Or, instead, you may define \UDL{down} as the positive direction and \UDL{up} as negative, in which case $+3$ means \UDL{down, 3} and $-5$ means \UDL{up, 5}. Obviously, if you use the sign ($\pm$) to describe the direction, \Emph{YOU must specify which direction is positive}.

\Remark{
  Using natural English is nice, but \Emph{clarity} is much more important in science. We should pay greater attention to ensure that our expressions are clear and does not cause any misunderstanding.
}

\newpage

\subsubsection{Directions in two-dimensional space}
Now, how can we describe the direction in two-dimensional space?

\paragraph{Basic}
The easiest method is to use the following expressions:
\begin{tabbing}
 \fakebullet \= downward   \== toward the bottom      \=\kill
 \fakebullet \> leftward   \>= toward the left        \>= to the left\\[\itemsep]
 \fakebullet \> rightward  \>= toward the right       \>= to the right\\[\itemsep]
 \fakebullet \> upward     \>= toward the top         \>= to the top\\[\itemsep]
 \fakebullet \> downward   \>= toward the bottom      \>= to the bottom
% \item forward / backward (if the front and back are obvious, e.g., a car or bicycle)
\end{tabbing}
If you need to describe the third dimension which is perpendicular to the textbook's page or the sheet (or the blackboard), you can use
\begin{tabbing}
  \fakebullet \= out of the \= (sheet | page | blackboard) \= = away from the \= (sheet | page | blackboard)\kill
  \fakebullet \> into   the \> (sheet | page | blackboard) \> = toward    the \> (sheet | page | blackboard)\\[\itemsep]
  \fakebullet \> out of the \> (sheet | page | blackboard) \> = away from the \> (sheet | page | blackboard)
\end{tabbing}
Sometimes the word ``the horizontal'' is useful:
\begin{miniitemize}
  \item $10^\circ$ above the horizontal / $30^\circ$ below the horizontal
\end{miniitemize}

\paragraph{N-E-W-S}
For 2d case, \Emph{if you specify the ``north'' direction,} you can use the following expressions:
\begin{miniitemize}
 \item northward, southward, eastward, westward (= (to | toward) the north, etc.)
 \item northwestward, southwestward, etc. (= (to | toward) the northwest, etc.)
\end{miniitemize}
The following expressions are equivalent:
\begin{multicols}{2}\begin{miniitemize}
 \item northwestward
 \item $45$ degrees (= $45^\circ$) west of north
 \item $45^\circ$ counterclockwise from due north
 \item $-45$ degrees clockwise from due north
\end{miniitemize}\end{multicols}
\noindent
Try to find your favorite expression and use it, but don't forget to define the ``north'' direction!
\Remark{$45$ and $45^\circ$ are completely different: $45^\circ$ means 45 degrees, while 45 is \emph{always} understood as 45 radians ($= 45\times180/\pi\unit{degrees}\simeq2578^\circ$). In other words, $\cos60^\circ=1/2$ but $\cos60\simeq-0.95$.
}

\paragraph{With axes}
Usually, we define $x$-axis and $y$-axis (and $z$-axis, if 3d). \Emph{If such axes are defined} (or once you have defined them), we can use
\begin{tabbing}
  \fakebullet \= in the negative $x$-direction \= = in the $-x$ direction\kill
  \fakebullet \> in the positive $x$-direction \> = in the $+x$ direction\\[\itemsep]
  \fakebullet \> in the positive $y$-direction \> = in the $+y$ direction\\[\itemsep]
  \fakebullet \> in the negative $x$-direction \> = in the $-x$ direction\\[\itemsep]
  \fakebullet \> in the negative $y$-direction \> = in the $-y$ direction
 \end{tabbing}
In particular, for 2d cases, we can use the angle from the positive $x$-axis (counterclockwise):
 \begin{miniitemize}
  \item $30^\circ$ from the $+x$ axis (= $30^\circ$ counterclockwise from the positive $x$-axis)
  \item $90^\circ$ from the $+x$ axis (= in the $+y$ direction)
  \item $170^\circ$ from the $+x$ axis (= $10^\circ$ clockwise from the negative $x$-axis)
  \item $-15^\circ$ from the $+x$ axis (= $15^\circ$ clockwise from the positive $x$-axis)
  \item angle $\theta$ from the $+x$ axis, where $\tan\theta=0.1$ and $0<\theta<\pi/2$.
 \end{miniitemize}
\Remark{It is difficult to describe general 3d directions by words. For this purpose, we usually use mathematical expressions (i.e., vectors).}


\subsubsection{Relationship of two vectors}
When you need to describe the angle between two vectors, you can use the following expressions:
If the angle between $\vc a$ and $\vc b$ is $90^\circ$, we say
\begin{tabbing}
  \fakebullet \= \kern18em \=\kill
\fakebullet \> $\vc a$ is \Emph{perpendicular to} $\vc b$. {\footnotesize (most common)} \> \fakebullet $\vc a$ and $\vc b$ are perpendicular to each other.\\[\itemsep]
  \fakebullet \> $\vc a$ is orthogonal to $\vc b$.   \>    \fakebullet $\vc a$ and $\vc b$ are orthogonal  to each other.\\[\itemsep]
  \fakebullet \> $\vc a$ is normal to $\vc b$.       \>    \fakebullet $\vc a$ and $\vc b$ are normal to each other.
 \end{tabbing}
If the angle is $180^\circ$, we say
\begin{tabbing}
  \fakebullet \= \kern18em \=\kill
\fakebullet \> $\vc a$ is \Emph{anti-parallel to} $\vc b$. {\footnotesize (most common)} \> \fakebullet $\vc a$ and $\vc b$ are anti-parallel.\\[\itemsep]
  \fakebullet \> $\vc a$ is opposite to $\vc b$.      \> \fakebullet $\vc a$ and $\vc b$ are in opposite directions.
\end{tabbing}
Finally, if the angle is $0^\circ$,
\begin{tabbing}
  \fakebullet \= \kern16em \=\kill
  \fakebullet \> $\vc a$ is \Emph{in the same direction as} $\vc b$. \> \fakebullet $\vc a$ and $\vc b$ are in the same direction.
\end{tabbing}
For $0^\circ$, we should avoid the word ``parallel'' because it is ambiguous; some of your readers may think it includes both $0^\circ$ and $180^\circ$ cases. Clarity is important in science.

\OutputNote



\begin{problems}
 \Problem[S] {}\label{q:vecbasic} Vectors are drawn on the grid. Answer the following questions.
 \par\smallskip\par\usebox{\VectorSetA}

\begin{enumerate}
  \item Assume the grid has a spacing of 1. Describe the direction (in English words) and magnitude of each vector. Try to use multiple expressions.
  \item Describe relationship between the directions of the following vector pairs:\\
   ($\vc a$ and $\vc b$), ($\vc b$ and $\vc c$), ($\vc p$ and $\vc q$), ($\vc p$ and $\vc r$), ($\vvex$ and $\vvey$), and ($\vc a$ and $\vvex$).\\
   For example, ``$\vc a$ is perpendicular to $\vc b$''.
   \item Draw a vector that is normal to $\vc p$ and has a length $\sqrt 8$.
   \item Can you draw another vector that is normal to $\vc p$ and has a length $\sqrt 8$, but different from the one you drew in the previous question?
\end{enumerate}

\end{problems}

\subsubsection{Extra note: How to describe the direction of rotations}
\begin{minipage}[b]{0.66\textwidth}
To express the direction of rotations or revolution, the words \Emph*{clockwise} and \Emph*{counterclockwise} are the best useful.
For example, the disk in the right figure is rotating
\begin{miniitemize}
  \item counterclockwise if viewed from the positive $x$-direction
  \item clockwise if viewed from the negative $x$-direction
\end{miniitemize}

If the rotation is about an arrow (or an axis), you can use the following expressions:
\end{minipage}\hfill
\begin{minipage}[b]{0.32\textwidth}
\begin{tikzpicture}[rotate=-90,scale=0.9]
\tikzstyle{disk}=[line width=0.5,orange!30!black,fill=orange!40!black!10, top color=orange!40!black!20,bottom color=orange!40!black!10,shading angle=20]
    \draw[disk] (-2,0) --++ (0,-0.2) arc(180:360:{2} and {0.4*2}) --++ (0,0.2);
    \draw[disk] (0,0) ellipse({2} and {0.4*2});
    \draw[] (20:0.6*2) arc(20:85:{0.6*2} and {0.18*2});
    \draw[<-] (160:0.6*2) arc(160:95:{0.6*2} and {0.18*2});
    \draw[->] (200:0.6*2) arc(200:340:{0.6*2} and {0.18*2});
%    \node at (-1.3,0.3) { $\omega$ };
    \draw[very thick,->] (0,0) --++ (0,3)  node [right]{$x$};
    \draw[very thick] (0,-2)--(0,-1);
\end{tikzpicture}
\end{minipage}
\begin{miniitemize}
  \item following the right-hand rule
  \item following left-hand rule
\end{miniitemize}
Here, you use your right hand (or left hand): align the thumb to the arrow (or the axis) and curl your fingers. The direction of the curling fingers is the direction of the rotation. For example,

\begin{miniitemize}
  \item The above disk is rotating about the $x$-axis following the \emph{right-hand} rule.
\end{miniitemize}

\Remark{Later in General Physics lecture, you will learn that any 3d rotations can be described by vectors (called angular speed vectors). For a rotation with the anguar speed vector $\vc \omega$, the direction of $\vc \omega$ describes the axis of the rotation (under the right-hand rule) and $|\vc \omega|$ is the angular speed $\dd\theta/\dd t$.}

\newpage

\subsection{Vector Basics}
These are a few basic things that you have learned in highschool, such as
\par\smallskip
\tikzfading[name=fade left, left color=transparent!100, right color=transparent!0]
\begin{tikzpicture}
  \node[inner sep=0pt,inner ysep=2pt,outer sep=0pt,clip] (A) {\makebox[0.95\linewidth][l]{$
  \displaystyle\vc a=\pmat{1\\2},\qquad\vc b=\pmat{4\\4}\quad\then\quad \vc a+\vc b=\pmat{1+4\\2+4}=\pmat{5\\6}.$}};
   \path (A.north) -- (A.north east) coordinate[midway] (M);
  \fill[white,path fading=fade left] (A.south west) rectangle (M);
\end{tikzpicture}
\par\smallskip\noindent
However, we have forgotten that vectors are described by numbers!
We need to reinvent everything by thinking \emph{vectors as arrows}. For example, there is a special arrow with length zero:


\begin{definition}{Zero vector}{zero}
There is a special vector $\vc 0$, called \Emph*{zero vector}.
Its magnitude is 0 (zero) and it has no direction.
\end{definition}
Notice that $\vc 0$ is a vector quantity and $|\vc 0|=0$ is a scalar quantity.
\begin{quizzes}
  \Quiz[S] Explain the difference between $\vc 0$ and 0. Explain why $|\vc 0|=0$.
\end{quizzes}

Arrows can be extended or added, which are \emph{our definition} of scalar multiplication and addition.
%
\begin{definition}{Scalar multiplication}{scamul}
If $\vc v$ is a vector and $k$ is a number, $k\vc v$ is a vector specified by the following rule:
\begin{miniitemize}
  \item If $k>0$, the direction of $k\vc v$ is the same as $\vc v$ and the magnitude is $k$-times larger than $\vc v$.
  \item If $k<0$, $k\vc v$ is anti-parallel to $\vc v$ and the magnitude is $|k|$-times larger than $\vc v$.
  \item If $k=0$, then $k\vc v$ is the zero vector.
\end{miniitemize}
In other words,
\begin{miniitemize}
  \item The magnitude of $k \vc v$ is equal to $|k||\vc v|$.
  \item The direction of $k \vc v$ is the same as (anti-parallel to) $\vc v$ if $k$ is positive (negative).
\end{miniitemize}
\end{definition}

\begin{quizzes}
  \Quiz[S] Most of first-year students misunderstand the formula
\begin{equation}
  |k\vc v|=|k|\,|\vc v|.
\end{equation}
  Explain the meaning of $|k|$, $|\vc v|$, and $|k\vc v|$. (They are not the same!) Then, explain the meaning of $|k|\,|\vc v|$. Finally, explain why $|k\vc v|$ is equal to $|k|\,|\vc v|$.

  \Hint[Remark:~]{Here you \Emph{cannot} use $\vc v=\spmat{a\\b\\c}$ etc.\ since we have forgotten this expression.}
\end{quizzes}

\newpage

\begin{definition}{Addition}{addition}
\begin{minipage}[b]{0.5\textwidth}
If $\vc a$ and $\vc b$ are vectors, $\vc a+\vc b$ is a vector that is obtained by, as in the right figure, placing the tail of $\vc b$ at the head of $\vc a$.
\vspace{3em}
\end{minipage}\hfill
\begin{minipage}[b]{0.4\textwidth}
\begin{tikzpicture}
    \coordinate (A) at (-2, 1.3);
    \coordinate (B) at (3, 0.7);
    \coordinate (C) at (1, 2);
    \draw[very thick,->] (O) -- (A) node[pos=0.7, below left] {$\vec{a}$};
    \draw[very thick,->] (O) -- ++(B) node[pos=0.5,above] {$\vec{b}$};
    \draw[very thick,->] (A) -- ++(B) node[pos=0.5,above] {$\vec{b}$};
    \draw[->,very thick,pBlue] (O) -- (C) node[pos=0.6,left] {$\vec{a} + \vec{b}$};
    \draw[dashed] (A) -- (C);
    \draw[dashed] (O) -- (A);
\end{tikzpicture}
\end{minipage}
\end{definition}
There are two $\vc b$s in the figure but it is not a mistake. The two arrows have the same direction and the same magnitude, so they are the same vector. Both of the two arrows are $\vc b$.

\pagebreak[3]

\begin{quizzes}
  \Quiz[A] While $\vc a+\vc a$ and $2\vc a$ are the same vectors, their meaning is slightly different. Explain the difference \Emph{based on the above definitions}.
\end{quizzes}

It is not difficult to prove the following formulae from the above definitions:
\begin{align}
  &\vc a+\vc b=\vc b+\vc a,&
  &p\vc a+q\vc a=(p+q)\vc a,&
  &\vc a+(\vc b+\vc c)=(\vc a+\vc b)+\vc c,\notag\\
  &\vc a+\vc 0=\vc a,&
  &p\vc a+p\vc b=p(\vc a+\vc b),&
  &|\vc a+\vc b|\le|\vc a|+|\vc b|.\label{eq:addformula}
\end{align}



%If we combine scalar multiplication and addition, we obtain the following statement:
%\begin{miniitemize}
%  \item If $p$ and $q$ are numbers and $\vc a$ and $\vc b$ are vectors,
%  \[ p\vc a+q\vc b \]
%  is a vector. This is called \Emph*{linear combination} of $\vc a$ and $\vc b$.
%\end{miniitemize}
%Of course, $p$ and $q$ can be positive, zero, or negative.\addnote{They can actually be a complex number, but we consider only real numbers in this course.}
%\OutputNote



\begin{problems}
 \Problem[S] Vectors are drawn on the grid, which has a spacing of 1.
 \par\smallskip\par\usebox{\VectorSetA}
 \begin{enumerate}
   \item Describe $\vc a$ by using $\vvex$ (and a number).  Describe $\vc b$ and $\vc c$ by using $\vvey$.
   \item Describe $\vc B$ by using $\vc r$. Describe $\vc p$ by using $\vc r$.
   \item Describe $\vc A$ and $\vc B$ by using $\vvex$ and $\vvey$.
   \item Describe $\vvex$ and $\vvey$ by using $\vc A$ and $\vc B$.
    \Hint{See your answer of the previous problem.}
   \item A vector $\vc \beta$ has the same direction as $\vc b$ but its magnitude is 1. Describe $\vc \beta$ by using $\vc b$. Also, draw $\vc \beta$ in the above grid.
 \end{enumerate}
\end{problems}
\begin{problems}
 \Problem[S] Consider a vector $\vc s$ whose length is 3, i.e., $|\vc s|=3$.
 \begin{enumerate}
   \item Calculate the magnitude of $2\vc s$, $-3\vc s$, $0\vc s$, and $\dfrac{\vc s}{|\vc s|}$.
 \end{enumerate}
 Consider another vector $\vc w$, whose magnitude is not zero. We define $\vc{e_w}= \dfrac{\vc w}{|\vc w|}$.
 \begin{enumerate}[resume]
  \item Calculate the magnitude of $\vc{e_w}$.
 \end{enumerate}
Consider another vector $\vc t=-3\vc s$. Assume $k$ is a real number.
\begin{enumerate}[resume]
  \item Calculate the magnitude of $\vc t$ and $k\vc t$. \Hint{$k$ can be negative.}
  \item If $|\vc s+k\vc t|=0$, what is $k$?
  \item Find the magnitude and the direction of $\vc s+k\vc t$.
 \end{enumerate}
 \Problem[B] For each of the six equations in \eqref{eq:addformula}, explain why it is true. However, you need to explain it based only on the above definitions. Namely, you cannot use the expression like $\vc v=\spmat{a\\b}$ in the explanation.
\end{problems}

Here is one more important concept:
\begin{definition}{Unit vector}{unitvector}
If a vector has a magnitude of one, it is called a \Emph*{unit vector}.
\end{definition}
If $\vc a$ is not the zero vector,
\begin{itemize}
  \item the unit vector in the same direction as $\vc a$ is given by $\dfrac{\vc a}{|\vc a|}$.
\end{itemize}
We write it as $\vc{\hat{a}}$ (hat + arrow).
\Remark{In the textbook, vectors are written by $\vc{\mathbfup{a}}$ and unit vectors are written by $\hat{\mathbfup{a}}$. Since this notation may cause confusion among some students (especially in General Physics 2),
Sho will use $\vc{\hat a}$, i.e., put both hat and arrow, for unit vectors.
}

\begin{problems}
 \Problem[S] Assume $\vc e$ is a unit vector and $k$ is a real constant. Assume $\vc a\neq \vc 0$.
 \begin{enumerate}
   \item Find the magnitude of $-3\vc e$.
   \item Find the magnitude of $k\vc e$, $k^2\vc e$, $k^2\vc e/5$, and $-k\vc e$.
   \item Find the magnitude of $\dfrac{\vc a}{|\vc a|}$, $\dfrac{k\vc a}{|\vc a|}$, and $\dfrac{-k\vc a}{|\vc a|}$. \quad\Hint{Are they scalar? or vector?}
 \end{enumerate}
 \Problem[S] Consider $\vc a$ and $\vc b$, which satisfy $|\vc a|=3$ and $|\vc b|\neq 0$. Assume $k>0$.
 \begin{enumerate}
  \item Find the unit vector whose direction is the same as $\vc a$.
   \item Find the vector whose direction is the same as $\vc a$ and has a magnitude of 3.
   \item Find the vector whose direction is the same as $\vc a$ and has a magnitude of 6.
   \item Find the vector whose direction is the same as $\vc a$ and has a magnitude of $k$.
   \item Find the unit vector that is anti-parallel to $\vc a$.
   \item Find the vector that is anti-parallel to $\vc a$ and has a magnitude of $k$.
   \item Find the unit vector whose direction is the same as $\vc b$.
   \item Find the vector which is in the same direction as $\vc b$ and has a magnitude of $k$.
 \end{enumerate}
 \Problem[S] Consider two points A and B and assume the distance between them is 5. We can consider the vector $\vvv{AB}$, which is an arrow pointing from A to B.
  \begin{enumerate}
    \item Calculate $|\vvv{AB}|$.
    \item Describe the unit vector whose direction is the same as $\vvv{AB}$.
    \item Assume $\vc F=\displaystyle\frac{k}{4\pi}\frac{\vvv{AB}}{|\vvv{AB}|^3}$, where $k$ is a real number. Describe the direction and magnitude of $\vc F$.
  \end{enumerate}
 \Problem[A] Consider $\vc s$ and $\vc t$, which satisfy $|\vc s|=3$ and $|\vc t|=2$. Assume $k>0$.
 \begin{enumerate}
   \item Do we know $|\vc s+\vc t|$? If yes, calculate it. If no, explain why we do not know it.
   \item Calculate the magnitude of the vector $\dfrac{\vc s+\vc t}{|\vc s+\vc t|}$.
   \item Find the unit vector whose direction is the same as $\vc s+\vc t$.
   \item Find the vector which is in the same direction as $\vc s+\vc t$ and has a magnitude of $k$.
 \end{enumerate}
\Problem[B] Consider $\vc s$ and $\vc t$, which satisfy $|\vc s|=3$ and $|\vc t|=2$. Find the minimal and maximal values of $|\vc s+\vc t|$ and when they are achieved.
\end{problems}

\newpage

\subsection{Inner product}
Now we want to discuss inner product of vectors. However, we (still) have forgotten that vectors are described by numbers!
We have to define the inner product in a different way.
\begin{definition}{Inner product}{inner}
  For vectors $\vc a$ and $\vc b$, the \Emph*{inner product} of $\vc a$ and $\vc b$ is defined by
  \[ \vc a\cdot\vc b\deq|\vc a||\vc b|\cos\theta, \qquad \text{where $\theta$ is the angle between $\vc a$ and $\vc b$.} \]
\end{definition}
Notice this definition only uses the magnitude and the direction of vectors.
Now it is straightforward to prove the next formula \emph{(extremely important!)}
\begin{enumerate}[label=\textsf{(\arabic*)}]
\item
  For any vector $\vc a$,\qquad $\vc a\cdot\vc a=|\vc a|^2$
\end{enumerate}
\begin{quizzes}
\Quiz[S] Prove this formula only from the above definition. Prove $|\vc a|=\sqrt{\vc a\cdot\vc a}$.
\end{quizzes}
There are a few more important formulae: for any vectors $\vc a$, $\vc b$, and $\vc c$ and any number $k$,
\begin{enumerate}[label=\textsf{(\arabic*)},resume]
    \item $\vc a\cdot\vc b=\vc b\cdot\vc a$,
    \item $(k\vc a)\cdot\vc b=k(\vc a\cdot\vc b)$,
    \item $\vc a\cdot\vc b=0$ if $\vc a=\vc 0$, $\vc b=\vc 0$, or $\vc a$ is perpendicular to $\vc b$.
    \item[\textsf{(\arabic{enumi}')}] If $\vc a\neq \vc 0$, $\vc b\neq\vc 0$, and $\vc a\cdot\vc b=0$, then $\vc a$ is perpendicular to $\vc b$.
    \item $-|\vc a||\vc b|\le \vc a\cdot\vc b\le|\vc a||\vc b|$,
    \item $(\vc a+\vc b)\cdot\vc c=\vc a\cdot\vc c+\vc b\cdot \vc c$.
\end{enumerate}
\begin{quizzes}
  \Quiz[B] Explain why the above formulae (2) (3) (4) (4') and (5) are correct.
  \\\phantom.\hfill\Hint{We skip the proof of (6) because it is a bit complicated.}
  \Quiz[S] Explain why the following equations are correct.
  \begin{align*}
    \text{\textsf{(a)}}\quad &(\vc a+\vc b)\cdot\vc a=|\vc a|^2+\vc a\cdot\vc b &
    \text{\textsf{(b)}}\quad &|\vc a+\vc b|^2=|\vc a|^2+2\vc a\cdot\vc b+|\vc b|^2
  \end{align*}
\end{quizzes}

\begin{problems}
  \Problem[S] \relax\relax\label{q:ipbasic}%
  Two vectors $\vc a$ and $\vc b$ satisfy $|\vc a|=2$, $|\vc b|=3$, and $\vc a\cdot\vc b=3$. Assume $x$, $y$, $p$, and $q$ are real numbers.
  \begin{enumerate}
    \item Calculate the angle between $\vc a$ and $\vc b$.
    \item Calculate $|\vc a|^2+2\vc a\cdot\vc b+|\vc b|^2$.
    \item Calculate $|\vc a+\vc b|$.
    \item Explain why $|x\vc a+y\vc b|^2$ \Emph{is not} equal to $x^2+y^2$.
    \item Calculate the magnitude of the vectors $4\vc a+3\vc b$ and $x\vc a+y\vc b$.
   \end{enumerate}
  Two other vectors $\vc e_1$ and $\vc e_2$ satisfy $|\vc e_1|=1$, $|\vc e_2|=1$, and $\vc e_1\cdot\vc e_2=0$.
  \begin{enumerate}[resume]
    \item Calculate the angle between $\vc e_1$ and $\vc e_2$.
    \item Explain why $(p\vc e_1+q\vc e_2)\cdot(x\vc e_1+y\vc e_2)$ \Emph{is} equal to $px+qy$.
    \item Explain why $|x\vc e_1+y\vc e_2|^2$ \Emph{is} equal to $x^2+y^2$.
    \item Explain why $|x\vc e_1+y\vc e_2|$ \Emph{is} equal to $\sqrt{x^2+y^2}$.
    \item Calculate the magnitude of the vectors $4\vc e_1+3\vc e_2$ and $x\vc e_1+y\vc e_2$.
   \end{enumerate}
  \Problem[A] Consider the same vectors $\vc a$, $\vc b$, $\vc e_1$, and $\vc e_2$ given in the previous problem.

 Two vectors $\vc s$ and $\vc t$ are given by $\vc s=3\vc a+3\vc b$ and $\vc t=2\vc a-\vc b$.
  \begin{enumerate}
    \item Calculate $|\vc s|$, $|\vc t|$, and $\vc s\cdot\vc t$. Find the angle between $\vc s$ and $\vc t$.
    \item Assume $\vc u=\vc a+k\vc b$ is perpendicular to $\vc s$. Find the value of $k$.
  \end{enumerate}
  Two vectors $\vc S$ and $\vc T$ are given by $\vc S=3\vc e_1+3\vc e_2$ and $\vc T=2\vc e_1-\vc e_2$.
  \begin{enumerate}[resume]
    \item Calculate $|\vc S|$, $|\vc T|$, and $\vc S\cdot\vc T$. Find the angle between $\vc S$ and $\vc T$.
    \item Assume $\vc U=\vc a+k'\vc b$ is perpendicular to $\vc S$. Find the value of $k'$.
  \end{enumerate}
  Now, assume $\vc p=a\vc e_1+b\vc e_2$, which is not equal to $\vc 0$.
  \begin{enumerate}[resume]
    \item Calculate $|\vc p|$, $\vc p\cdot\vc e_1$, and $\vc p\cdot\vc e_2$. Check that $\vc p=(\vc p\cdot\vc e_1)\vc e_1+(\vc p\cdot\vc e_2)\vc e_2$.
    \item Find the unit vector whose direction is the same as $\vc p$.
  \end{enumerate}
\end{problems}

\begin{problems}
  \Problem[B] Let $ \vc a=a_1\vc i+a_2\vc j+a_3\vc k$ and $\vc b=b_1\vc i+b_2\vc j+b_3\vc k$.
   \label{q:ipexpand}
  \begin{enumerate}
    \item First, assume that $\vc i$, $\vc j$, and $\vc k$ are (any) vectors. Fill in the blanks:
    \begin{align*}
      &|\vc a|^2=a_1^2|\vc i|^2+2a_1a_2(\vc i\cdot\vc j)+\GrayBox{\strut\hspace{8em}}+a_3^2|\vc k|^2,\\
      &\vc a\cdot\vc b=a_1b_1|\vc i|^2+a_1b_2(\vc i\cdot\vc j)+\GrayBox{\strut\hspace{8em}}+a_3b_3|\vc k|^2.
    \end{align*}
  \end{enumerate}
  Now, assume they satisfy $|\vc i|=|\vc j|=|\vc k|=1$ and $\vc i\cdot\vc j=\vc j\cdot\vc k=\vc k\cdot\vc i=0$.
  \begin{enumerate}[resume]
    \item Simplify the above two equations.
    \item By using $a_{1,2,3}$, $b_{1,2,3}$, $\vc i$, $\vc j$, and $\vc k$, express $\vc{\hat a}$ (the unit vector in $\vc a$'s direction) and the angle between $\vc a$ and $\vc b$.
  \end{enumerate}

  \Problem[B] \relax\relax\label{q:cpexpand}%
  Three vectors $\vc A$, $\vc B$, and $\vc C$ satisfy $|\vc A|=2$, $|\vc B|=3$, $\vc A\cdot \vc B=3\sqrt2$, and $\vc A\cdot\vc C=-1$.
  \begin{enumerate}
    \item Find the angle between $\vc A$ and $\vc B$.
    \item Calculate $|\vc A+\vc B|^2$, $|\vc A-\vc B|^2$, and $|2\vc A+4\vc B|^2$.
    \item Calculate $(\vc A-2\vc B)\cdot(2\vc A+\vc B+\vc C)+2\vc B\cdot\vc C$.
    \item Assume $|\vc A+k\vc B|=\sqrt{10}$. Find the value of $k$.
    \item Assume $\vc B+c\vc C$ is perpendicular to $\vc A$. Find the value of $c$.
    \item What do we know about the values of $|\vc C|$ and $|\vc A-\vc C|$?
  \end{enumerate}
  \Problem[C] \relax\label{q:vecbasisrot}
  Two vectors $\vvex$ and $\vvey$ satisfy  $|\vvex|=|\vvey|=1$ and $\vvex\cdot\vvey=0$. Define
\begin{align*}
  \vc{a_x}&=\frac12\vvex+\frac12\vvey,&
  \vc{f_x}&=\vvex\cos\theta+\vvey\sin\theta,  \\
  \vc{a_y}&=\vvex+\frac12\vvey,&
  \vc{f_y}&=-\vvex\sin\theta+\vvey\cos\theta.
\end{align*}
First, consider $\vc A=A\vc{a_x}+B\vc{a_y}$.
\begin{enumerate}
  \item Calculate $\vc{a_x}\cdot\vc{a_y}$, $|\vc{a_x}|$, and $|\vc {a_y}|$. Also, calculate $|\vc A|$.
  \item Write down $\vc A$ by using $\vc{e_x}$, $\vc{e_y}$, $A$, and $B$.
\end{enumerate}
Next, consider $\vc P=P\vc{f_x}+Q\vc{f_y}$ and $\vc S=S\vc{f_x}+T\vc{f_y}$.
\begin{enumerate}[resume]
  \item Calculate $\vc{f_x}\cdot\vc{f_y}$, $|\vc{f_x}|$, and $|\vc {f_y}|$.
  \item Calculate $\vc P\cdot\vc S$, $|\vc P|$, and $|\vc S|$.
  \item Write down $\vc P$ by using $\vvex$, $\vvey$, $P$, $Q$, and $\theta$.
  \item Write down $\vvex$ and $\vvey$ by using $\vc{f_x}$, $\vc{f_y}$, and $\theta$.
  \item Verify the following equation is correct:
  \[\vc P=(\vc P\cdot\vc{f_x})\vc{f_x}+(\vc P\cdot\vc{f_y})\vc{f_y}
  =(\vc P\cdot\vc{e_x})\vc{e_x}+(\vc P\cdot\vc{e_y})\vc{e_y}\]
\end{enumerate}
\end{problems}

\pagebreak
\subsection{Cross product}\label{sec:cross}
The definition of the cross product may also be different from highschool mathematics.
\begin{definition}{Cross product}{cross}
  For vectors $\vc a$ and $\vc b$, the \Emph*{cross product} of $\vc a$ and $\vc b$ is defined as follows:
\begin{itemize}
  \item it is a vector and denoted by $\vc a\times\vc b$;
  \item its magnitude $|\vc a\times\vc b|$ is given by $|\vc a||\vc b|\sin\theta$, where $\theta$ is the angle between $\vc a$ and $\vc b$;
  \item if the magnitude is not zero, its direction is specified so that
  \begin{itemize}
    \item $\vc a\times\vc b$ is perpendicular to both $\vc a$ and $\vc b$;
    \item The triplet $(\vc a,\vc b,\vc a\times\vc b)$ satisfies the right-hand rule.
  \end{itemize}
\end{itemize}
\end{definition}
%
\def\TEMP{More precisely, if $\vc p$ and $\vc q$ are on the sheet and the direction of $\vc q$ is $\theta$ clockwise from $\vc p$ with $0^\circ<\theta<180^\circ$, then $\vc p\times\vc q$ is normal to the sheet and points out of the sheet. If $180^\circ<\theta<360^\circ$, then $\vc p\times \vc q$ is normal to the sheet and points toward the sheet.}
%
Here, ``$(\vc p,\vc q,\vc r) $ satisfies the right-hand rule'' means that if you hold your right hand so that your thumb points in the direction of $\vc p$ and your index finger $\vc q$, then your middle finger points in the direction of $\vc r$. Or, if you draw $\vc p$ upward and $\vc q$ leftward on a sheet, then $\vc p\times\vc q$ points out of the sheet toward you\addnote{\noexpand\TEMP}.%

\OutputNote

The following properties are important: for any vectors $\vc a$, $\vc b$, and $\vc c$ and any real number $k$,
\begin{enumerate}[label=\textsf{(\arabic*)}]
    \item $\vc a\times\vc a=\vc 0$.
    \item $\vc a\times\vc b=-\vc b\times\vc a$.
    \item $(\vc a+\vc b)\times\vc c=\vc a\times\vc c+\vc b\times \vc c$.
    \item $0\le|\vc a\times\vc b|\le|\vc a||\vc b|$.
    \item $(k\vc a)\times\vc b=\vc a\times(k\vc b)=k(\vc a\times\vc b)$.
\end{enumerate}
Furthermore, cross products have the following properties:
we can prove the following formulae:
\begin{enumerate}[label=\textsf{(\arabic*)},resume]
    \item $\vc a\times\vc 0=\vc 0\times\vc a=\vc 0$.
    \item $\vc a\times\vc b=\vc 0$ if $\vc a$ and $\vc b$ are parallel or anti-parallel.
    \item $\vc a\cdot(\vc a\times\vc b)=\vc b\cdot(\vc a\times\vc b)=0$.
    \item $(\vc a\cdot\vc b)^2+|\vc a\times\vc b|^2=|\vc a|^2|\vc b|^2$.
    \item $\vc a\cdot(\vc b\times\vc c)=\vc b\cdot(\vc c\times\vc a)=\vc c\cdot(\vc a\times\vc b)$.
    \item $\vc a\times(\vc b\times\vc c)=(\vc a\cdot\vc c)\vc b-(\vc a\cdot\vc b)\vc c$.
\end{enumerate}
\begin{quizzes}
  \Quiz[B] Explain why the above formulae (1), (2), (4), and (6)--(9) are valid. You can use only the above definitions.~\Hint{We skip (3) because its proof is extremely complicated.}

  \Quiz[S]{}\QSub{1}  Explain why $(\vc a+\vc b)\times \vc a=-\vc a\times\vc b$.
  \\\TAB\QSub{2}
    See the vectors in Problem~\ref{q:vecbasic}. Describe $\vc a\times \vc b$, $\vc b\times \vc a$, $\vc b\times \vc c$, and $\vc a\times \vc p$.
\end{quizzes}

\begin{problems}
  \Problem[A] Three vectors $\vvex$, $\vvey$, and $\vvez$ satisfy
  \[ |\vvex|=|\vvey|=|\vvez|=1,
   \quad \vvex\times\vvey=\vvez,
   \quad \vvey\times\vvez=\vvex,
   \quad \vvez\times\vvex=\vvey.
  \]
  \begin{enumerate}
    \item Calculate $\vvex\cdot\vvey$. Find the angles among $\vvex$, $\vvey$, and $\vvez$.
    \item Expand $(a\vvex+b\vvey+c\vvez)\times(p\vvex+q\vvey+r\vvez)$ and simplify.\\
      \phantom.\hfill\Hint{You are asked to do the similar operation as in Problem~\ref{q:ipexpand}\QSub1.}
  \end{enumerate}
\end{problems}


\subsection[Operation on Vectors]{Operation on vectors — Introduction to the Vector/Scalar/Not game}
Let's summarize what we have discussed: with $k$ being a real number and $\vc a$ and $\vc b$ being vectors,
\begin{tabbing}
\fakebullet \= \Emph*{scalar multiplication}:~~\= $\vc a\times\vc b$\quad \= \kill
\fakebullet \> \Emph*{magnitude}:              \> $|\vc a|$           \> ($\leftarrow$ scalar)\\
\fakebullet \> \Emph*{scalar multiplication}:~~\> $k\vc a$            \> ($\leftarrow$ vector)\\
\fakebullet \> \Emph*{addition}:               \> $\vc a+\vc b$       \> ($\leftarrow$ vector)\\
\fakebullet \> \Emph*{inner product}:          \> $\vc a\cdot\vc b$   \> ($\leftarrow$ scalar)\\
\fakebullet \> \Emph*{cross product}:          \> $\vc a\times\vc b$  \> ($\leftarrow$ vector)
\end{tabbing}
For vectors, \Emph{you only need to learn these operations!} In other words, only these operations are possible on vectors, and any complicated expressions, such as
\[
\frac{\vc a\times \vc b+(\vc x\cdot\vc y)\vc d}{(\vc a\cdot\vc b)|\vc b\times\vc c+\vc d|}
\]
can be obtained by combining these five.
\Remark{The numerator is the sum of two vectors: $\vc a\times\vc b$ plus $(\vc x\cdot\vc y)\vc d$. Here, $\vc x\cdot \vc y$ is just a number. The denominator has two numbers: $\vc a\cdot\vc b$ and $|\vc b\times\vc c+\vc d|$, where the latter is the magnitude of the vector $\vc b\times\vc c+\vc d$.}
Therefore, \Emph{we cannot do any other operations}. For example, these are all \Emph{\C{invalid}} expression:
\begin{align*}
    \C{(\vc a)^2},
 && \C{\frac{1}{\,\vc a\,}},
 && \C{k+\vc a},
 && \C{\vc a\vc b},
 && \C{\sqrt{\vc a}},
 && \C{\frac{\vc a}{\,\vc a\,}},
 && \C{a+\vc a},
 && \C{\vc a\cdot\vc b+\vc a\times\vc b},
 && \C{(\vc a+\vc b)\vc c};
\end{align*}
these never appears in correct math equations and do not express any mathematical concept.

Now, let's try the \Emph*{Vector/Scalar/Not game}.
\begin{problems}
\Problem[S] For each expression, answer \Emph*{V} if it is vector; \Emph*{S} if it is scalar, and \Emph*{N} if it is an invalid expression (neither vector nor scalar).
\begin{align*}
  & 3+|\vc a|
  && \vc a-\vc b
  && \vc a\vc b
  && \vc a/\vc b
  && \vc p\times(\vc q\times\vc r)
\\[.5em]& 3\vc a
  && \vc 0+1
  && \vc 0
  && -\vc a
  && \vc p\cdot(\vc q\times\vc r)
\\[.5em]& 3\times\vc a
  && \vc k\times\vc a
  && x+|\vc a|
  && |\vc a|^{-1}\vc a
  && p\times(\vc q\times\vc r)
\\[.5em]& \frac{1}{\vc x+\vc y}
  && \frac{x+y}{|\vc x+\vc y|}
  && \frac{\vc x+\vc y}{|\vc x+\vc y|}
  && \frac{\vc x+\vc y}{\vc x+\vc y}
  && \vc p(\vc q\times\vc r)
\\[.5em]& \frac{1}{(\vc a\cdot\vc b)^2}
  && \frac{\vc a}{(\vc a\cdot\vc b)^2}
  && \frac{1}{(\vc a)^2}
  && \frac{1}{|\vc a|^2}
  && p(\vc q\times\vc r)
\end{align*}
\end{problems}

\newpage
\subsection{Axes and Components}\label{sec:component}
Sho wants to emphasize a remarkable fact: in all our discussions so far, \Emph{we have not talked about the components of vectors!} In highschool, vectors are expressed as $\vc a=(5,-7)$, where $5$ is the $x$-component and $-7$ is the $y$-component. However, our discussion so far does not include this form of vectors; we have not defined $x$- or $y$-axes yet.

This fact demonstrates that the aforementioned concepts, including inner product and cross product, are free from a coordinate system ($x$, $y$, and $z$-axes) or from the component-wise notation. \Emph{Vectors are independent of any coordinate system} and the notations $(5,-7)$ etc.\ are \emph{superficial}.

It is not so surprising in physics. Physics discusses phenomena in this Universe, but our Universe does not have any $x$- or $y$-direction predefined. Axes and coordinates are artificial, human-made concepts. \Emph{We} define $x$-, $y$-, and $z$-axes so that we can easily calculate vectors and we can clearly express the vector direction in this three-dimensional universe.


Now, in order to discuss our \emph{three}-dimensional universe, we will introduce coordinate systems with \emph{three} axes.
The \Emph{Cartesian coordinate system} is the simplest coordinate system. It has ${x}$-axis, ${y}$-axis, and ${z}$-axis, and these axes are defined so that each one is perpendicular to the other two axes.
Namely, if we \emph{define}
\begin{definition}{Unit vectors representing the axes}{unit}
  The symbol $\vvex$ denotes the unit vector in the positive $x$-direction; $\vvey$ and $\vvez$ are similarly defined.\\[.5em]
  (In the textbook, they are denoted by $\hat{\mathbfup{i}}$, $\hat{\mathbfup{j}}$, and $\hat{\mathbfup{k}}$. Please not be confused.)
\end{definition}
then
\begin{miniitemize}
  \item any vectors $k\vvex$ with $k>0$ (such as $\vvex$, $2\vvex$, or $\pi\vvex$) direct toward the positive $x$-direction,
  \item any vectors $k\vvez$ with $k<0$ direct toward the negative $z$-direction, and
  \item the \emph{basis vectors} satisfy $\vvex\cdot\vvey=0$, $\vvey\cdot\vvez=0$, and $\vvez\cdot\vvex=0$.
\end{miniitemize}
Beware that the Cartesian coordinate system is not unique but two-fold. Draw $x$- and $y$-axis so that $\vvex\cdot\vvey=0$ and observe you have two (and only two) options for the $z$-axis direction.
In physics, we always use \Emph*{right-handed Cartesian coordinate system}:
\begin{definition}{Right-handed Cartesiaon Coordinate system}{cartesian}
  The \Emph*{Cartesian coordinate system} is characterized by unit vectors $(\vvex,\vvey,\vvez)$ satisfying
  \begin{align}
    &|\vvex|=|\vvey|=|\vvez|=1,&
    &\vvex\cdot\vvey=\vvey\cdot\vvez=\vvez\cdot\vvex=0;
  \end{align}
  namely, with $x$-, $y$-, and $z$-axes chosen so that they are perpendicular to each other.

 \Emph*{Right-handed Cartesian coordinate system} is defined so that the triplet $(\vvex,\vvey,\vvez)$ satisfies the right-hand rule. In other words, they satisfy
  \begin{align}
    &\vvex\times\vvey=\vvez,
    &&\vvey\times\vvez=\vvex,
    &&\vvez\times\vvex=\vvey
  \end{align}
\end{definition}
So, you \emph{have to} draw $z$-axis so that (1) it is normal to both $x$- and $y$-axes and (2) they corresponds to
 $\langle x$-axis\ $\to$\ thumb$\rangle$,
 $\langle y$-axis\ $\to$\ index finger$\rangle$,
 $\langle z$-axis\ $\to$\ middle finger$\rangle$ of your \Emph{right} hand.
(Observe that you get a different result if you use your left hand.)
\Remark{The definition of cross product uses right-hand rule because we use right-handed Cartesian coordinate system.}

\pagebreak

\Emph{After we define the axes}, we can express vectors in their components:
\begin{definition}{Components of vectors}{components}
  If a vector $\vc v$ can be written by
  \[ \vc v=A\vvex+B\vvey+C\vvez, \]
  we say $A$ is the $x$-component of $\vc v$, $B$ is the $y$-component of $\vc v$, and $C$ is the $z$-component of $\vc v$. We write this vector $\vc v$ as
  \[
    \vc v=\pmat{A\\B\\C},
  \]
  where a Cartesian coordinate system is assumed (but it is \emph{understood} and not explicitly written).
\end{definition}
\newcommand*\idrel{\mathrel{\stackrel{!!}{~\longleftrightarrow~}}}
It may be helpful to understand this rule as the identification of
$
  A\vvex+B\vvey+C\vvez \idrel \pmat{A\\B\\C}.
$
Check that $\vvex\idrel\pmat{1\\0\\0}$,\quad$\vvey\idrel\pmat{0\\1\\0}$,\quad$\vvez\idrel\pmat{0\\0\\1}$,\quad etc.\ are correct.
\Remark{You probably use the horizontal notation $(A,B,C)$ in highschool. In university, try to use this vertical notation $\spmat{A\\B\\C}$ because it is consistent with the notation of matrices and thus natural when you learn linear algebra.}

\noindent In practice, we simply use the equal sign:
$\displaystyle \vc v=2\vvex+\vvey-3\vvez=\pmat{2\\1\\-3}$, $\displaystyle -\vvez=\pmat{0\\0\\-1}$, etc.

\Remark{Consider the equation
\[
  \pmat{3\\2}+\pmat{5\\-3}=\pmat{8\\-1}.
\]
This was a kind of \emph{definition} in high school mathematics, but in this Boot Camp, this equation \emph{should be proved} by the following way:
\[
  \pmat{3\\2}+\pmat{5\\-3}\idrel(3\vvex+2\vvey)+(5\vvex-3\vvey)=(3+5)\vvex+(2-3)\vvey\idrel\pmat{8\\-1},
\]
where we have used Eq.~\eqref{eq:addformula} at the equal ($=$) symbol.
}


\Remark{It might be interesting to observe the following transformation:
\begin{align*}
  \vc x&=\makebox[10em][l]{$(p\vvex+q\vvey+r\vvez)    $} \idrel \makebox[3em][r]{$\vc x=\,$}\pmat{p\\q\\r}\\
       &=\makebox[10em][l]{$(p\vvex)+(q\vvey)+(r\vvez)$} \idrel \makebox[3em][r]{$=\,$}\pmat{p\\0\\0}+\pmat{0\\q\\0}+\pmat{0\\0\\r}\\
       &=\makebox[10em][l]{$p(\vvex)+q(\vvey)+r(\vvez)$} \idrel \makebox[3em][r]{$=\,$}p\pmat{1\\0\\0}+q\pmat{0\\1\\0}+r\pmat{0\\0\\1}
\end{align*}
In university mathematics, vectors are \emph{defined} based on these equations.}

\begin{problems}
\Problem[A] Let $\vc A=\pmat{a\\b\\c}$ and $\vc B=\pmat{p\\q\\r}$. \Emph{Prove} the following equations.
\begin{enumerate}
  \item $|\vc A|=\sqrt{a^2+b^2+c^2}$
  \item $\vc A\cdot\vc B=ap+bq+cr$
  \item $\vc A\times\vc B=\pmat{br-cq\\cp-ar\\aq-bp}$
\end{enumerate}
\Hint{Actually, you have already proved these equations in the previous problems.
  For (1) or (2), check your answer to Problems \ref{q:ipbasic} and \ref{q:ipexpand}. For (3), see Problem \ref{q:cpexpand}.}
  \Problem[C] Verify that $\displaystyle \vc v=\pmat{\vc v\cdot\vvex\\\vc v\cdot\vvey\\\vc v\cdot\vvez}$. Review Problem \ref{q:vecbasisrot} and think why Sho includes that problem in this Boot Camp.
\end{problems}

\newpage
\subsection{Position Vectors}
\begin{problems}
  \Problem[S] Three points are defined at the right.\par
  \noindent\begin{minipage}[b]{0.55\textwidth}
    \begin{enumerate}
      \item Define a point O (at any place).
    \end{enumerate}
    Based on your point O, we define $\vc a=\vvv{OA}$, $\vc{b}=\vvv{OB}$, and $\vc{c}=\vvv{OC}$.
    \begin{enumerate}[resume]
      \item Draw $\vc a$, $\vc c$, and $\vc b+\vc c$.
      \item Draw $\vc b-\vc a$ and $\vc a-\vc c$.
      \item Express $\plen{AB}$ by using $\vc a$ and $\vc b$.
    \end{enumerate}
  \end{minipage}
    \begin{minipage}[b]{0.35\textwidth}
    \vspace{-1.5em}\par
    \hfill\usebox{\VectorSetB}
    \par\vspace{1em}
  \end{minipage}
\begin{enumerate}[start=5]
  \item We want to define M so that $\vvv{OM}=(\vc b+\vc c)/2$. Draw $(\vc b+\vc c)/2$. Where is M?
  \item Let G be the geometric center of the triangle ABC. Show that $\vvv{OG}=(\vc a+\vc b+\vc c)/3$.
\end{enumerate}
\end{problems}
\noindent
This problem describes the last topic of this Boot Camp, which is important in General Physics 2.
Namely, by defining a point O, we can identify a position (in $xyz$-space) and a vector.
\begin{align*}
&\text{point A}\idrel \vc a,&
&\text{point B}\idrel \vc b,&
&\text{point M}\idrel \vc m,&
&\text{point G}\idrel \vc g, \text{~etc.}
\end{align*}
\begin{definition}{Position vectors}{posvec}
  We can identify a point P by a vector $\vvv{OP}$, which is called a \Emph{position vector} of Point P.
  This is \Emph{one-to-one correspondence}, i.e., $\vvv{OP}=\vvv{OQ}$ if and only if P and Q are the same point; different points have different position vectors, and different vectors represent different points.
\end{definition}
Sho often writes $\posvec Pp$ to express that $\vc p$ is the position vector of Point P, i.e., $\vc p=\vvv{OP}$.

Position vectors are useful because of the following properties:
\begin{itemize}
  \item $\vvv{AB}=\vc b-\vc a$ for points A and B.
  \item The midpoint of A and B is given by $(\vc a+\vc b)/2$.
  \item Any points between A and B is given by $(1-t)\vc a+t\vc b$ with $0<t<1$.
  \item Consider a line (with infinite length) passing through A and B. Any points on the line is given by $(1-t)\vc a+t\vc b$, where $t$ is any real number.
  \item The geometric center of A, B, and C is given by $(\vc a+\vc b+\vc c)/3$.
\end{itemize}
Notice that these properties are independent of the choice of Point O, as you have observed in the previous problem.




\begin{quizzes}
  \Quiz[S] Draw two Points $\posvec Aa$ and $\posvec Bb$ as you like. Namely, we consider two points A and B and define $\vc a\coloneq\vvv{OA}$ and $\vc b\coloneq\vvv{OB}$.
  \begin{enumerate}
    \item Explain why $\vvv{AB}=\vc b-\vc a$.
    \item Explain why $\plen{AB}=|\vc a-\vc b|$, where $\plen{AB}$ is the distance between A and B.
    \item Explain why $\plen{AB}=\sqrt{|\vc a|^2-2\vc a\cdot\vc b+|\vc b|^2}$.
    \item Consider a point P defined by $\vvv{OP}=\vvv{OA}+(1/2)\vvv{AB}$. Draw Point P.
    \item Check that $\vvv{OP}=(\vc a+\vc b)/2$.
  \end{enumerate}
  \end{quizzes}


\begin{problems}
  \Problem[S] Consider two points $\posvec Aa$ and $\posvec Bb$. Two vectors are defined as
  \begin{align*}
    & \vc F_1=\frac{q}{4\pi C|\vc b-\vc a|^3}\left(\vc b-\vc a\right)
    &&\vc F_2=\frac{q}{4\pi C|\vc a-\vc b|^3}\left(\vc a-\vc b\right),
  \end{align*}
  where $q$ and $C$ are real constants; $q\neq 0$ and $C>0$, but $q$ can be positive and negative.
  \begin{enumerate}
    \item Express $\vc F_1$ by using $\vvv{AB}$ but not using $\vc a$ and $\vc b$.
    \item Explain why $\vc F_1=-\vc F_2$.
    \item Calculate the magnitude of $\vc F_1$ and $\vc F_2$.
    \item Describe the direction of $\vc F_1$ and $\vc F_2$.\\
      \Hint{Answer will be $\langle$it is in the same direction as $\vvv{AB}$$\rangle$ or $\langle$it is anti-parallel to $\vvv{AB}$$\rangle$.}
  \end{enumerate}
  Assume $\vc e_1$ is the unit vector in the same direction as $\vc F_1$ and $\vc e_2$ is the unit vector in the same direction as $\vc F_2$.
  \begin{enumerate}[resume]
    \item Explain why $\vc e_1=\dfrac{q}{|q|}\dfrac{\vc b-\vc a}{|\vc b-\vc a|}$.
    \item Explain why $\vc e_1$ and $\vc e_2$ are anti-parallel.
    \item Explain why $\vc F_1=\dfrac{|q|}{4\pi C}\dfrac{\vc e_1}{|\vc b-\vc a|^2}$.
  \end{enumerate}
\end{problems}


\begin{problems}
\Problem[S] Points $\posvec Aa$, $\posvec Bb$, and $\posvec Cc$ are drawn on the grid, which has a spacing of 1.

\medskip

\par
\noindent\begin{minipage}[b]{0.45\textwidth}
  \begin{enumerate}
    \item Express $\vc a$ and $\vc c$ by using $\vvex$ and $\vvey$.
    \item Express $\vc a$ and $\vc c$ by its components, i.e., in the $\spmat{x\\y}$ notation.
    \item Express $\vvv{AB}$, $\vvv{AC}$, $\vvv{BC}$, $\vvv{BA}$, and $\vvv{CB}$ by using $\vc a$, $\vc b$, and $\vc c$.
    \item Express $\plen{AB}$ by using $\vc a$ and $\vc b$.
    \item Find $\vvv{AB}\cdot\vvv{AC}$, $|\vvv{AB}|$, and $|\vvv{AC}|$.
    \item Find $\pangle{BAC}$.
    \item Find the area of the triangle ABC.
    \item Calculate $|\vvv{AB}\times\vvv{AC}|/2$.
  \end{enumerate}
\end{minipage}\kern1.5em
\begin{minipage}[b]{0.44\textwidth}
  \usebox{\VectorSetC} \par\vspace{2em}
\end{minipage}

\smallskip

Now, we define a point $\posvec Pp$ by $\vc p=x\vvex+y\vvey=\pmat{x\\y}$.
\begin{enumerate}[start=9]
\item Express $\vvv{AP}$ and $\vvv{BP}$ by using $\vc a$, $\vc b$, and $\vc p$.
\item Express $\vvv{AP}$ and $\vvv{BP}$ by using $\vvex$, $\vvey$, $x$, and $y$.
\item If $\vvv{AP}\cdot\vvv{BP}=0$ and $y=2$, what is the value of $x$? Draw the point $(x,y)$ on the grid.
\end{enumerate}

\Problem[B] Let $\posvec Aa$ at $(1,3)$ as in the previous problem. We define
\[f(x,y)=\frac{1}{\sqrt{(x-1)^2+(y-3)^2}}.  \]
We can understand that this function $f(x,y)$ is defined on the $xy$-plane. Namely, a number $f(x,y)$ is associated for each point P$(x,y)$.
\begin{enumerate}
  \item Verify the following: if we define $\vc p=\pmat{x\\y}$, then $f(x,y)=\dfrac{1}{|\vc p-\vc a|}$.
\end{enumerate}
 Now, we can identify $f(x,y)\idrel f(\vc p)$\quad from\quad $\mathrm{P}=(x,y)\idrel \vc p=x\vvex+y\vvey$.
 \begin{enumerate}[start=2]
  \item Express the following function $g$ and $h$ by using $x$ and $y$ (and other numbers), where $\vc b=5(\vvex+\vvey)$:
  \[g(\vc p)=\frac{1}{|\vc p-\vc a|^2},\qquad
  h(\vc p)=|\vc p-\vc a|+|\vc p-\vc b|.
  \]
\end{enumerate}
\end{problems}

\newpage
\subsection{Extra Exercise}
These problems are taken from \JA{高校数学例題&問題集} \url{https://web.math-aquarium.jp/} thanks to the author's generosity. You may find explanations and more problems at (in Japanese)\\
\phantom{extra space} \url{https://web.math-aquarium.jp/rennsyuu-heimennjounobekutoru-y.pdf}

\smallskip

\noindent\Hint[Note:~]{Some problems are skipped.}

\bigskip

\begingroup
\renewcommand\theenumi{\arabic{enumi}}
\noindent
\begin{minipage}[t]{0.55\textwidth}
\begin{enumerate}
 \itemA Find the following from the right figure.
\begin{enumerate}[leftmargin=1.5em,labelsep=-0.3em]
 \item equal vectors.
 \item vectors of equal magnitude.
 \item vectors with the same direction.
\end{enumerate}
\end{enumerate}
\end{minipage}
\begin{minipage}[t]{0.4\textwidth}\kern20pt\vspace{-2.5em}
 \begin{tikzpicture}[scale=0.6]
\draw [help lines,thin,dashed] (-0.2,0.6) grid (10.2,5.4);
\draw[->,very thick] (3,5)--node [below] {$\vc a$} ++(-2,-1);
\draw[->,very thick] (4,5)--node [left] {$\vc b$} ++(0,-2);
\draw[->,very thick] (8,3)--node [left] {$\vc c$} ++(-1,2);
\draw[->,very thick] (3,2)--node [above] {$\vc d$} ++(-2,0);
\draw[->,very thick] (7,3)--node [above] {$\vc e$} ++(-4,-2);
\draw[->,very thick] (9,2)--node [above] {$\vc f$} ++(-2,-1);
\end{tikzpicture}
\end{minipage}

\begin{enumerate}[start=3]
\itemA
\begin{enumerate}[leftmargin=1.5em,labelsep=-0.3em]
  \item A vector $\vc x$ satisfies $2(\vc a+2\vc x)-4\vc a=5(\vc x-3\vc b)$. Express $\vc x$ by $\vc a$ and $\vc b$.
 \item Vectors $\vc x$ and $\vc y$ satisfy $\vc x+\vc y=\vc a$ and $3\vc x+2\vc y=\vc b$. Express them by $\vc a$ and $\vc b$.
\end{enumerate}
\itemA Consider a triangle ABC, where $\plen{AB}=1$, $\plen{AC}=\sqrt5$, $\plen{BC}=2$, and $\angle{\mathrm B}=90^\circ$.
Let $\vc e$ be a unit vector whose direction is the same as $\vvv{BC}$. Express $\vc e$ by $\vvv{AB}$ and $\vvv{AC}$.
 \itemA Let $\vc a=\pmat{2\\-4}$, $\vc b=\pmat{5\\ -3}$, and $\vc c=-3\vc a+2\vc b$. Find the components of $\vc c$ and calculate $|\vc c|$.
 \itemB Let $\vc a=\pmat{2\\-4}$, $\vc b=\pmat{5\\ -3}$, and $\vc p=\pmat{7\\0}$. Find $s$ and $t$ that satisfy $\vc p=s\vc a+t\vc b$.
\itemB
\begin{enumerate}[leftmargin=1.5em,labelsep=-0.3em]
\item Four points A$(2,-4)$, B$(5,-3)$, C$(2,1)$, and D are on $xy$-plane, where the quadrilateral ABCD is a parallelogram. Find the coordinate of D.
\item Two vectors $\vc a=\pmat{2\\-4}$ and $\vc b=\pmat{5+t\\-3-t}$ are parallel. What is $t$?
\end{enumerate}
\itemA
\begin{enumerate}[leftmargin=1.5em,labelsep=-0.3em]
\item Let $\vc a$ and $\vc b$ satisfy $|\vc a|=3$ and $|\vc b|=2$ and form an angle $45^\circ$. Find $\vipro a b$.
\item Let $\vc a=\pmat{2\\-4}$ and $\vc b=\pmat{5\\3}$. Find $\vipro ab$.
\end{enumerate}
\itemA
\begin{enumerate}[leftmargin=1.5em,labelsep=-0.3em]
\item Let $\vc a=\pmat{3\\7}$ and $\vc b=\pmat{-5\\-2}$. Find the angle formed by them.
\item Two vectors $\vc a=\pmat{2\\-4}$ and $\vc b=\pmat{5+x\\3+x}$ are normal to each other. Find $x$.
\end{enumerate}
\itemB Two vectors $\vc a$ and $\vc b$ satisfy $|\vc a|=3$, $|\vc b|=4$, and $|\vc a+2\vc b|=7$. Find the angle $\theta$ formed by them, assuming $0\le\theta\le\pi$.

\item[\GB{\sffamily\bfseries[12]}] \begin{enumerate}[leftmargin=1.5em,labelsep=-0.3em]
    \item Consider two points $\posvec Aa$ and $\posvec Bb$. Two points P and M are on the line AB. They are between A and B and satisfy $\overline{\mathrm{AP}}:\overline{\mathrm{PB}}=5:3$ and $\overline{\mathrm{AM}}:\overline{\mathrm{MB}}=1:1$. Describe the position vectors of P and M by using $\vc a$ and $\vc b$.
\item Consider a triangle ABC with points $\posvec Aa$, $\posvec Bb$, and $\posvec Cc$. Let $\posvec Gg$ be its geometric center and $\posvec{H}{h}$ be the geometric center of the triangle GBC. Describe $\vc h$ by $\vc a$, $\vc b$, and $\vc c$.
\end{enumerate}
\item[\GC{\sffamily\bfseries[15]}] Consider a triangle ABC and let O be its circumcenter. Consider a point H satisfying $\vvv{OH}=\vvv{OA}+\vvv{OB}+\vvv{OC}$. Prove H is the orthocenter of the triangle ABC.
\end{enumerate}
\endgroup

\end{document}
