%!TEX program=lualatex

\def\LectureName/{General Physics I}
%\def\LectureNumber/{}
%\def\LectureDate/{}
\PassOptionsToPackage{fleqn}{amsmath}
\PassOptionsToPackage{hyperfootnotes=false}{hyperref}
\documentclass[11pt,pdfa,lastpage]{MishoNote}
\title{General Physics I: Integral Boot Camp}
\author{Sho Iwamoto}
\hypersetup{
  pdflang={en-US},
  pdfauthortitle={Assistant Professor, National Sun Yat-sen University},
  pdfsubject={Integral Boot Camp as a preparation to General Physics 1 and 2 lectures.},
  pdfcontactemail={iwamoto@g-mail.nsysu.edu.tw},
  pdfcontacturl={https://www2.nsysu.edu.tw/iwamoto/},
  pdfcaptionwriter={Sho Iwamoto},
  pdfcopyright={2024 Sho Iwamoto\textLF This document is licensed under the Creative Commons CC BY–NC 4.0 International Public License.},
  pdflicenseurl={https://creativecommons.org/licenses/by-nc/4.0/},
}

\usepackage{GP}
\usepackage{pgfornament}
\setlist{itemsep=4pt}
\def\thesection{C}

\newcommand\intdx{\displaystyle\int\!\dd x\,}
\newcommand\incdx[2]{\displaystyle\int_{#1}^{#2}\!\dd x\,}
\newcommand\INT[2][\relax]{\item$\displaystyle\int\!#2\,\dd x$\ifx\relax#1\relax\relax\relax\else{\quad#1}\fi}
\newcommand\INN[2][\relax]{\item$\displaystyle\int\!#2$\ifx\relax#1\relax\relax\relax\else{\quad#1}\fi}
\newcommand\INP[1]{\INT{\left(#1\right)}}
\newcommand\INC[3]{\item$\displaystyle\int^{#2}_{#1}#3\,\dd x$}

\newcommand\hrefFN[2]{\href{#1}{#2}\addnote{\url{#1}}}
\newcommand\starskip{\bigskip\begin{center}\pgfornament[width=7cm]{88}\end{center}\medskip}
\newcommand\fakebullet{\makebox[2.5em][r]{\textbullet\kern.5em}}
\makeatletter
\RenewDocumentCommand\new@problem{mm}{%
  \stepcounter{Problem}\item[\kern-2em\kern-\labelsep{\makebox[4.5em][r]{%
        \IfValueTF{#2}{\csname G#2\endcsname}{}{\sffamily\bfseries[\Alph{Problem}]}}}]\relax}
\makeatother

\let\origfootnote\footnote
\let\origfootnoterule\footnoterule

\begin{document}%
\title{Integral Boot Camp}
\begin{maketitle}
\let\footnote\origfootnote
\let\footnoterule\origfootnoterule

\subsection*{Prerequisite}
Highschool mathematics. Derivative Boot Camp all-pass (basic + true + extra sessions).

\subsection*{Preface}
The last Boot Camp on integrals are a bit different from the previous ones.
As you have seen (otherwise you need to go back!), derivative calculation is straightforward; you just need to apply the machinery you learned on the expressions.
Meanwhile, integration may requires extreme trick to obtain the answer, or even not have a clear-cut answer.
Therefore, Sho \emph{always} use Mathematica or \hrefFN{https://www.wolframalpha.com/}{Wolfram Alpha} for integral calculations, and you can do so.

However, in a very rare occasion, you may need to calculate easy integrals by hand.
Then, most of other people around you will solve that easily, even without Mathematica, and it will be shameful if you solely cannot solve it.
This may happen only once in a decade, but people's reputation is usually determined by such occasions.

\subsection*{Remarks}
Most of those integrals here will not be discussed during Sho's lecture (General Physics 1 and 2), so you do not have to be so serious now.
Meanwhile, some faculties in your department think that students should easily calculate them by hand.
This exercise may help you somewhat good feelings from your future boss.

\medskip

Sho never provides you with solutions because of his principle as a scientist.
\Emph{You students} need to make the solution. To this end,
\begin{miniitemize}
  \item Take derivative of the obtained answer to check if the result matches the integrand.
  \item Share your answers to other colleagues, using LINE or \hrefFN{https://docs.google.com/}{Google Docs}. Compare your answers with theirs.
  \item Ask questions to colleagues, to the TA, or to Sho. You can utilize Sho's \hrefFN{https://www2.nsysu.edu.tw/iwamoto/}{office hours}.
\end{miniitemize}

\OutputNote

\enlargethispage{-5em}

\makeatletter
\begin{tikzpicture}[remember picture,overlay]
  \begingroup
  \fontsize{9}{13}\selectfont
    \node[xshift=\@total@leftsep,yshift=25.5mm,anchor=south west,align=left,text width=\textwidth] at (current page.south west) {%
      \href{https://creativecommons.org/licenses/by-nc/4.0/}{\includegraphics[width=2.2cm]{../figs/by-nc.pdf}}\\[.4em]
      \noindent\textsf{\color{gray}%
      Visit \url{https://github.com/misho104/LecturePublic} for further information, updates, and to report issues.}\par
    };
    \node[xshift=-\@total@leftsep+26mm,yshift=32mm,align=left,text width=\textwidth,anchor=south east] at (current page.south east) {%
      \noindent\textsf{\color{gray}%
      This document is licensed under
      \href{https://creativecommons.org/licenses/by-nc/4.0/}{the Creative Commons CC--BY--NC 4.0 International Public License.}\\
      You may use this document only if you do in compliance with the license.}\par
    };
  \endgroup
\end{tikzpicture}
\makeatother
\end{maketitle}
\newpage


\subsection{Very Basic Integrals}
Integral is the reverse operation of derivative, so we begin with derivatives.
\begin{menumerate}{3}
  \item[\textbullet] $(\sin x)'=$
  \item[\textbullet] $(\cos x)'=$
  \item[\textbullet] $(\tan x)'=$
  \item[\textbullet] $(\cot x)'=-\csc^2x$
  \item[\textbullet] $(\csc x)'=$
  \item[\textbullet] $(\sec x)'=$
  \item[\textbullet] $(\ee^x)'=\ee^x$
  \item[\textbullet] $(\ln x)'=$
  \item[\textbullet] $|\ln x|'=$
  \item[\textbullet] $(\sinh x)'=$
  \item[\textbullet] $(\cosh x)'=$
  \item[\textbullet] $(\tanh x)'=$
  \item[\textbullet] $(\coth x)'=-\csch^2x$
  \item[\textbullet] $(\csch x)'=$
  \item[\textbullet] $(\sech x)'=$
\end{menumerate}
\begin{quizzes}
\Quiz[S]{Calculate the above derivatives (on $x$). Great attention is required for $|\ln x|'$.}
\end{quizzes}
Now it is easy for us to solve some integrals:
\begin{align}
  &\intdx (-\csc^2x)=\cot x+C,&
  &\intdx 4\ee^x=-4\ee^x+C,&
  &\int \frac{\dd x}{\sinh^2x}=-\coth x.
\end{align}


\begin{problems}
\Problem[A] Level 0. Here, $r$ is a real number.
\begin{menumerate}{3}
  \INT{x^2}
  \INT{\sin x}
  \INT{\sqrt x}
  \INN{\frac{\dd x}{x^2}}
  \INT[$(r\ge 0)$]{x^r}
  \INN[$(r\ge 2)$]{\frac{\dd x}{x^r}}
  \INN{\frac{\dd x}{\cos^2x}}
  \INN{\frac{\dd x}{\cosh^2x}}
  \INT{}
\end{menumerate}
\end{problems}

Simple replacements of variables would be easy for you.
\[
\intdx\frac{x^3}{\sqrt{x^2-1}}= \int\!\frac{2t\dd t}{2x}\,\frac{x^3}{t}=\int\!\dd t\,(t^2+1)=\frac{t^3}3+t+C=
\frac{(x^2-1)^{3/2}}3+\sqrt{x^2-1}+C,
\]
where $t=\sqrt{x^2-1}$, $t^2=x^2-1$, and thus $2x\dd x=2t\dd t$. Instead, we can use $u=x^2-1$.
\[
\intdx\frac{x^3}{\sqrt{x^2-1}}= \int\!\frac{\dd u}{2x}\,\frac{x(u+1)}{\sqrt{u}}
=\int\!\dd u\,\frac{u^{1/2}+u^{-1/2}}2
=\frac{u^{3/2}}{3}+u^{1/2}+C=\cdots
\]
and the same result is obtained, where $\dd u=2x\dd x$.

\begin{problems}
\Problem[A] Level 0.5. Try a few ideas of replacement. Some of them will work.
\begin{menumerate}{3}
  \INT{\frac{x^3}{\sqrt{x^2+1}}}
  \INT{\frac{x}{\sqrt{x^2+1}}}
  \INT{\sin(2x+5)}
  \INT{\ee^{2x+5}}
  \INT{\sinh4x}
  \INT{x\ee^{x^2}}
\end{menumerate}
\end{problems}

\newpage
\subsection{Must-Know Integrals (1)}
You may often need these derivatives when you calculate integrals.
\begin{menumerate}{2}
  \item[\textbullet] $(\tan x)'=\tan ^2x+1$
  \item[\textbullet] $(-\cot x)'=\cot ^2x+1$  \phantom{$\dfrac{f'(x)}{f(x)}$}
  \item[\textbullet] $(x \ln x)'=\ln x+1$
  \item[\textbullet] $\Bigl[\ln|f(x)|\Bigr]'=\dfrac{f'(x)}{f(x)}$
\end{menumerate}
The last one is surprisingly useful:
\[
\intdx \cot x=\intdx \frac{\cos x}{\sin x}=\ln|\sin x|+C,\qquad
\intdx \frac{5x^4+4x^3}{x^5+x^4+1}=\ln|x^5+x^4+1|+C.
\]
\begin{quizzes}
\Quiz[S]{Prove all the above. Calculate $\intdx \tan^2x$ and $\intdx \tan x$.}
\end{quizzes}
The integrals related to logarithm is a bit tricky. It is better just to memorized
\begin{equation}
  \intdx \frac{1}{x}=\ln|x|+C\qquad\Bigl(\ln (x)+C \text{~is insufficient/incorrect}\Bigr).
\end{equation}
\Remark{ Consider the function $f(x)=1/x$ and its integral $\displaystyle F(x)=\int\!\frac{\dd x}{x}$.
As $f(x)$ is defined for $x\neq 0$, i.e., $x>0$ and $x<0$, $F(x)$ should be obtained for both regions.
Hence $F(x)=\ln x+C_1$, defined only for $x>0$, is insufficient. We need to consider $F(x)=\ln(-x)+C_2$, defined for $x<0$ and satisfies $F'(x)=(-1)/(-x)=f(x)$. Comgining both functions, we should write $F(x)=\ln|x|+C$.}

Another technique you may often use is \Emph*{partial fraction decomposition}:
\[\frac {1}{x(x+1)}=\frac {1}{x}-\frac {1}{x+1},\qquad
\frac{1-2 x}{(2 x+1)^2}=\frac{2}{(2 x+1)^2}-\frac{1}{2 x+1},
\]
where you need to transform the left-hand side to right-hand side, i.e., decompose a fraction to two or more fractions whose numerators do not contain $x$. Then
\begin{align*}
& \int\!\frac{\dd x}{x(x+1)}=\ln|x|-\ln|x+1|+C=\ln\left|\frac{x}{x+1}\right|+C,\\
& \int\!\frac{1-2 x}{(2 x+1)^2}\,\dd x=-\frac{1}{2 x+1}-\frac{1}{2} \ln|2 x+1|+C
\end{align*}

\begin{quizzes}
\Quiz[S]{Check the above equations. Practice for $\dfrac{x+1}{(x+2) (x+3)}$ and $\dfrac{x}{(2x+3)(x+3)}$.}
\Quiz[A]{Check the integration by taking the derivative of the result.}
\end{quizzes}
\Remark{It is worth memorizing the formula $\displaystyle\dfrac1{AB}=\frac{1}{A-B}\left(\dfrac1B-\dfrac1A\right)$.}

\begin{problems}
  \Problem[S] Level 1.
  \begin{menumerate}{3}
    \INT{\ln x}
    \INT{\tan x}
    \INT{\cot 7x}
    \INT{\ln 7x}
    \INT{\frac{1}{(x+2)(x+3)}}
    \INT{\frac{2x+5}{(x+2)(x+3)}}
    \INT{\frac{1/x}{\ln x}}
    \INT{\frac{2}{x^2-1}}
    \INT{\frac{2x}{x^2-1}}
 \end{menumerate}
  \Problem[S] Calculate $\intdx x^a$, where $a$ is a (any) real number.
  \Problem[A] Level 1.5. \Hint[Note:~]{These are the minimal requirement for Sho's General Physics 1 and 2.}
  \begin{menumerate}{2}
    \INT{(x+1)^4}
    \INT{\sin(2x+1)}
    \INT{\tan(2x+1)}
    \INT{\frac1{\sqrt{1-x}}}
    \INT{\frac1{(1-x)\sqrt{1-x}}}
    \INT{\sqrt{1-x}}
    \INT{\left(\sqrt x+1\right)\left(\sqrt x+2\right)}
    \INT{\ee^{-x}}
  \end{menumerate}
  \Problem[B] Level 2.
  \begin{menumerate}{3}
    \INT{\frac{1}{x^2 - 4x + 3}}
    \INT{\cos^3x}
    \INT{2x\ee^{x^2}}
    \INT{\frac{1}{\cos^2(2x-1)}}
    \INT{\ee^{\sin x}\cos x}
    \INT{\ee^{\ee^x+x}}
  \end{menumerate}
  \end{problems}

\newpage
\subsection{Must-Know Integrals (2)}
Definite integrals are just an extension of indefinite integrals, but there are two important properties that you need to understand \emph{graphically} without math manipulation.
\[
\incdx st f(x)=-\incdx ts f(x),\qquad
\incdx st f(x)=\incdx{s+\Delta}{t+\Delta} f(x-\Delta),\\
\]
The following manipulation looks very tricky but is common, so you should be prepared.
\[
\incdx st f(x)
=\int_{s}^{t}\dd(-z)\,f(-z)
=\int_{-s}^{-t}(-\dd z)\,f(-z)
=-\int_{-s}^{-t}\dd z\,f(-z)
=-\int_{-s}^{-t}\dd x\,f(-x),
\]
where the last equality is just a replacement of variables.

Combining these facts,
\begin{theorem}{Odd and Even functions}{thm:odd}\vspace{-1em}
  \begin{align}
    \int_{-A}^{A}f(x)\,\dd x&=2\int_{0}^{A}f(x)\,\dd x
    &&\text{if $f(x)=f(-x)$\qquad  (i.e., $f$ is an \Emph*{even function}),}\\
    \int_{-A}^{A}f(x)\,\dd x&=0
    &&\text{if $f(x)+f(-x)=0$\qquad  (i.e., $f$ is an \Emph*{odd function}).}
  \end{align}
\end{theorem}
\begin{quizzes}
  \Quiz[B]{Prove.}
\end{quizzes}

One more thing that you need to memorize is the \Emph*{Gaussian integral}:
\begin{theorem}{Gaussian integral}{thm:gauss}\vspace{-1em}
  \begin{align}
  &\int_{-\infty}^{+\infty}\ee^{-x^2}\,\dd x=\sqrt{\pi},\\\notag
    &\text{and thus, for $k>0$, }\quad\int_{-\infty}^{+\infty}\ee^{-kx^2}\,\dd x=\sqrt{\frac{\pi}{k}},\quad
    \int_{-\infty}^{+\infty}\exp\left(-\frac{x^2}{k}\right)\,\dd x=\sqrt{k\pi}.
  \end{align}
\end{theorem}
The proof is too technical and skipped, but scientists use this integral at least 100 times per year.
\begin{problems}
  \Problem[A] Level 1.5.
  \begin{menumerate}{3}
\INC{-1}{1}{x^3}
\INC{-2}{2}{x^4}
\INC{-\pi}{\pi}{\sin x}
\INC{-\pi/6}{\pi/6}{\cos x}
\INC{-2}{2}{\tanh x}
\INC{0}{2}{\sin(x-1)}
\INC{0}{2}{\cosh(x-1)}
\INC{-\infty}{\infty}{\ee^{-2x^2}}
\INC{0}{\infty}{\ee^{-x^2}}
\end{menumerate}\end{problems}
\newpage
\subsection{Advanced Techniques}
Sho thinks that, for more complicated integrals, we should use computers or online resources. If, however, you want to solve them by hand, you may practice the following technique:
\begin{theorem}{Integration by parts}{thm:parts}\vspace{-.7em}
  \begin{align}
    &\int f(x)g'(x)\,\dd x=f(x)g(x)-\int f'(x)g(x)\,\dd x,
\end{align}
\end{theorem}
In particular, $\intdx f(x)=f(x)x-\intdx xf'(x)$ is useful if $xf'(x)$ becomes simpler than $f(x)$.

Furthermore, the following hints could be useful.
\begin{itemize}
  \item If $\dfrac1{x^2+a^2}$ appears, try $x=a\tan\theta$.
  \item If $\dfrac1{x^2-a^2}$ appears, try $x=a\sinh\theta$.
  \item If $\sqrt{x^2-a^2}$ appears, try $x=a\sin\theta$.
  \item If $\sqrt{x^2+a^2}$ appears, try $x=a\sinh\theta$.
  \item If (only) trigonometric functions ($\sin x$ etc.) appear and you are lost, try $x=2\arctan t$, i.e., $\tan(x/2)=t$.
  \item King property: $\incdx ab f(x)=\incdx abf(a+b-x)$.
\end{itemize}

\vspace{3em}

There are many well-prepared problems on integrals on the internet. If you are interested in, try
\begin{itemize}
  \item \href{https://sites.google.com/view/chhs-math/%E4%B8%8D%E5%AE%9A%E7%A7%AF%E5%88%86/%E7%A7%AF%E5%88%86%E7%BB%83%E4%B9%A0/%E7%A7%AF%E5%88%86100}{\JA{積分100題}} on CHHS mathematics
\end{itemize}
which seems to contain many nice problems.
\end{document}
