%!TEX program=lualatex

\def\LectureName/{General Physics I}
%\def\LectureNumber/{}
%\def\LectureDate/{}
\PassOptionsToPackage{fleqn}{amsmath}
\PassOptionsToPackage{hyperfootnotes=false}{hyperref}
\documentclass[11pt,pdfa,lastpage]{MishoNote}
\title{General Physics I: Derivative Boot Camp Extra Session}
\author{Sho Iwamoto}
\hypersetup{
  pdflang={en-US},
  pdfauthortitle={Assistant Professor, National Sun Yat-sen University},
  pdfsubject={An extra session of Derivative Boot Camp, which is a preview of General Physics 1 lectures.},
  pdfcontactemail={iwamoto@g-mail.nsysu.edu.tw},
  pdfcontacturl={https://www2.nsysu.edu.tw/iwamoto/},
  pdfcaptionwriter={Sho Iwamoto},
  pdfcopyright={2024 Sho Iwamoto\textLF This document is licensed under the Creative Commons CC BY–NC 4.0 International Public License.},
  pdflicenseurl={https://creativecommons.org/licenses/by-nc/4.0/},
}

\usepackage{GP}
\usepackage{pgfornament}
\setlist{itemsep=0.3\baselineskip}
\makeatletter
\def\thesection{A$\mathbf{'}$}
\def\theenumi{\@Alph\c@enumi$\mathbf{'}$}
\makeatother

\newcommand\INT[2][\relax]{\item$\displaystyle\int#2\,\dd x$\ifx\relax#1\relax\relax\relax\else{\quad#1}\fi}
\newcommand\INP[1]{\INT{\left(#1\right)}}
\newcommand\INC[3]{\item$\displaystyle\int^{#2}_{#1}#3\,\dd x$}
\NewDocumentCommand\DIFF{O{x}om}{\item$\displaystyle\gdiff[\dd][#2]{}{#1}#3$}

\newcommand\hrefFN[2]{\href{#1}{#2}\footnote{\url{#1}}}
\newcommand\starskip{\bigskip\begin{center}\pgfornament[width=7cm]{88}\end{center}\medskip}

\let\origfootnote\footnote
\let\origfootnoterule\footnoterule

\begin{document}%
\title{Derivative Boot Camp Extra Session}\begin{maketitle}
\let\footnote\origfootnote
\let\footnoterule\origfootnoterule

\subsection*{Preface}
Welcome to your first year of university!
You, who completed the true form of the Derivative Boot Camp, are now ready to take on more challenging tasks in your engineering studies.
For those who still have the energy and motivation to go deeper, Sho has prepared this extra session of the Boot Camp.

This session aims to complete your knowledge and skills of derivatives to the university standard.
By the end of the exercises, you will be able to calculate ``simple'' derivatives such as
\[
\diff{}{x}\frac{x^2+\tan(\ln x)}{\sinh(2x^2+1)}.
\]


Tasks are classified to $\bigstar$ (\Emph{minimal}), $^{***}$ (basic), $^{**}$ (intermediate), and $^{*}$ (for motivated students).
You don't have to solve all of them, but try to solve as many as you can.
Good Luck!

\enlargethispage{-5em}

\makeatletter
\begin{tikzpicture}[remember picture,overlay]
  \begingroup
  \fontsize{9}{13}\selectfont
    \node[xshift=\@total@leftsep,yshift=25.5mm,anchor=south west,align=left,text width=\textwidth] at (current page.south west) {%
      \href{https://creativecommons.org/licenses/by-nc/4.0/}{\includegraphics[width=2.2cm]{../figs/by-nc.pdf}}\\[.4em]
      \noindent\textsf{\color{gray}%
      Visit \url{https://github.com/misho104/LecturePublic} for further information, updates, and to report issues.}\par
    };
    \node[xshift=-\@total@leftsep+26mm,yshift=32mm,align=left,text width=\textwidth,anchor=south east] at (current page.south east) {%
      \noindent\textsf{\color{gray}%
      This document is licensed under
      \href{https://creativecommons.org/licenses/by-nc/4.0/}{the Creative Commons CC--BY--NC 4.0 International Public License.}\\
      You may use this document only if you do in compliance with the license.}\par
    };
  \endgroup
\end{tikzpicture}
\makeatother

\end{maketitle}
\newpage




\newpage
\begin{enumerate}
\itemB \textsf{(a bit tough problems)}
\begin{menumerate}{3}
  \DIFF{(5x^2 - 2x + 1)^{-3}}
  \DIFF{(x^4+1)^{-3/5}}
  \DIFF{\sin^2[(x^2+2x)^2]}
  \DIFF{\tan \left(x+\sqrt{2x}\right)}
  \DIFF{\frac{\sin(x^2 + 1)}{\sin(x - 1)}}

  \DIFF{\frac{1}{\sqrt{x^2 + 4}}}
  \DIFF{x^{-1/2}\cos x^2}
  \DIFF{x^2\sin x\tan2x}
  \DIFF{x(x - 1)^{-3/4}}
  \DIFF{\frac{2\sin x}{(x - 1)^{3/4}}}

  \DIFF{\frac {x^3\tan x}{\cos x}}
  \DIFF{\frac {x\sin x}{\cos2x}}
  \DIFF{\frac {x^2\sin^2x}{\cos x^2}}
  \DIFF{\frac{x^3 + 1}{\sqrt{x - 1}}}
  \DIFF{\tan^2\left(x \sqrt{x}\right)}
\end{menumerate}
\end{enumerate}

\subsection{Workout 3: Harder Practice for motivated students}
If you have not satisfied with the above problems, you can try the following ones.
First, you need to understand the following procedure.
\begin{itemize}
  \item Consider $\sin\sqrt{x^2+1}$ and we let $u=x^2+1$.
  \[
    \diff{}{x}\sqrt{x^2+1}=\diff{}{x}\sqrt u=\diff{\sqrt u}{u}\diff ux=\frac{1}{2\sqrt u}\cdot2x=\frac{x}{\sqrt{x^2+1}}.
  \]
  Therefore, with letting $v=\sqrt{x^2+1}$,
  \[
  \diff{}{x}\sin\sqrt{x^2+1}=\diff{\sin v}{x}=\diff{\sin v}{v}\diff vx=\cos v\cdot\frac{x}{\sqrt{x^2+1}}=\frac{x\cos\sqrt{x^2+1}}{\sqrt{x^2+1}}.
  \]
  \item You can calculate it at once, where we let $u(x)=x^2+1$ and $v=v(u)=\sqrt{u}$:
  \[
   \diff{}{x}\sin\sqrt{x^2+1}=\diff{\sin v}{v}\diff{v}{u}\diff{u}{x}=(\cos v)\frac{1}{2\sqrt{u}}(2x)=\frac{x\cos\sqrt{x^2+1}}{\sqrt{x^2+1}}.
   \]
\end{itemize}
Now you are ready to calculate very complicated functions.
\starskip

\begin{enumerate}
 \itemC[*] \textsf{(The final problems)}
   \begin{menumerate}{2}
     \DIFF{\left[\sin\left(x+\sqrt{x}\right)\right]^2}
     \DIFF{\frac{1}{\sqrt{\cos(2x^2 + 1)}}}
     \DIFF{\frac{3x + 1}{\sin(3x+1)}}
     \DIFF{\cos^2(x^2 - 1)}
     \DIFF{\left(x+ \tan x\right)^{4/3}}
     \DIFF{\left(\frac{\sin x}{3x^2 + 2x} \right)^4}
     \DIFF{\left(x - \sqrt{x^2 + 1} \right)^2}
     \DIFF{\tan\sqrt{x+x^{-1}}}
     \DIFF{\cos^5(x^2 + x)}
     \DIFF{\cos(\sin(\cos x))}
   \end{menumerate}
 \end{enumerate}



\newpage
\subsection{Exponential Functions}
We want to consider the functions
\[f(x)=a^x\qquad\text{with}\quad a>0.\]
They are called exponential functions.
Let's begin with the basic part, and then try to calculate its derivative.
(Recall that any derivatives can be calculated by the definition of the derivative!)

\starskip

\begin{enumerate}
  \itemA Let $f(x)=2^x$ and $g(x)=0.5^x$.  Calculate the following values. You can use a calculator \Emph{only for} (5) and (6).
  \begin{menumerate}{2}
    \item $f(0)$, $f(1)$, $f(2)$, $f(10)$.
    \item $g(0)$, $g(1)$, $g(2)$.
    \item $f(-1)$, $f(-3)$, $g(-1)$, $g(-2)$.
    \item $f(1/2)$, $f(-1/2)$, $g(1/2)$, $g(-1/2)$.
    \item $f(1.585)$, $g(1.585)$, $f(1.585\times 2)$.
    \item $f(0.01)$, $f(-0.01)$, $g(0.01)$, $g(-0.01)$.
  \end{menumerate}
  \itemA Simplify the following expressions, where $x$ and $y$ are real numbers, $a>0$, and $b>0$.
  \begin{menumerate}{3}
    \item $4^{2.5}\times 4^{3.5}$
    \item $3^{x}\times 3^{y}$
    \item $\dfrac{7^{x+5}}{7^{2}}$
    \item $1/a^x$
    \item $a^x/a^y$
    \item $\dfrac{(ab)^3}{b^2}$
    \item $a^5\times a^5\times a^5$
    \item $(a^x)^2$
    \item $\dfrac{(a^x)^y}{a^y}$
  \end{menumerate}

 \itemB Check the following equations based on the definition od differentials.
\begin{equation*}
  \diff{}{x}2^x = 2^x\times\lim_{h\to 0}\frac{2^h-1}{h},\qquad
  \diff{}{x}3^x = 3^x\times\lim_{h\to 0}\frac{3^h-1}{h}.
\end{equation*}
\itemB Using online resources, evaluate the above limits. You will find
\[
  \lim_{h\to 0}\frac{2^h-1}{h}\simeq0.7, \qquad \lim_{h\to 0}\frac{3^h-1}{h}\simeq1.1.
\]
\itemC The next goal is to find a number $\xi$ that satisfies $\displaystyle\lim_{h\to 0}\frac{\xi^h-1}{h}=1$.
Let us consider a function \[ \lambda(x)=\lim_{h\to 0}\frac{x^h-1}{h};\]
then we need to find $\xi$ satisfying $f(\xi)=1$.
Now, observing the result of the previous problem, we can guess $2<a<3$ because we have found $\lambda(2)\simeq 0.7$ and $\lambda(3)\simeq1.1$.

Using online resources, evaluate $\lambda(2.7)$, $\lambda(2.71828)$, and $\lambda(2.8)$.
\end{enumerate}

\newpage
\subsection{Napier's Constant}
The number,
\begin{equation}
  \lim_{n\to\infty}\left(1+\frac{1}n\right)^n=\lim_{n\to\infty}\sum_{k=0}^n\frac{1}{k!}=2.7182818284590452\cdots\label{eq:napier}
\end{equation}
is called Napier's constant, and is denoted by $\ee$. It is the number which satisfies $\lambda(\ee)=1$ in the previous page:
\begin{equation}
  \lim_{h\to 0}\frac{\ee^h-1}{h}= \lim_{h\to 0}\frac{(2.718\cdots)^h-1}{h}=1.
\end{equation}

Logarithmic function with base $\ee$ is called the natural logarithm:
\[
  \ln x \coloneq \log_{\ee} x,
\]
which we will review in the next page. Here, you need to recall  the relation
\[
  a^{\log_a b}= b,\qquad\text{thus}\quad \ee^{\ln x}=x.
\]
\Remark{Students tend to understand this as a complicated theorem, but it isn't. Rather, this is a \emph{definition} of $\log_a b$.\\
Recall that the number $x$ satisfying $a^x=b$ is defined as $x=\log_a b$.
Then, if we put it onto the shoulder of $a$, it should give $a^x=a^{\log_a b}= b$.}

Now we can prove this theorem:

\begin{theorem}{}{}
  \begin{equation}
      \diff{}x{\ee^x}=\ee^x,\qquad
      \diff{}x{a^x}=\ee^x\ln a\qquad(a>0).
  \end{equation}
\end{theorem}
Note that this is valid even for $a=1$. (What is $\ln 1$?)

\starskip


\begin{enumerate}[resume]
  \itemA Prove the first part of the above theorem. You should start from the definition of derivatives and find
  \begin{equation}
    \diff{}x{\ee^x}=\lim_{h\to 0}\frac{\ee^h-1}{h}\times\ee^x=\ee^x.
  \end{equation}
  \itemA Prove the latter part of the above theorem, using $a^x=\ee^{x\ln a}$ and the formula of derivatives for composite functions.
\end{enumerate}

\newpage
\subsection{Logarithm}
Derivatives of logarithmic functions are also important.

\begin{theorem}{}{}
  \begin{equation}
      \diff{}x{\ln x}=\frac{1}{x},\qquad
      \diff{}x{\log_a x}=\frac{1}{x\ln a}\qquad(a>0,a\neq 1).
  \end{equation}
\end{theorem}
The proof of the first part, $(\ln x)'=1/x$, is a bit tricky. We will skip the proof here.
If we accept it, we can easily prove the latter part if you recall the formula
 \[\log_a b = \frac{\log_c b}{\log_c a}
 \qquad(a>0,~~b>0,~~c>0;\quad a\neq 1,~~c\neq 1).
 \]

\starskip

\begin{enumerate}[resume]
  \itemA Using $(\ln x)'=1/x$, prove the latter part of the above theorem.\\
   \Hint{Replace $c$ in the above equation by $\ee$.}
\end{enumerate}

\bigskip
\begin{enumerate}[resume]
  \itemB Calculate the following values. You will need a calculator for, but only for, (5) and (6).
\begin{menumerate}{2}
  \item $\log_{10}100$, $\log_{10}0.01$, $\log_{10}0.001$.
  \item $\log_2 4$, $\log_{3}27$, $\log_5\sqrt{5}$, $\log_{9}3$.
  \item $\ln e$, $\ln \sqrt{\\e}$, $\ln \dfrac{1}{e}$.
  \item $\log_{0.5}0.5$, $\log_{0.5}0.25$, $\log_{1/3}3$.
  \item $\ln 1$, $\ln 1.001$, $\ln 1.000001$
  \item $\ln 0.1$, $\ln 0.001$, $\ln 0.000001$
\end{menumerate}
\itemA Simplify the following expressions. Here, $x$ and $y$ are real numbers and  $a>0$, $a\neq 1$.
\begin{menumerate}{3}
  \item $\dfrac{\log_{2}8}{\log_{2}4}$
  \item $\log_a{10}+\log_a{5}$
  \item $\log_a{10}-\log_a{5}$
  \item $\ln x-\ln y$
  \item $\log_a x^{10}$
  \item $\log_a \dfrac{1}{x^{10}}$
\end{menumerate}
\itemB Simplify the following expressions. Here, $a>0$, $a\neq 1$, and $b>0$, $b\neq 1$.
\begin{menumerate}{3}
  \item $\dfrac{\log_{a}8}{\log_{a}4}$
  \item $\dfrac{\log_{a}5}{\log_{b}5}$
  \item $\ln 2^{20}+\ln 3^{10}$
  \item $\dfrac{\log_{a}5}{\log_{1/a}5}$
  \item $\ln 8-2\ln 4+\ln\dfrac{1}{20}$
  \item $(\log_a b)(\log_b a)^2$
\end{menumerate}
\itemB Express the following expressions using natural logarithm ($\ln x$). Here, $a>0$ and $b>0$.
\begin{menumerate}{3}
  \item $\log_2 3$
  \item $\log_2a^b$
  \item $\log_7(3\ee^2/49)$
\end{menumerate}
\end{enumerate}

\newpage
\subsection{Hints for Motivated Students}
These are more mathematical, more advanced, and less important. You may well skip them.

\starskip

\begin{enumerate}[resume]
  \item For $f(x)=a^x$, check the following facts.
  \begin{enumerate}
    \item If $a>0$, $f(x)$ is defined for any $x\in\RR$. Also, $f(x)$ is always positive and continuous.
    \item If $a<0$, $f(x)$ is defined only for $x\in\ZZ$. It is difficult to define, e.g., $(-2)^{0.49}$.
    \item If $a=0$, $f(x)$ is defined only for $x>0$. The value is always zero.\addnote{%
After you learn complex analysis, we can define $a^x$ for negative $a$. Namely,
\[
a^x=(-1)^x|a|^x=(\ee^{\ii\pi})^x|a|^x=|a|^x\ee^{\ii\pi x},
\]
which is not real but complex for $x\not\in\ZZ$.
On the other hand, $0^x$ is well-defined only for $x>0$. In fact, there is no good way to define $0^0$. (We sometimes use $0^0=1$, but this is mathematically incorrect.)
}
  \end{enumerate}
\end{enumerate}
  \OutputNote
\begin{enumerate}[resume]
  \item Try to derive $(\ln x)'=1/x$ from the definition. Notice that $x>0$.
  \begin{enumerate}
    \item Prove the following equation: \Hint{$\tilde h=h/x$}
     \[\diff{}x{\ln x}=\frac{1}{x}\lim_{\tilde h\to 0}\frac{\ln(1+\tilde h)}{\tilde h}\]
    \item From the equation~\eqref{eq:napier}, prove that
    \[\ee=\lim_{h\to+0}(1+h)^{1/h}.\]
  \end{enumerate}
  Now, we want to use $\ee=\lim_{h\to0}(1+h)^{1/h}$. This is slightly different from the above expression and, mathematically, an independent proof is required.
  However, to simplify discussion, we accept this formula.
  \begin{enumerate}[resume]
    \item Using $\ee=\lim_{h\to0}(1+h)^{1/h}$, show that
    \[
      \lim_{h\to 0}\frac{\ln(1+h)}{h}=\ln\ee.
    \]
    \item Combine the above discussion to show $(\ln x)'=1/x$.
  \end{enumerate}

\end{enumerate}



\end{document}
