%!TEX program=lualatex

\def\LectureName/{General Physics I}
%\def\LectureNumber/{}
%\def\LectureDate/{}
\PassOptionsToPackage{fleqn}{amsmath}
\PassOptionsToPackage{hyperfootnotes=false}{hyperref}
\documentclass[11pt,pdfa,lastpage]{MishoNote}
\title{General Physics I: Derivative Boot Camp (Basic)}
\author{Sho Iwamoto}
\hypersetup{
  pdflang={en-US},
  pdfauthortitle={Assistant Professor, National Sun Yat-sen University},
  pdfsubject={Derivative Boot Camp (Basic) as an introduction (and an initiation) to the university. A part of General Physics 1 lecture.},
  pdfcontactemail={iwamoto@g-mail.nsysu.edu.tw},
  pdfcontacturl={https://www2.nsysu.edu.tw/iwamoto/},
  pdfcaptionwriter={Sho Iwamoto},
  pdfcopyright={2024 Sho Iwamoto\textLF This document is licensed under the Creative Commons CC BY–NC 4.0 International Public License.},
  pdflicenseurl={https://creativecommons.org/licenses/by-nc/4.0/},
}

\usepackage{GP}
\usepackage{pgfornament}
\setlist{itemsep=4pt}
\def\thesection{A}

\newcommand\INT[2][\relax]{\item$\displaystyle\int#2\,\dd x$\ifx\relax#1\relax\relax\relax\else{\quad#1}\fi}
\newcommand\INP[1]{\INT{\left(#1\right)}}
\newcommand\INC[3]{\item$\displaystyle\int^{#2}_{#1}#3\,\dd x$}
\NewDocumentCommand\DIFF{O{x}om}{\item$\displaystyle\gdiff[\dd][#2]{}{#1}#3$}

\newcommand\hrefFN[2]{\href{#1}{#2}\footnote{\url{#1}}}
\newcommand\starskip{\bigskip\begin{center}\pgfornament[width=7cm]{88}\end{center}\medskip}
\makeatletter
\RenewDocumentCommand\new@problem{mm}{%
  \stepcounter{Problem}\item[\kern-2em\kern-\labelsep{\makebox[4.5em][r]{%
        \IfValueTF{#2}{\csname G#2\endcsname}{}{\sffamily\bfseries[\Alph{Problem}]}}}]\relax}
\makeatother


\let\origfootnote\footnote
\let\origfootnoterule\footnoterule

\begin{document}%
\title{Derivative Boot Camp (Basic)}\begin{maketitle}
\let\footnote\origfootnote
\let\footnoterule\origfootnoterule

\subsection*{Preface}
Welcome to your first year of university!
As a university student in engineering, \Emph{you must be able to calculate derivatives of ``simple'' functions} such as
\[
\diff{}{x}\frac{x^2+\tan x}{\sin(2x+1)}.
\]
This Boot Camp is designed to help you prepare for your first year, which is unexpectedly tough for most of you!
Take your time, go through each problem carefully, and don't hesitate to ask for help!

To motivate you, we will have a \Emph{mini test} at the beginning of the second lecture; the problems will be from this Boot Camp.
I hope this preparation will make your university life easier, more enjoyable, and more satisfactory. Good luck!

\subsection*{Remarks}
Sho never provides you with solutions. \Emph{You students} need to make the solution. To this end,
\begin{miniitemize}
  \item Use online resources such as \hrefFN{https://www.wolframalpha.com/}{Wolfram Alpha}.
  \item Share your answers to other colleagues, using LINE or \hrefFN{https://docs.google.com/}{Google Docs}. Compare your answers with theirs.
  \item Ask questions to colleagues, to the TA, or to Sho. You can utilize Sho's \hrefFN{https://www2.nsysu.edu.tw/iwamoto/}{office hours}.
\end{miniitemize}


\enlargethispage{-5em}

\makeatletter
\begin{tikzpicture}[remember picture,overlay]
  \begingroup
  \fontsize{9}{13}\selectfont
    \node[xshift=\@total@leftsep,yshift=25.5mm,anchor=south west,align=left,text width=\textwidth] at (current page.south west) {%
      \href{https://creativecommons.org/licenses/by-nc/4.0/}{\includegraphics[width=2.2cm]{../figs/by-nc.pdf}}\\[.4em]
      \noindent\textsf{\color{gray}%
      Visit \url{https://github.com/misho104/LecturePublic} for further information, updates, and to report issues.}\par
    };
    \node[xshift=-\@total@leftsep+26mm,yshift=32mm,align=left,text width=\textwidth,anchor=south east] at (current page.south east) {%
      \noindent\textsf{\color{gray}%
      This document is licensed under
      \href{https://creativecommons.org/licenses/by-nc/4.0/}{the Creative Commons CC--BY--NC 4.0 International Public License.}\\
      You may use this document only if you do in compliance with the license.}\par
    };
  \endgroup
\end{tikzpicture}
\makeatother

\end{maketitle}
\newpage
\subsection{The first step: High-school review}
We begin with a review of high-school mathematics, but with taking care of a typical pitfall.
Namely, some students are confused by the notation of derivatives.

Consider a function $f(x)$. The derivative of $f(x)$ is written by $f'(x)$, which is (usually) a different function from $f(x)$.
We express this function $f'(x)$ by
\[ f'(x) = \diff{f}{x}(x) = \diff{f(x)}{x} = \diff{}{x}f(x); \]
these three expressions are equivalent.
Don't be confused by the tricky notation!

Sho recommends you to identify $\displaystyle \diff{f}{x} \stackrel!= f'$. Then, you will easily see that
\begin{itemize}
  \item $\displaystyle f'(x)=\diff{f}{x}(x)=\diff{f(x)}{x}=\diff{}{x}f(x)$ is a function,
  \item but we often omit the ``$(x)$'' part and write $\displaystyle f'$, $\displaystyle\diff fx$, or $\displaystyle\diff{}xf$.
  \item $\displaystyle f'(a)=\diff{f}{x}(a)=\diff{}xf(a)$ means the value of $f'(x)$ at $x=a$.
\end{itemize}

\begin{problems}
  \Problem[S] Calculate the following derivatives.
  \begin{menumerate}{3}
    \DIFF{x^2}
    \DIFF{\left(3x^8+2x^5-1\right)}
    \DIFF{\left(x+1\right)^3}
  \end{menumerate}
  \Problem[S] Let $f(x)=x^4$. Calculate $f'(x)$, $f'(c)$, $f'(0)$ and $f'(1)$.
\end{problems}

\subsection{Trigonometric functions}
For arguments of \Emph{trigonometric functions} $\sin x$, $\cos x$, etc., we usually use \Emph{radians} instead of degrees. The degree $180^\circ$ is equal to $\pi$ radian, so
\begin{equation*}
  180^\circ = \pi\unit{rad},\qquad\text{i.e.,}\qquad 1\unit{rad}=\frac{180^\circ}\pi\approx57.30^\circ.
\end{equation*}
Furthermore, the unit ``rad'' is often omitted. Namely,
\begin{equation*}
  \cos\frac{\pi}{3} = \cos\left(\frac\pi3\unit{rad}\right) = \cos60^\circ = \frac12,
\end{equation*}
etc. You need to get accustomed to this convention.

\Remark{Many students are confused by the notation. Notice $(\sin x^2)\neq(\sin x)^2$, i.e.,
\begin{equation*}
  \sin^2x~~=~~(\sin x)^2~~\neq~~\sin x^2~~=~~\sin(x^2).
\end{equation*}
}

\begin{problems}
  \Problem[S] Calculate the following values.
  \begin{menumerate}{4}
    \item $\displaystyle\sin\frac\pi6$
    \item $\displaystyle\sin\frac\pi4$
    \item $\displaystyle\sin\frac\pi3$
    \item $\displaystyle\cos\frac{2\pi}3$
    \item $\displaystyle\tan\frac\pi6$
    \item $\displaystyle\tan\frac{\pi}3$
    \item $\tan$
    \item $\cos2\pi$
  \end{menumerate}
\end{problems}
\subsection{Get into the University}
Recall that derivatives are \Emph{defined by}
\begin{equation*}
\diff{f}{x}(x_0)=\lim_{x\to x_0}\frac{f(x)-f(x_0)}{x-x_0}
\end{equation*}
or, equivalently,
\begin{equation}
\diff{}{x}f(x)=\lim_{h\to 0}\frac{f(x+h)-f(x)}{h}.\label{eq:diff}
\end{equation}
Any derivatives can be calculated based on this definition.


\begin{problems}
\Problem[S]\relax
  \Emph{Based on the definition}~\eqref{eq:diff}, calculate the following derivatives.
  \begin{menumerate}{3}
    \DIFF{x^2}
    \DIFF{\left(2x^3\right)}
    \DIFF{\left(4x\right)}
  \end{menumerate}
 \Problem[S] We know that a function $f(x)$ satisfies an equation $f'(x)=4x$. Can you find what $f(x)$ is? If you can, find more than one.
\end{problems}

\subsection{The formulae you need to memorize}
Now it's time for more complicated functions. First, you need to memorize the following formulae.
I mean, \emph{you need to do exercise until you've memorized them and can use them without any hesitation}.

\begin{theorem}{}{}
  Derivatives of trigonometric functions are given by
  \begin{align}
   \diff{}{x}\sin x &= \cos x,&
   \diff{}{x}\cos x &= -\sin x,&
   \diff{}{x}\tan x &= \frac1{\cos^2x}.
  \end{align}
\end{theorem}
\begin{theorem}{}{}
  Derivatives of the reciprocal, product, and quotient of a function(s) are given by
  \begin{align}
   &\label{eq:reci} \diff{}{x}\frac{1}{f(x)}=-\frac{f'(x)}{[f(x)]^2},\\[0.5em]
   &\label{eq:prod} \diff{}{x}\Bigl[f(x)g(x)\Bigr]=f'(x)g(x)+f(x)g'(x),\\[0.5em]
   &\label{eq:quot} \diff{}{x}\frac{g(x)}{f(x)}=\frac{f(x)g'(x)-f'(x)g(x)}{[f(x)]^2},
   \end{align}
   where  $f(x)$ and $g(x)$ are any differentiable functions.
\end{theorem}

\begin{problems}
  \Problem[S]
  Use online resources to check that these are correct (i.e., Sho didn't make any typo).
\end{problems}
\begin{problems}
  \Problem[S] Calculate the following derivatives.
  \begin{menumerate}{3}
    \DIFF{3\sin x}
    \DIFF{\left(2\sin x+\tan x\right)}
    \DIFF{\left(1+x+\tan x\right)}
  \end{menumerate}
  \Problem[S] Calculate the following derivatives, using Eqs.~\eqref{eq:reci}--\eqref{eq:quot}.
  \begin{menumerate}{3}
    \DIFF{(x+1)(x^2 + 2)}
    \DIFF{x  \sin x}
    \DIFF{(3x^2 + 2x + 1)^2}
    \DIFF{\sin^2 x}
    \DIFF{\frac{1}{x+1}}
    \DIFF{\frac{1}{\tan x}}
    \DIFF{\frac{x^2+1}{x+1}}
    \DIFF{\frac{\sin x}{x}}
    \DIFF{\frac{\sin x}{\cos x}}
  \end{menumerate}
\end{problems}

\subsection{Workout 1: Practice!}
\begin{problems}
\Problem[S] Practice for the formulae~\eqref{eq:reci} and~\eqref{eq:prod}.
\begin{menumerate}{3}
  \DIFF{\frac{1}{\sin x}}
  \DIFF{\frac{1}{3x^2+1}}
  \DIFF{\frac{1}{x^2+2x+1}}
  \DIFF{(x-1)(\cos x + x)}
  \DIFF{\sin x\cos x}
  \DIFF{\cos x\tan x}
  \DIFF{(x^2 + 1)^2}
  \DIFF{x^5\cos x}
  \DIFF{(x^2\cdot x^3)}
\end{menumerate}
\Problem[S] Practice for the formulae~\eqref{eq:reci} and~\eqref{eq:quot}. Here, $n$ is a positive integer.
\begin{menumerate}{4}
  \DIFF{\frac{x+1}{x-1}}
  \DIFF{\frac{x+1}{x^2+1}}
  \DIFF{\frac{x^2-1}{x-1}}
  \DIFF{\frac{\sin x}{x^2}}
  \DIFF{\frac{1-\cos x}{1+\cos x}}
  \DIFF{\frac{x^3+1}{x^2+1}}
  \DIFF{\frac{1}{x^n}}
  \DIFF{x^{-3}\cos x}
\end{menumerate}
\end{problems}
\subsection{One more formula}
The formula you know well, $(x^n)'=nx^{n-1}$, can be generalized to any real number $a$.
Notice you can use this formula for $a=1/2$, $a=-1$, $a=-3/2$, or even $a=0$.
\begin{theorem}{}{}
  \begin{equation}
  \text{For \Emph{any real number}}~a,\qquad \diff{}{x}x^a = ax^{a-1}.
  \end{equation}
\end{theorem}

\begin{problems}
\Problem[S] Calculate the following derivatives based on the theorem above.
\begin{menumerate}{3}
  \DIFF{x^{64}}
  \DIFF{\sqrt x}
  \DIFF{x^{1/3}}
  \DIFF{x^{-2}}
  \DIFF{\frac{1}{x^{1/3}}}
  \DIFF{\frac{1}{x\sqrt x}}
\end{menumerate}
\end{problems}

\subsection{The last step: Composite functions}
Now, we consider \Emph{composite functions}, which are functions of functions. For example, consider $f(x)=\sin(x^2)$. Its derivative can be calculated with the next theorem.
\begin{theorem}{}{}
  Consider $f(u)$, which is a function of $u$. Assume $u=u(x)$ is a function of $x$. Then, we can consider $f(u)$ as a function of $x$, i.e., $f(x)=f(u(x))$, and
  \begin{equation}
   \diff{f}{x}=\diff{f}{u}\diff{u}{x}.
  \end{equation}
\end{theorem}
\noindent
This theorem is complicated but you need to get accustomed to. For example,
  \begin{itemize}
    \item For $f(x)=\sin(x^2)$, we set $u=x^2$ and $f(u)=\sin u$. Then, $\diff fu=\cos u$ and $\diff ux=2x$, which lead to the conclusion $\diff fx=(\cos u)\cdot2x=2x\cos x^2$.
    \item Consider $g(x)=(x^2+2x+1)^4$. We use the theorem with $f(u)=u^4$ and $u=x^2+2x+1$. Then, $\diff fu=4u^3$ and $\diff ux=2x+2$, which lead to $\diff fx=4(x^2+2x+1)^3\cdot(2x+2)$.
    \\\relax
    [Notice that this is equal to $8(x+1)^7$.]
    \item This theorem helps us a lot. For example, the derivative of the function $(x^2+1)^3$ can be easily calculated, with $u=x^2+1$, as $3u^2\cdot 2x=6x(x^2+1)^2$.
  \end{itemize}
You can calculate more complicated functions. One example is $\cos(\sin x)$. If we let $u=\sin x$,
  \[
  \diff{}{x}\cos(\sin x)=\diff{\cos{u}}{x}=\diff{\cos{u}}{u}\cdot\diff ux=-\sin u\cdot\cos x=-\sin(\sin x)\cdot\cos x.
  \]

\begin{problems}
  \Problem[S] Practice.
  \begin{menumerate}{3}
    \DIFF{(x^2+1)^3}
    \DIFF{(x^2+2x+1)^4}
    \DIFF{\sin x^4}
    \DIFF{\sin^4x}
    \DIFF{\sqrt{x+1}}
    \DIFF{\sqrt{\tan x}}
    \DIFF{\sin x^{-2}}
    \DIFF{\cos \sqrt{x}}
    \DIFF{(\sin x)^{-2}}
    \DIFF{(\cos x)^{-1/2}}
    \DIFF{\frac1{\sqrt{\tan x}}}
  \end{menumerate}
\end{problems}

You will also be asked to combine with the formulae you've learned so far. For example,
  \begin{align*}
  \diff{}{x}\frac{x}{\cos(x^2+1)}
  &= \frac{(x)'\cos(x^1+1)-x\left[\cos(x^2+1)\right]'}{\left[\cos(x^2+1)\right]^2}
  = \frac{\cos(x^2+1)-x(-\sin u)(u)'}{\cos^2(x^2+1)}
  \\&= \frac{\cos(x^2+1)+x\cdot 2x\cdot \sin (x^2+1)}{\cos^2(x^2+1)} = \frac{1+2x^2\tan(x^2+1)}{\cos(x^2+1)}
  \end{align*}
where we let $u=x^2+1$.


\subsection{Workout 2: Practice, Practice, Practice!}
Now you need to practice more to get accustomed to the calculation.
Some of the following may be a bit complicated, but you can solve them by combining the previous formulae.
If you are lost, try using online resources, ask your colleagues, or ask Sho.

\begin{problems}
  \Problem[S] Practice more.
\begin{menumerate}{3}
  \DIFF{(5x^2 - 2x + 1)}
  \DIFF{(x^4 + \sqrt{x})}
  \DIFF{(\sin x + \cos x)}
  \DIFF{(\tan x + \sqrt{2x})}
  \DIFF{\frac{x^2 + 1}{x - 1}}
  \DIFF{\left(\frac{1}{x} + x^2\right)}
  \DIFF{\sqrt{x}\cos x}
  \DIFF{x^2\sin x}
  \DIFF{\cos x^2}
  \DIFF{\cos^2 x}
  \DIFF{(x-1)^{-1/2}}
  \DIFF{\frac 1{\cos x}}
  \DIFF{\frac x{\cos x}}
  \DIFF{\frac {\cos x^2}{x^2+1}}
  \DIFF{\frac x{\sqrt{\cos x}}}
\end{menumerate}
\end{problems}

\vfill
\subsection*{Afterwords}

Have you finished all the problems? Great job! You're now ready for the mini test and well-prepared for your upcoming university life!

If you are going to take Sho's lecture, \Emph{you can email your answers to Sho}.
Sho will look at it and give you feedback. This will \emph{not} be included in the grade evaluation, but Sho will acknowledge see your hard work and you might get some recognition for your effort.


\vfill

\noindent
\emph{P.S.~This is not an end, but just the beginning of your university learning. I mean, \textbf{just the beginning of the Derivative Boot Camp}. You will find more challenging problems and further extra topics.}

\end{document}
