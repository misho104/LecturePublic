%!TEX program=lualatex
%!TEX options=-synctex=1 -interaction=nonstopmode -halt-on-error "%DOC%.tex"
\def\LectureName/{General Physics}
%\def\LectureNumber/{}
%\def\LectureDate/{}
\PassOptionsToPackage{fleqn}{amsmath}
\PassOptionsToPackage{hyperfootnotes=false}{hyperref}
\documentclass[11pt,pdfa,lastpage]{MishoNote}

\newif\ifBasicForm\BasicFormtrue
\begin{luacode*}
  function j(s) return string.sub(tex.jobname, -#s) == s end
  if  j("_true") then tex.sprint("\\global\\BasicFormfalse") end
\end{luacode*}
\def\TitleShort/{\ifBasicForm{Derivative Boot Camp (Basic)}\else{Derivative Boot Camp}\fi}

\title{\LectureName/: \TitleShort/}
\author{Sho Iwamoto}
\hypersetup{
  pdflang={en-US},
  pdfauthortitle={Assistant Professor, National Sun Yat-sen University},
  pdfsubject={\TitleShort/ as an introduction (and an initiation) to the university. A part of \LectureName/ lecture.},
  pdfcontactemail={iwamoto@g-mail.nsysu.edu.tw},
  pdfcontacturl={https://www2.nsysu.edu.tw/iwamoto/},
  pdfcaptionwriter={Sho Iwamoto},
  pdfcopyright={2024–2025 Sho Iwamoto\textLF This document is licensed under the Creative Commons CC BY–NC 4.0 International Public License.},
  pdflicenseurl={https://creativecommons.org/licenses/by-nc/4.0/},
}

\usepackage{GP}
\setlist{itemsep=4pt}

\def\thesection{A}

\newcommand\INT[2][\relax]{\item$\displaystyle\int#2\,\dd x$\ifx\relax#1\relax\relax\relax\else{\quad#1}\fi}
\newcommand\INP[1]{\INT{\left(#1\right)}}
\newcommand\INC[3]{\item$\displaystyle\int^{#2}_{#1}#3\,\dd x$}
\NewDocumentCommand\DIFF{O{x}om}{\item$\displaystyle\gdiff[\dd][#2]{}{#1}#3$}

\let\origfootnote\footnote
\let\origfootnoterule\footnoterule

\makeatletter
\ifBasicForm
\RenewDocumentCommand\Quiz{o+m}{\new@problem{#1}{#2}}
\RenewDocumentCommand\Problem{o+m}{\new@problem{#1}{#2}}
\RenewDocumentCommand\new@problem{m+m}{%
  \ifx#1S\relax\relax%
  \gdef\MaybeLevel{\IfValueTF{#1}{\csname G#1\endcsname}{\relax}}\relax\item#2%
  \else\stepcounter{enumi}\relax\fi}
\fi
\makeatother


\begin{document}%
\title{\TitleShort/}
\begin{maketitle}
\let\footnote\origfootnote
\let\footnoterule\origfootnoterule

\subsection*{Preface}
Welcome to your first year of university!
As a university student in engineering, \Emph{you must be able to calculate derivatives of ``simple'' functions} such as
\ifBasicForm \[ \diff{}{x}\frac{x^2+\tan x}{\sin(2x+1)}. \]
\else        \[ \diff{}{x}\frac{x^2+\tan(\ln x)}{\sinh(2x^2+1)}. \] \fi
This Boot Camp is designed to help you prepare for your first year, which is unexpectedly tough for most of you!
Take your time, go through each problem carefully, and don't hesitate to ask for help!

\ifBasicForm\relax\else
 \par\smallskip
  The problems are classified into four categories:
  $\bigstar$ (minimal), $^{***}$ (basic), $^{**}$ (intermediate), and $^{*}$ (for motivated students).
  I guess you already have solved all the \emph{minimal} problems, which is sufficient as the very fresh university student.
  It is always nice to try to solve as many as you can because eventually you will have to learn them all, but recall that they are not essential for now and you can stop wherever you want.
 \par\smallskip
\fi

To motivate you, we will have a \Emph{mini test} at the beginning of the second lecture; the problems will be from \ifBasicForm{this Boot Camp}\else{the minimal problems of this boot camp}\fi.
I hope this preparation will make your university life easier, more enjoyable, and more satisfactory. Good luck!

\subsection*{Remarks}
Sho never provides you with solutions. \Emph{You students} need to make the solution. To this end,
\begin{miniitemize}
  \item Use online resources such as \hrefFN{https://www.wolframalpha.com/}{Wolfram Alpha}.
  \item Share your answers to other colleagues, using LINE or \hrefFN{https://docs.google.com/}{Google Docs}. Compare your answers with theirs.
  \item Ask questions to colleagues, to the TA, or to Sho. You can utilize Sho's \hrefFN{https://www2.nsysu.edu.tw/iwamoto/}{office hours}.
\end{miniitemize}

\OutputNote

\enlargethispage{-5em}

\makeatletter
\begin{tikzpicture}[remember picture,overlay]
  \begingroup
  \fontsize{9}{13}\selectfont
    \node[xshift=\@total@leftsep,yshift=25.5mm,anchor=south west,align=left,text width=\textwidth] at (current page.south west) {%
      \href{https://creativecommons.org/licenses/by-nc/4.0/}{\includegraphics[width=2.2cm]{../figs/by-nc.pdf}}\\[.4em]
      \noindent\textsf{\color{gray}%
      Visit \url{https://github.com/misho104/LecturePublic} for further information, updates, and to report issues.}\par
    };
    \node[xshift=-\@total@leftsep+26mm,yshift=32mm,align=left,text width=\textwidth,anchor=south east] at (current page.south east) {%
      \noindent\textsf{\color{gray}%
      This document is licensed under
      \href{https://creativecommons.org/licenses/by-nc/4.0/}{the Creative Commons CC--BY--NC 4.0 International Public License.}\\
      You may use this document only if you do in compliance with the license.}\par
    };
  \endgroup
\end{tikzpicture}
\makeatother

\end{maketitle}
\newpage
\subsection{The first step: High-school review}
First we review high-school mathematics, but with taking care of a typical pitfall.
Namely, some students are confused by the notation of derivatives.

Consider a function $f(x)$. The derivative of $f(x)$ is written by $f'(x)$, which is (usually) a different function from $f(x)$. For example, if $f(x)=x^2$, then $f'(x)=2x$.
We may express it in several ways:
\[ f'(x) = \diff{f}{x}(x) = \diff{f(x)}{x} = \diff{}{x}f(x); \]
these expressions are all equivalent. Don't be confused!

Sho recommends you to identify $\displaystyle \diff{f}{x} \stackrel!= f'$. Then, you will easily see that
\begin{itemize}
  \item $\displaystyle f'(x)=\diff{f}{x}(x)=\diff{f(x)}{x}=\diff{}{x}f(x)$ is a function,
  \item but we often omit the ``$(x)$'' part and write $\displaystyle f'$, $\displaystyle\diff fx$, or $\displaystyle\diff{}xf$.
  \item \emph{If $a$ is a constant}, $\displaystyle f'(a)=\diff{f}{x}(a)=\diff{}xf(a)$ means the value of $f'(x)$ at $x=a$.
\end{itemize}

\begin{problems}
  \Problem[S]{Calculate the following derivatives, where $a$ is a real constant and $f(x)=x^4$.
  \begin{menumerate}{3}
    \DIFF{x^2}
    \DIFF{(-4x^3)}
    \item$\displaystyle\diff[2]{}x x^5$
    \DIFF{\left(x^8+x^3\right)}
    \DIFF{\left(x+1\right)^3}
    \item $\displaystyle\diff fx$
    \item $\displaystyle\diff{}{x}f(1)$
    \item $f''(1)$\phantom{$\displaystyle\frac12$}
    \item $f''(a)$\phantom{$\displaystyle\frac12$}
  \end{menumerate}
  }
  \Problem[C]{Let $f(x)=x^5$ and $g(x)=(x+1)^5$, calculate the following expressions.
  It's nice if you can find the values in a clever way, with less calculation.
  \begin{menumerate}{5}
      \item $f'(x)$
      \item $f'(3)$
      \item $g'(2)$
      \item $g'(3)$
      \item $g''(5)$
  \end{menumerate}
  }
\end{problems}

\newpage

\subsection{Trigonometric functions}
Functions such as $\sin x$ and $\cos x$ are called \Emph{trigonometric functions}. For their argument $x$, we usually use \Emph{radians} instead of degrees. The degree $180^\circ$ is equal to $\pi$ radian, so
\begin{equation*}
  180\unit{deg} = 180^\circ = \pi\unit{rad},\qquad\text{i.e.,}\qquad 1\unit{rad}=\frac{180^\circ}\pi\approx57.30^\circ.
\end{equation*}
Furthermore, the unit ``rad'' is often omitted. Namely, we write
\begin{equation*}
  \cos\frac{\pi}{3} = \cos\left(\frac\pi3\unit{rad}\right) = \cos60^\circ = \frac12
\end{equation*}
and you need to get accustomed to this convention.

Be sure that the small circle $^\circ$ makes the meaning very different: it means ``degrees''. For example, $30^\circ$ is equivalent to $30\unit{deg}$ and $30$ is equivalent to $30\unit{rad}$:
\[
30^\circ = \frac\pi6 \qquad\text{and}\qquad 30 = \frac{1}{\pi}\times5400^\circ
\]
because $30\unit{deg}=(\pi/6)\unit{rad}$ and $30\unit{rad}=(30/\pi)(\pi\unit{rad})=(30/\pi)\times180\unit{deg}$.
In general,
\begin{equation}
 x^\circ=x\unit{deg}=\left(\frac{\pi x}{180}\unit{rad}\right)=\frac{\pi x}{180}.
 \end{equation}


\Remark{\Emph{Almost all students are confused by the notation}, $(\sin x^2)\neq(\sin x)^2$.
\begin{equation*}
  \text{Namely,}\quad\sin^2x~~=~~(\sin x)^2~~\neq~~\sin x^2~~=~~\sin(x^2).
\end{equation*}
}
\Remark{We should use this ``$\sin^kx$'' notation only for $k>0$. Sho suggests you to use specifically for positive-\emph{integer} $k$, such as $\cos^2x$ or $\tan^24\theta$. For negative or fractional exponents, it is preferable to use expressions like $(\sin x)^{3/2}$ or $(\sin x)^{-2}$:
\begin{equation*}
  (\sin x)^{-2}=\frac{1}{\sin^2x},\qquad(\tan x)^{-1/2}=\frac1{(\tan x)^{1/2}}=\frac{1}{\sqrt{\tan x}}=(\cot x)^{1/2},\quad\text{etc.}
\end{equation*}
Avoid ambiguous expressions.\addnote{Do not write $\sin(x)^2$. It will only confuse readers.}\addnote{%
You will soon learn ``inverse trigonometric functions'' such as~~~$\arcsin x$~~~and~~~$\arctan \theta$~. They are sometimes written as $\sin^{-1}x$ or $\tan^{-1}\theta$.}
}
\OutputNote

\begin{problems}
  \Problem[S]{Calculate the following values.
  \begin{menumerate}{4}
    \item $\displaystyle\sin\frac\pi6$
    \item $\displaystyle\sin\frac\pi4$
    \item $\displaystyle\sin\frac\pi3$
    \item $\displaystyle\cos\frac{2\pi}3$
    \item $\displaystyle\tan\frac\pi6$
    \item $\displaystyle\tan\frac{\pi}3$
    \item $\tan0$
    \item $\cos2\pi$
  \end{menumerate}
  }
 \Problem[A]{Calculate the following values.
  \begin{menumerate}{4}
    \item $\displaystyle\sin\frac{7\pi}6$
    \item $\displaystyle\tan\frac{8\pi}3$
    \item $\displaystyle\cos\frac{-3\pi}2$
    \item $\displaystyle\sin\frac{-5\pi}6$
    \item $\displaystyle\cos\frac{-8\pi}3$
    \item $\cos(-4\pi)$
    \item $\displaystyle\cos^2\frac\pi6$
    \item $\displaystyle\sin^2\frac\pi4$
  \end{menumerate}
  }
  \Problem[S]{Find the following values, using calculators or online resources.
  \begin{menumerate}{4}
    \item $\displaystyle\sin1$
    \item $\displaystyle\sin1^\circ$
    \item $\displaystyle\sin^23$
    \item $\displaystyle\sin3^2$
  \end{menumerate}
  }
   \Problem[A]{Calculate the following values using calculators.
  \begin{menumerate}{2}
    \item $\cos60$
    \item $\cos\pi^2$
    \item $\sin0.0000123$
    \item $\tan0.0000777$
   \end{menumerate}
   }
\end{problems}

\ifBasicForm\else\pagebreak\fi

\subsection{Get into the University}
Recall that derivatives are \Emph{defined by}
$\displaystyle\diff{f}{x}(x_0)=\lim_{x\to x_0}\frac{f(x)-f(x_0)}{x-x_0}$, or equivalently,
\begin{equation}
\diff{}{x}f(x)=\lim_{h\to 0}\frac{f(x+h)-f(x)}{h}.\label{eq:diff}
\end{equation}
Any derivatives can be calculated based on this definition. For example, with $f(x)=x^2$,
\begin{equation}
  \diff{}{x}x^2=\lim_{h\to 0}\frac{(x+h)^2-x^2}{h}=\lim_{h\to 0}\frac{2xh+h^2}{h}=\lim_{h\to 0}(2x+h)=2x.
  \label{eq:diffdefex}
\end{equation}

\def\GrayBox#1#2{#1\fcolorbox{black}{lightgray}{\phantom{#2}}#1}
\def\GrayBoxTall{\GrayBox{~}{\phantom{$\frac12$xxxxxxxxx}}}
\begin{problems}
\Problem[S]{
  We can calculate other derivatives by the same method as in \eqref{eq:diffdefex}. Fill the blanks.
  \begin{enumerate}
    \item$\displaystyle\diff{}{x}x^3
    =\lim_{h\to 0}\frac{\GrayBox{~~}{xxxxxxxxx}}{h}
    =\lim_{h\to 0}\frac{\GrayBox{~~}{xxxxx}}{\GrayBox{}{xx}}
    =\lim_{h\to 0}\left(\GrayBox{~}{xxxxxxx}\right)
    =3x^2.$
    \item$\displaystyle\diff{}{x}4x
    =\lim_{h\to 0}\frac{\GrayBox{~~}{xxxxxxxxx}}{\GrayBox{}{xx}}
    =\lim_{h\to 0}\frac{\GrayBox{~~}{xxxxx}}{\GrayBox{}{xx}}
    =\lim_{h\to 0}\left(\GrayBox{~}{xxxxxxx}\right)
    =4.$
\end{enumerate}}
 \Problem[C]{As the previous problem, fill the blanks. You may use the equation $\displaystyle\sqrt{a}-\sqrt{b}=\dfrac{a-b}{\sqrt{a}+\sqrt{b}}$.
  \begin{enumerate}
    \item$\displaystyle\diff{}{x}\frac1x=\GrayBoxTall=\GrayBoxTall=\GrayBoxTall=-\frac{1}{x^2}$
    \item$\displaystyle\diff{}{x}x^{-2}=\GrayBoxTall=\GrayBoxTall=\GrayBoxTall=-\frac{2}{x^3}$
    \item$\displaystyle\diff{}{x}\sqrt{x}=\GrayBoxTall=\GrayBoxTall=\GrayBoxTall=\frac{1}{2\sqrt x}$
  \end{enumerate}
 }
  \Problem[S]{We know that a function $f(x)$ satisfies an equation $f'(x)=4x$. Can you find what $f(x)$ is? Is it unique? Or can we find more than one possibilities?
  }
  \Problem[B]{Answer the following problem. Some problems have more than one solutions; try to find whether it has only one solution or more than one solutions. (What are the other solutions if exist?)
    \begin{enumerate}
    \item We know $g(x)$ satisfies $g'(x)=x$. Find $g(x)$. If you can, find more than one.
    \item We know $h(x)$ satisfies $h'(x)=x$ and $h(0)=4$. Find $h(x)$.
    \item We know $F(x)$ satisfies $F''(x)=1$. Find $F(x)$.
    \item We know $G(x)$ satisfies $G''(x)=1$ and $G'(0)=2$. Find $G(x)$.
    \item We know $H(x)$ satisfies $H''(x)=1$, $H'(0)=2$, and $H(0)=k$, where $k$ is a real constant. Find $H(x)$.
    \item We know $\mathcal F(x)$ satisfies $\mathcal F''(x)=a$, $\mathcal F'(0)=b$, and $\mathcal F(0)=c$, where $a$, $b$, and $c$ are real constants. Find $\mathcal F(x)$.
    \end{enumerate}}
\end{problems}

\ifBasicForm\pagebreak\else\fi

\subsection{The formulae you need to memorize}
Now it's time for more complicated functions. First, you need to memorize the following formulae.
I mean, \emph{you need to do exercise until you've memorized them and can use them without any hesitation}.

\begin{theorem}{}{}
  Derivatives of trigonometric functions are given by
  \begin{align}
   \diff{}{x}\sin x &= \cos x,&
   \diff{}{x}\cos x &= -\sin x,&
   \diff{}{x}\tan x &= \frac1{\cos^2x}.\label{eq:trig}
  \end{align}
\end{theorem}
\begin{theorem}{}{}
  Derivatives of the reciprocal, product, and quotient of a function(s) are given by
  \begin{align}
   &\label{eq:reci} \diff{}{x}\frac{1}{f(x)}=-\frac{f'(x)}{[f(x)]^2},\\[0.5em]
   &\label{eq:prod} \diff{}{x}\Bigl[f(x)g(x)\Bigr]=f'(x)g(x)+f(x)g'(x),\\[0.5em]
   &\label{eq:quot} \diff{}{x}\frac{g(x)}{f(x)}=\frac{f(x)g'(x)-f'(x)g(x)}{[f(x)]^2},
   \end{align}
   where  $f(x)$ and $g(x)$ are any differentiable functions.
\end{theorem}

\begin{problems}
  \Problem[S]{
  Use online resources to check that these are correct (i.e., Sho didn't make any typo).}
  \Problem[S]{Calculate the following derivatives.
  \begin{menumerate}{3}
    \DIFF{3\sin x}
    \DIFF{\left(2\cos x-\sin x\right)}
    \DIFF{\left(1+x+\tan x\right)}
  \end{menumerate}}
  \Problem[S]{Calculate the following derivatives, using Eqs.~\eqref{eq:reci}--\eqref{eq:quot}.
  \begin{menumerate}{3}
    \DIFF{(x+1)(x^2 + 2)}
    \DIFF{x  \sin x}
    \DIFF{(3x^2 + 2x + 1)^2}
    \DIFF{\sin^2 x}
    \DIFF{\frac{1}{x+1}}
    \DIFF{\frac{1}{\tan x}}
    \DIFF{\frac{x^2+1}{x+1}}
    \DIFF{\frac{\sin x}{x}}
    \DIFF{\frac{\sin x}{\cos x}}
  \end{menumerate}}
\end{problems}
\ifBasicForm\else
\begin{problems}
 \Problem[C]{We can prove all the above theorems, \eqref{eq:trig}--\eqref{eq:quot}. It is not easy, but let's try!
 \begin{enumerate}
  \item Check that $\displaystyle 1+\cos\theta = \frac{\sin^2\theta}{1-\cos\theta}$. You may need this in the next question.
  \end{enumerate}
  To prove \eqref{eq:trig}, we need to use $\displaystyle\lim_{x\to 0}\frac{\sin x}{x}=1$. However, its proof is nontrivial and related to a deeper issue in the definition of sine functions. So, in this Boot Camp, we accept this equation.
  \begin{enumerate}[resume]
  \item Prove $\displaystyle\diff{}{x}\sin x=\cos x$ and $\displaystyle\diff{}{x}\cos x=-\sin x$ by the same method as Problem [H], using $\displaystyle\lim_{x\to 0}\frac{\sin x}{x}=1$.
  \end{enumerate}
  Next, we prove Eqs.~\eqref{eq:reci}--\eqref{eq:quot}. First, let's assume Eqs.~\eqref{eq:reci} and \eqref{eq:prod} are correct.
  \begin{enumerate}[resume]
  \item Prove Eq.~\eqref{eq:quot} from Eqs.~\eqref{eq:reci} and \eqref{eq:prod}.
  \end{enumerate}
  All that remains is to show Eqs.~\eqref{eq:reci} and \eqref{eq:prod}. So,
  \begin{enumerate}[resume]
  \item Prove Eqs.~\eqref{eq:reci} and \eqref{eq:prod} by the method of Problem [H]. \Hint{This is not easy.}
  \end{enumerate}}
\end{problems}
\fi

\pagebreak

\subsection{Workout 1: Practice!}
\ifBasicForm Practice makes perfect!
\else        Practice makes perfect, so I can provide you with as many problems as you want! \fi
\begin{problems}
\Problem[S]{Practice for the formulae~\eqref{eq:reci} and~\eqref{eq:prod}.
\begin{menumerate}{3}
  \DIFF{\frac{1}{\sin x}}
  \DIFF{\frac{1}{3x^2+1}}
  \DIFF{\frac{1}{x^2+2x+1}}
  \DIFF{(x-1)(\cos x + x)}
  \DIFF{\sin x\cos x}
  \DIFF{\cos x\tan x}
  \DIFF{(x^2 + 1)^2}
  \DIFF{x^5\cos x}
  \DIFF{(x^2\cdot x^3)}
\end{menumerate}}
\Problem[S]{Practice for the formulae~\eqref{eq:reci} and~\eqref{eq:quot}. Here, $n$ is a positive integer.
\begin{menumerate}{3}
  \DIFF{\frac{x+1}{x-1}}
  \DIFF{\frac{x+1}{x^2+1}}
  \DIFF{\frac{x^2-1}{x-1}}
  \DIFF{\frac{\sin x}{x^2}}
  \DIFF{\frac{1-\cos x}{1+\cos x}}
  \DIFF{\frac{x^3+1}{x^2+1}}
  \DIFF{\frac{1}{x^n}}
  \DIFF{x^{-3}\cos x}
  \DIFF{\frac{x^3-1}{x-1}}
\end{menumerate}
  \hfill\Hint[Advanced note:~]{Did you notice a better way to calculate (3) and (9)?}}
\Problem[A]{Practice more. Here, $n$ is a positive integer.
\begin{menumerate}{3}
  \DIFF{\tan^2x}
  \DIFF{(x^2 + 3x)(4x^3 - 2x)}
  \DIFF{\frac{2x + 1}{x^4}}
  \DIFF{\frac{x+\sin x}{x+\cos x}}
  \DIFF{(\sin x + \cos x)^2}
  \DIFF{\frac{x+1}{x \cos x}}
  \DIFF{\frac{x^4 + x^2 + 1}{x^3 + x}}
  \DIFF{\frac{\cos x}{x^3}}
  \DIFF{x^{-n}\sin x}
\end{menumerate}}
\Problem[B]{Use the formula \eqref{eq:prod} repeatedly to calculate the following derivatives.
\begin{menumerate}{2}
  \DIFF{\left(x\cdot\tan x \cdot\sin x\right)}
  \DIFF{(x^2 + 1)(x+1)(x+2)}
  \DIFF{(x^2+1)^3}
  \DIFF{x(\sin x + \cos x)^2}
  \DIFF{\frac{\sin^2 x}{\cos x}}
  \DIFF{(x^2+1)\sin^2x\tan x}
\end{menumerate}}
\end{problems}

\ifBasicForm\else\pagebreak\fi

\subsection{One more formula}
The formula you know well, $(x^n)'=nx^{n-1}$, can be generalized to any real number $a$.
\begin{theorem}[parbox=false]{}{}
  \begin{equation}
  \text{For \Emph{any real number}}~a,\qquad \diff{}{x}x^a = ax^{a-1}.
  \end{equation}
\end{theorem}
Notice you can use this formula for $a=1/2$, $a=-1$, $a=-3/2$, or even $a=0$.

\begin{problems}
\Problem[S]{Calculate the following derivatives based on the theorem above.
\begin{menumerate}{4}
  \DIFF{x^{64}}
  \DIFF{x^{-10}}
  \DIFF{x^{-2}}
  \DIFF{\frac{1}{x}}
  \DIFF{\sqrt x}
  \DIFF{x^{1/3}}
  \DIFF{\frac{1}{x^{1/3}}}
  \DIFF{\frac{1}{\sqrt x}}
  \DIFF{\frac{1}{x\sqrt x}}
  \DIFF{x^0}
  \DIFF{x^{0.1}}
  \DIFF{x^{4\pi}}
\end{menumerate}}
\end{problems}

\ifBasicForm\pagebreak\else\fi

\subsection{The last step: Composite functions}
Now, we consider \Emph{composite functions}, which are functions of functions. For example, consider $f(x)=\sin(x^2)$. Its derivative can be calculated with the next theorem.
\begin{theorem}{}{}
  Consider $f(u)$, which is a function of $u$. Assume $u=u(x)$ is a function of $x$. Then, we can consider $f(u)$ as a function of $x$, i.e., $f(x)=f(u(x))$, and
  \begin{equation}
   \diff{f}{x}=\diff{f}{u}\diff{u}{x}.
  \end{equation}
\end{theorem}
\noindent
This theorem is complicated but you need to get accustomed to. For example,
  \begin{itemize}
    \item For $f(x)=\sin(x^2)$, we set $u=x^2$ and $f(u)=\sin u$. Then, $\diff fu=\cos u$ and $\diff ux=2x$, which lead to the conclusion $\diff fx=(\cos u)\cdot2x=2x\cos x^2$.
    \item Consider $g(x)=(x^2+2x+1)^4$. We use the theorem with $f(u)=u^4$ and $u=x^2+2x+1$. Then, $\diff fu=4u^3$ and $\diff ux=2x+2$, which lead to $\diff fx=4(x^2+2x+1)^3\cdot(2x+2)$.
    \\\relax
    [Notice that this is equal to $8(x+1)^7$.]
    \item This theorem helps us a lot. For example, the derivative of the function $(x^2+1)^3$ can be easily calculated, with $u=x^2+1$, as $3u^2\cdot 2x=6x(x^2+1)^2$.
  \end{itemize}
You can calculate more complicated functions. One example is $\cos(\sin x)$. If we let $u=\sin x$,
  \[
  \diff{}{x}\cos(\sin x)=\diff{\cos{u}}{x}=\diff{\cos{u}}{u}\cdot\diff ux=-\sin u\cdot\cos x=-\sin(\sin x)\cdot\cos x.
  \]

\begin{problems}
  \Problem[S]{Practice.
  \begin{menumerate}{3}
    \DIFF{(x^2+1)^3}
    \DIFF{(x^2+2x+1)^4}
    \DIFF{\sin x^4}
    \DIFF{\sin^4x}
    \DIFF{\sqrt{x+1}}
    \DIFF{\sqrt{\tan x}}
    \DIFF{\sin x^{-2}}
    \DIFF{\cos \sqrt{x}}
    \DIFF{(\sin x)^{-2}}
    \DIFF{(\cos x)^{-1/2}}
    \DIFF{\frac1{\sqrt{\tan x}}}
  \end{menumerate}}
  \Problem[B]{Practice with the following problems, which are a bit tough.
  \begin{menumerate}{3}
    \DIFF{\cos(\sin x)}
    \DIFF{\sin(3x^2 + 2x)}
    \DIFF{\sqrt{5x^2 + 7x}}
    \DIFF{\frac1{\sqrt{5x^2 + 7x}}}
    \DIFF{\sin\sqrt{x}}
    \DIFF{(3x^2+ 1)^{5/2}}
    \DIFF{\tan(6x^2 - 5x)}
    \DIFF{\tan x^3}
    \DIFF{\tan(\tan x)}
  \end{menumerate}}
\end{problems}

You will also be asked to combine with the formulae you've learned so far. For example,
  \begin{align*}
  \diff{}{x}\frac{x}{\cos(x^2+1)}
  &= \frac{(x)'\cos(x^2+1)-x\left[\cos(x^2+1)\right]'}{\left[\cos(x^2+1)\right]^2}
  = \frac{\cos(x^2+1)-x(-\sin u)(u)'}{\cos^2(x^2+1)}
  \\&= \frac{\cos(x^2+1)+x\cdot 2x\cdot \sin (x^2+1)}{\cos^2(x^2+1)} = \frac{1+2x^2\tan(x^2+1)}{\cos(x^2+1)}
  \end{align*}
where we let $u=x^2+1$.

\ifBasicForm\pagebreak\fi

\subsection{Workout 2: Practice, Practice, Practice!}
Once you have learned, you need to get accustomed to the calculation. To this end, you need to practice more.
Some of the following may be a bit complicated, but you can solve them by combining the previous formulae.
If you are lost, try using online resources, ask your colleagues, or ask Sho.

\begin{problems}
  \Problem[S]{Practice more.
\begin{menumerate}{3}
  \DIFF{(5x^2 - 2x + 1)}
  \DIFF{(x^4 + \sqrt{x})}
  \DIFF{(\sin x + \cos x)}
  \DIFF{(\tan x + \sqrt{2x})}
  \DIFF{\frac{x^2 + 1}{x - 1}}
  \DIFF{\left(\frac{1}{x} + x^2\right)}
  \DIFF{\sqrt{x}\cos x}
  \DIFF{x^2\sin x}
  \DIFF{\cos x^2}
  \DIFF{\cos^2 x}
  \DIFF{(x-1)^{-1/2}}
  \DIFF{\frac 1{\cos x}}
  \DIFF{\frac x{\cos x}}
  \DIFF{\frac {\cos x^2}{x^2+1}}
  \DIFF{\frac x{\sqrt{\cos x}}}
\end{menumerate}}
\ifBasicForm\else\end{problems}\begin{problems}\fi

\Problem[A]{\textsf{(intermediate-level problems)}
\begin{menumerate}{3}
  \DIFF{(5x^2 - 2x + 1)^3}
  \DIFF{\sqrt{x^4+1}}
  \DIFF{\sin(x^2+2x+2)}
  \DIFF{\tan\sqrt{2x}}
  \DIFF{\frac{(x + 1)^2}{(x - 1)^2}}

  \DIFF{\left(\frac{1}{x} + x\right)^2}
  \DIFF{x^{-1/2}\cos x}
  \DIFF{x^2\tan2x}
  \DIFF{(x - 1)^{-1/4}}
  \DIFF{\frac{2}{(x - 1)^{1/4}}}

  \DIFF{\frac {x^3}{\cos x}}
  \DIFF{\frac {x}{\cos2x}}
  \DIFF{\frac {x^2}{\cos x^2}}
  \DIFF{\cos(x^2+1)^2}
  \DIFF{\tan^2(x^2+1)}
\end{menumerate}}

\Problem[B]{\textsf{(a bit tough problems)}
\begin{menumerate}{3}
  \DIFF{(5x^2 - 2x + 1)^{-3}}
  \DIFF{(x^4+1)^{-3/5}}
  \DIFF{\sin^2[(x^2+2x)^2]}
  \DIFF{\tan \left(x+\sqrt{2x}\right)}
  \DIFF{\frac{\sin(x^2 + 1)}{\sin(x - 1)}}

  \DIFF{\frac{1}{\sqrt{x^2 + 4}}}
  \DIFF{x^{-1/2}\cos x^2}
  \DIFF{x^2\sin x\tan2x}
  \DIFF{x(x - 1)^{-3/4}}
  \DIFF{\frac{2\sin x}{(x - 1)^{3/4}}}

  \DIFF{\frac {x^3\tan x}{\cos x}}
  \DIFF{\frac {x\sin x}{\cos2x}}
  \DIFF{\frac {x^2\sin^2x}{\cos x^2}}
  \DIFF{\frac{x^3 + 1}{\sqrt{x - 1}}}
  \DIFF{\tan^2\left(x \sqrt{x}\right)}
\end{menumerate}}
\end{problems}

\vfill
\subsection*{Afterwords}

\ifBasicForm
Have you finished all the problems? Great job! You're now ready for the mini test and well-prepared for your upcoming university life!

If you are going to take Sho's lecture, \Emph{you can send your answers/questions to Sho} via an email (\url{mailto:iwamoto@g-mail.nsysu.edu.tw}).
Sho will look at it and give you feedback. It will \emph{not} be included in the grade evaluation, but Sho will acknowledge your hard work and you might get some recognition for your effort.


\vfill

\noindent
\emph{By the way... Did you notice the problem numbering had some gaps...? Yes, this is not an end, but just the beginning of your university learning. I mean, \textbf{just the beginning of the Derivative Boot Camp} and many problems are hidden in this document. You will find more challenging problems and further extra topics.}

\else

Have you finished all the problems? Great job! You're now 100\% ready for your university learning!

If you are going to take Sho's lecture, \Emph{you can send your answers/questions to Sho} via an email (\url{mailto:iwamoto@g-mail.nsysu.edu.tw}).
Sho will look at it and give you feedback. It will \emph{not} be included in the grade evaluation, but Sho will acknowledge your hard work and you might get some recognition for your effort.


\vfill

\noindent
\emph{By the way... This is the end of this Boot Camp, but just the beginning of your learning. I mean, \textbf{I am ready to provide motivated students with more advanced tasks.} Send me an email if you are interested.}
\fi
\end{document}
