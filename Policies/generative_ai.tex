%!TEX program=lualatex
%!TEX options=-synctex=1 -interaction=nonstopmode -halt-on-error "%DOC%.tex"
\documentclass[11pt,pdfa,lastpage,minititle]{MishoNote}
\title{Guidelines for Using Generative AI in Sho's Courses (Aug.~2025)}
\author{Sho Iwamoto}
\hypersetup{
  pdflang={en-US},
  pdfauthortitle={Assistant Professor, National Sun Yat-sen University},
  pdfsubject={Guidelines for Using Generative AI (August 2024 version) applied to Sho's lecture courses at National Sun Yat-sen University.},
  pdfcontactemail={iwamoto@g-mail.nsysu.edu.tw},
  pdfcontacturl={https://www2.nsysu.edu.tw/iwamoto/},
  pdfcaptionwriter={Sho Iwamoto},
  pdfcopyright={2024–2025 Sho Iwamoto},
  pdflicenseurl={https://www2.nsysu.edu.tw/iwamoto/},
}
\begin{document}
\maketitle
Generative AIs (GAIs), such as ChatGPT, are powerful tools that can assist in many tasks%
\addnote{Previous versions of this document are available. You may notice how much the situation has changed due to the rapid improvement of GAIs.\\\url{https://github.com/misho104/LectureCommon/commits/main/Notes/generative_ai.pdf}
}%
\addnote{This document was prepared with the assistance of ChatGPT. However, this kind of remark is no longer meaningful since we always use GAIs.}.
They have improved greatly in recent years.
% Today, using them properly is not optional but an essential skill, especially for university students.


Unfortunately, GAIs will make your life more difficult.
The situation is no longer ``we can use ChatGPT to improve our work.'' In 2025, it has become ``we must use GAIs to maintain the quality of our work.''
Put simply, \Emph{our output must be better than what GAIs can generate. Otherwise, people will not want to work with us, and companies will have no reason to hire us.}

You must therefore learn how to use GAIs effectively. Overreliance will weaken your own thinking, while ignoring them entirely will leave you unable to compete with others.

This guideline is based on Sho's opinion to maximize your academic success, applying as a rule to all assignments you submit to Sho.
Violations may result in \Emph{reduced, zero, or negative points}\addnote{%
  It is impossible to detect GAI use with 100\% confidence. On the other hand, Sho reserves the right to give you reduced or negative score for your submission.}.


\subsection*{Regulations on Coding Assignments}
Since August 2025, Sho's programming assignments assume that students use GAIs. Sho evaluates your work on the assumption that you have used GAIs. As a rule,
\begin{miniitemize}
\item You are \emph{encouraged} to use GAIs. However, \emph{you must be able to explain each line of your code.}
      Sho may ask you to explain your code; if you fail to do so adequately, you may be penalized.
\end{miniitemize}

\subsection*{Regulations on Other Assignments}
For other assignments, Sho assumes that you \emph{do not} use GAIs. As a rule,
\begin{miniitemize}
 \item You are \emph{not forbidden} to use GAIs. However, if your submission relies heavily on them, Sho may assign zero or even negative points.
\end{miniitemize}
For example, assume you simply copy-paste GAI output, others do the same, and your submission is similar to theirs. This clearly shows a lack of effort or creativity, and you may be penalized.

\subsection*{Other Remarks}
As of August 2025, GAIs are really useful for coding or language-related tasks.
Meanwhile, in scientific fields, they are still poor and often produce entirely incorrect answers.
It is beneficial for you to gain experience through proper, critical use.

The following resources may be useful. However, due to the rapid development of GAIs, documents written in 2022--2024 now seem obsolete.
\begin{miniitemize}
 \item \href{https://teaching.ucla.edu/resources/teaching-guides/using-generative-ai-reflectively-and-responsibly-in-teaching-and-learning/}{Using Generative AI Reflectively and Responsibly in Teaching and Learning} (UCLA)
 \item \href{https://about.open.ac.uk/policies-and-reports/policies-and-statements/gen-ai/generative-ai-students}{Generative AI for Students} (Open University)
 \item \href{https://info.atcoder.jp/entry/llm-rules-en}{AtCoder Rules against Generative AI --- Version 20241206}
 \item \href{https://utelecon.adm.u-tokyo.ac.jp/docs/ai-tools-in-classes}{\JA{AIツールの授業における利用について}} (a guideline in the University of Tokyo)
\end{miniitemize}
\OutputNote

\end{document}
